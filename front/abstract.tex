\chapter*[Abstract]{}

\begin{center}
  Abstract

  \vspace{1.5\baselineskip}
  \MyTitle\\
  \vspace{\baselineskip} by\\
  \vspace{\baselineskip} \MyAuthor\\
  Doctor of Philosophy in Physics\\
  University of California, Berkeley\\
  Professor Kam-Biu Luk, Chair
  \vspace{\baselineskip}
\end{center}

\noindent
%
The Daya Bay reactor neutrino experiment measures the oscillation of electron antineutrinos produced by the Daya Bay and Ling Ao-I \& II Power Plants (totaling 17.4~GW$_\mathrm{th}$) in southern China. The fluxes and spectra of the antineutrinos are observed using eight functionally identical gadolinium-doped liquid scintillator detectors, with a total target mass of 160~tons, divided among two near and one far experimental halls, resulting in baselines of order 500~m (near) and 2~km (far). The oscillation phenomenon appears as a deficit of electron antineutrinos at the far hall relative to the yield at the near halls.

This work describes a spectral measurement of the neutrino mixing parameters $\SinSq$ and $\Dmsqee$ using a six-year sample of over 2~million inverse beta-decay (IBD) events identified via the outgoing neutron's subsequent capture on gadolinium. A defining feature of this analysis is its ability to be efficiently repeated for arbitrary choices of the \emph{IBD cuts}, that is, the criteria used for selecting IBD events. This required the implementation of a flexible and efficient IBD selection, which was coupled to a parallelized upgrade of the official Daya Bay oscillation fitter developed at Lawrence Berkeley National Laboratory. The generalization of the analysis to arbitrary IBD cuts (beyond the two nominal IBD cuts used in official Daya Bay results) involved studies of various efficiencies and backgrounds and their dependence on the cuts, which we describe here.

With this infrastructure, we are able to explore how the best-fit oscillation parameters and their measured uncertainties vary as functions of the cuts, enabling, first, the characterization of the systematic uncertainty associated with the freedom to vary the cuts, and second, the exploration of whether the nominal IBD cuts should be modified in order to gain a reduced uncertainty on the oscillation parameters. We find that, compared to the two nominal cuts, there is no alternative IBD cut that provides a substantial reduction in uncertainty. The cut-variation study suggests the assessment of an additional systematic uncertainty of 0.0006 on $\SinSq$ and $1.9\times10^{-5}$~eV$^2$ on $\Dmsqee$, leading to a 2\% increase in the total uncertainty of $\SinSq$ and a 4\% increase in that of $\Dmsqee$. Our final measurement is $\SinSq = 0.0850 \pm 0.0030$ and $\Dmsqee = (2.5010 \pm 0.0072) \times 10^{-3}$~eV$^2$ in the standard three-flavor model.

The methods developed here serve to demonstrate that our oscillation fit is both robust and optimal with respect to variations in the cuts. Given the overall similarity between our oscillation analysis and those used in official Daya Bay results, similar conclusions can be drawn for the latter. Our findings thus validate past Daya Bay results, while future results will be able to benefit from this groundwork, enabling additional validation of the results and broadened assessment of their systematic uncertainties.

% spectral measurement proceeding from raw data through calibration, reconstruction, event selection, background subtraction, and finally to a fit that yields $\sin^2 2\theta_{13} = 0.849 \pm 0.00287$ and $\Delta m^2_{ee} = (2.47 \pm 0.0715) \times 10^{-5}$~eV$^2$.
%
% In addition, we perform a 2D search for a light sterile neutrino, setting stringent limits on $\sin^2 2\theta_{14}$ for $10^{-5} < \Delta m^2_{41} < 10^{-2}$. Compared to previous Daya Bay analyses, a novel feature of this work is the comprehensive variation of event selection criteria, providing, for the first time, a quantitative demonstration that the analysis is stable with respect to such variations and that there is no need to alter the standard criteria for the sake of minimizing the uncertainty on the oscillation parameters.

\SingleSpacing
\vspace{2.5\baselineskip}
\hfill
\begin{minipage}{0.4\textwidth}
  \hrule\vspace{0.4\baselineskip}
  Professor Kam-Biu Luk\\
  Dissertation committee chair
\end{minipage}

\clearpage
