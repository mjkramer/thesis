\MyDoubleSpacing

\begin{center}
  \mylarge
  \textbf{Abstract}

  \vspace{1.5\baselineskip}
  Precision Measurement of Reactor Neutrino Oscillation with Daya Bay\\
  \vspace{\baselineskip} by\\
  \vspace{\baselineskip} Matthew Jacob Kramer\\
  Doctor of Philosophy in Physics\\
  University of California, Berkeley\\
  Professor Kam-Biu Luk, Chair
  \vspace{\baselineskip}
\end{center}

\normalsize
\MyDoubleSpacing
\noindent
%
The Daya Bay Experiment measures the oscillation of electron antineutrinos
produced by the Daya Bay and Ling Ao-I \& II Power Plants (totaling 17.4
GW$_\mathrm{th}$) in southern China. The fluxes and spectra are observed using
eight functionally identical gadolinium-doped liquid scintillator detectors,
with a total target mass of 160 tons, divided among two near and one far
experimental halls, resulting in baselines of order 500~m (near) and
2~km (far). Using a sample of over 2 million events from $\sim$6 years of livetime,
this work describes a spectral measurement proceeding from raw data through
calibration, reconstruction, event selection, background subtraction, and
finally to a fit that yields $\sin^2 2\theta_{13} = 0.845 \pm
0.06\,\mathrm{(syst.)} \pm 0.04\,\mathrm{(stat.)}$ and $\Delta m^2_{ee} = [ 2.34
\pm 0.23\,\mathrm{(syst.)} \pm 0.34\,\mathrm{(stat.)} ] \times 10^{-5}$~eV$^2$.
%In addition, we perform a 2D search for a light sterile neutrino, setting stringent limits on $\sin^2 2\theta_{14}$ for $10^{-5} < \Delta m^2_{41} < 10^{-2}$.
Compared to previous Daya Bay analyses, a novel feature of this work is
the comprehensive variation of event selection criteria, providing, for the first time, a quantitative demonstration that the analysis is stable with respect to such variations and that there is no need to alter the standard criteria for the sake of minimizing the uncertainty on the oscillation parameters.

\SingleSpacing
\vspace{2.5\baselineskip}
\hfill
\begin{minipage}{0.4\textwidth}
  \hrule\vspace{0.4\baselineskip}
  Professor Kam-Biu Luk\\
  Dissertation committee chair
\end{minipage}

\clearpage