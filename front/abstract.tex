\MyDoubleSpacing

\begin{center}
  \mylarge
  \textbf{Abstract}

  \vspace{1.5\baselineskip}
  Precision Measurement of Reactor Neutrino Oscillation with Daya Bay\\
  \vspace{\baselineskip} by\\
  \vspace{\baselineskip} Matthew Jacob Kramer\\
  Doctor of Philosophy in Physics\\
  University of California, Berkeley\\
  Professor Kam-Biu Luk, Chair
  \vspace{\baselineskip}
\end{center}

\normalsize
\MyDoubleSpacing
\noindent
%
The Daya Bay experiment measures the oscillation of electron antineutrinos
produced by the Daya Bay and Ling Ao-I \& II power plants (totaling 17.4
GW$_\mathrm{th}$) in southern China. The flux and spectrum are observed using
eight functionally identical Gadolinium-doped liquid scintillator detectors,
with a total target mass of 120~T, divided among two near and one far
experimental halls, resulting in varying baselines of order 500~m (near) and
2~km (far). Using a sample of over 2~M events from $\sim$6 years of livetime,
this work describes a spectral measurement of the mixing angle $\theta_{13}$ and
effective mass splitting $\Delta m^2_{ee}$, proceeding from raw data through
calibration, reconstruction, event selection, background subtraction, and
finally to a fit that yields $\sin^2 2\theta_{13} = 0.845 \pm
0.06\,\mathrm{(syst.)} \pm 0.04\,\mathrm{(stat.)}$ and $\Delta m^2_{ee} = 2.34
\pm 0.23\,\mathrm{(syst.)} \pm 0.34\,\mathrm{(stat.)} \times 10^{-5}$~eV$^2$. In
addition, we perform a 2D search for a light sterile neutrino, setting stringent
limits on $\sin^2 2\theta_{14}$ for $10^{-5} < \Delta m^2_{41} < 10^{-2}$.
Compared to previous Daya Bay publications, a novel feature of this analysis is
the joint optimization of event selection criteria, resulting in a 42$\%$
average reduction in the uncertainty of the measured oscillation parameters.

\SingleSpacing
\vspace{2.5\baselineskip}
\hfill
\begin{minipage}{0.4\textwidth}
  \hrule\vspace{0.4\baselineskip}
  Professor Kam-Biu Luk\\
  Dissertation committee chair
\end{minipage}

\clearpage