\chapter*{Preface}
\addcontentsline{toc}{chapter}{Preface}

In order to provide the reader with a reasonably clear understanding of the complete path from reactors and detectors to Daya Bay's raw data and then to our final results, we discuss each step in sequence, aiming to give at least a minimal amount of detail. This author has contributed to a number of these steps, but given the complexity of the experiment and its analysis, our results are necessarily built upon the work of the many others who have contributed to Daya Bay over the years. This thesis focuses primarily on our high-level analysis work, as detailed mainly in \Autoref{chap:selection,chap:bkg,chap:accDMC,chap:cutVary}. Discussions of hardware and lower-level analysis are provided for completeness, but kept relatively concise, with additional details provided for some of the areas in which this author contributed personally. Certain elements of the high-level analysis, including the reactor model, the fitting framework, and some of the subdominant background predictions, are adopted from the work of others, to which citations are provided. For the interested reader, we review these subjects in the Appendix.

% Collaboration is an essential feature of science, and this is certainty true when it comes to Daya Bay. The results in this thesis would have been impossible without the efforts of many individuals who have contributed to the experiment (and its analysis) over the years. In order to present a coherent whole, this thesis attempts to adequately detail every relevant feature of the experiment, and every step along the journey from raw data to final result. Some of those details represent this author's original work, while others are the work of fellow collaborators. In the text, we have aimed to be clear about which contributions are our own, and for those that aren't, we provide references. For the sake of any reader who wishes to dive into this author's original work, we highlight here some of the author's personal contributions to the experiment and analysis, with references to further discussion in the text.

Working on Daya Bay has given this author the opportunity to contribute to a broad range of the pieces that together form our final results. As a summary of this work, we proceed to list a number of our main contributions, along with references to corresponding sections of the text.\footnote{The order of this list roughly follows that of the text.}

\begin{itemize}
\item Onsite participation in PMT testing and installation, AD assembly, detector commissioning, and data-taking.
\item Assistance in development of the algorithm for calculating the amount of calibrated charge for each PMT hit, particularly by studying the response of the electronics to multiple hits spaced closely in time.
\item Implementation of the algorithm for calculating the calibrated time of each PMT hit, and preparation of the calibration constants.
\item Implementation and validation (but not the original development) of the time-dependent nonuniformity correction for the energy reconstruction.
\item Assistance in generation of calibration constants for the energy reconstruction.
\item A novel derivation of the time-to-last-muon distribution for $^9$Li/$^8$He backgrounds, as well as early studies of the rate of this background (the code from which was extended and used for Daya Bay's official results), and later reevaluations and generalizations of the rate measurement.
\item Development of Monte Carlo methods for predicting the muon veto efficiency from the rates of different classes of muons.
\item A novel method for analytically calculating the singles rate (and, by extension, the accidentals rate and multiplicity cut efficiency) using the Lambert W function.
\item A Monte Carlo method for computing the statistical uncertainty of the accidentals rate.
\item Detailed studies of the differences between the IBD selections of different analysis groups.
\item The IBD selection.
\item Selector framework.
\item Parallelization and reorganization of the fitter.
\item Studies of nonuniformity in EH3-AD4 etc.
\item Technical implementation of AdSimpleNL.
\item Running production.
\item Maintenance and debugging of NuWa.
\item Convening DQWG. Development of scripts, procedures, docs.
\item Development of DQ website.
\item Maintenance of CQ DB. Scripts, procedures, docs.
\item Cut variation.
\item Remote DAQ for the tabletop NL measurement.
\item Muon track reconstruction?
\item Detective work, e.g. NominalCharge bug.
\item Techniques to correct background rates for changes in cuts, without rerunning the original background analysis.
\item Calibration automation (spallation neutrons).
\item Integration of all the pieces of the chain, consistency of inputs etc.
\item Methods of measuring the neutron capture spectra and calculating delayed efficiency.
\item DQ shift procedures.
\end{itemize}