\documentclass[../thesis.tex]{subfiles}

% \MyDoubleSpacing
% \begin{center}
%   {\large\textbf{Acknowledgments}} 
% \end{center}

\begin{document}

\chapter*[Acknowledgments]{Acknowledgments}
\addcontentsline{toc}{chapter}{Acknowledgments}

First and foremost, this work owes its existence to the patient guidance and support of my friend and advisor, Professor Kam-Biu Luk, I have been extraordinarily lucky to have him as a mentor, one from whom I have learned so much not only about physics, but also about scientific integrity, service, and being a decent human being. His example truly provides an inspiration to strive toward. Moreover, his detailed and insightful feedback was invaluable over the course of preparing this work.

In a similar vein, I wish to thank the other members of my dissertation committee, Professors Yury Kolomensky and Karl van Bibber, for taking the time to review this work and provide suggestions for improvement. I owe Professor Kolomensky additional thanks for the many 290E seminars that he organized and led over the years. In particular, his lecture slides on statistics remain an extremely useful reference, all these years later. I cannot neglect to also note how much I enjoyed and learned from the many courses I took from him both as an undergraduate and a graduate student.

I have also learned a great deal from the many colleagues I've been privileged to work with on Daya Bay. Naturally, I've worked most closely with those who have formed part of the LBNL analysis group at one time or another. In no particular order, these include Cheng-Ju Lin, Herb Steiner, Dan Dwyer, Pedro Ochoa, Yasu Nakajima, Patrick Tsang, Sam Kohn, Chris Marshall, and Henoch Wong. My extreme gratitude also goes out to my Czech colleague Beda Roskovec; in addition to his collaboration on numerous analysis tasks, I truly appreciate his generosity in volunteering to assist me, and eventually lead the charge, in carrying out the activities of Daya Bay's Data Quality Working Group. My Chinese colleague Wenqiang Gu was also extremely helpful in teaching me about Daya Bay's data quality procedures. I also wish to thank my Russian colleagues, Maxim Gonchar and Kostya Treskov, for numerous delightful conversations on physics, programming, and life, and for their assistance in tracking down some particularly thorny bugs in my IBD selection.

The analysis presented in this work depends on a number of ingredients for which credit is due. Namely, the reactor model, including the predicted flux and spectra and the associated covariance matrix, is due to Christine Lewis, while the LBNL fitter and toy Monte Carlo were built by Yasu Nakajima, Patrick Tsang, Pedro Ochoa, Cheng-Ju Lin, and Henoch Wong. The fast-neutron background prediction comes from Bei-Zhen Hu, and the predictions for the AmC and $\alpha$--$n$ backgrounds from Lianghong Wei. The energy nonlinearity model was provided by Yongbo Huang. References to their work can be found in the appropriate sections of the text. Of course, I must also collectively credit the many scientists and engineers on the Collaboration who have contributed to the overall experimental apparatus, the calibration, the reconstruction, and so forth. Daya Bay is a team effort, and I have been fortunate to work with such an incredible team. I've also been fortunate to have access to the computing and storage facilities at NERSC, and I thank the engineers who scramble behind the scenes while incompetent users, such as this author, try as hard as possible to crash the system.

This work was prepared entirely using Free Software, which I capitalize to emphasize that such software not only is free-of-charge, but also respects and enables the freedom of the user to inspect, modify, and share the code. I am a strong believer that science should be open and accessible to all, and Free Software is, in my view, essential for achieving this goal. I thus would like to thank the many individuals who have contributed to Linux, Git, \LaTeX\, GCC, ROOT, Python, Julia, Geant4, Numpy, and all of the time-tested scientific libraries responsible for much of the heavy lifting in this field. Of all the software I have used in my work, however, the one that is closest to my heart is GNU~Emacs. As far as I am concerned, this 40-year-old relic is still a century ahead of its time, and I cannot imagine working without it.

Finally, I wish to thank my family, my friends, and Cal Sailing Club. I am truly blessed to have all of you in my life.

\end{document}
