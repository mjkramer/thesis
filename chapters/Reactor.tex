\documentclass[../thesis.tex]{subfiles}

\begin{document}

\chapter{Reactor prediction}
\label{chap:bkg}

In principle, a reactor-based near/far experiment can measure the oscillation parameters without any recourse to reactor physics: Simply measure the spectrum at the near site, ``unoscillate'' it back to the reactor, and then oscillate it out to the far site. The oscillation parameters are then those that give the best fit to the far-site data. No information is needed on reactor power, isotope fractions, or theoretical antineutrino spectra.

In practice, the situation is complicated by the fact that Daya Bay features two reactor complexes and two near sites. Instead of two baselines, there are effectively six, and the reactors can differ from each other in terms of power and burnup. These factors must be accounted for when predicting the far site spectra from the near ones.

In the simplest approach, one can just use the known power and baseline information to split each near spectrum into components from each reactor, and then unoscillate/oscillate each component separately to the far site. This can be done separately for each near site, and the two far site predictions can then be averaged. This is sufficient for getting a good measurement of $\theta_{13}$.

However, measuring $\Delta m^2$ involves examining the detailed shapes of the near and far spectra. In the simple approach just described, it is assumed that all reactors are producing the same spectral shape (and ratio of flux to power). This is not necessarily accurate, as the reactors may be in different stages of their fuel cycles. Therefore, to get the most out of a rate+shape analysis, we must perform a full prediction of the spectrum from each reactor separately. This will, naturally, introduce uncertainties from the imperfect nature of the predictions. As long as the predictions are applied consistently, their uncertainties will be largely (but not completely) cancelled in the near/far ratios, so this primary benefit of a near/far measurement is not lost.

Unfortunately, the multi-reactor/site layout does mean that we lose the ability to completely disregard the question of predicting the antineutrino spectrum from a nuclear reactor. This chapter describes this prediction and how it fits into the overall oscillation analysis.

This chapter describes the reactor prediction, beginning with the question of predicting the antineutrino spectrum from a single fission of a given isotope, and then proceeding to discuss how these are combined in the Daya Bay reactors, using information on power and burnup (i.e., fission fractions) to form a full spectrum prediction.

\section{Spectrum prediction}
\label{sec:specpred}

\def\urfive{$^{235}$U}
\def\punine{$^{239}$Pu}
\def\puone{$^{241}$Pu}
\def\ureight{$^{238}$U}

In a conventional nuclear reactor, virtually all of the power originates from the fission of four isotopes: \urfive, \punine, \puone, and \ureight. In a fresh fuel assembly of low-enriched uranium, initially some 92\% of fissions will be of \urfive, while the remaining 8\% will be fast-neutron fissions of \ureight. This \ureight fraction remains nearly constant throughout a fuel cycle. At the same time, some of the \ureight will undergo neutron capture and subsequent beta decay to \punine, whose fission rate rapidly reaches 10-20\% of the total and which will eventually reach that of \urfive by the end of the cycle ($\sim$450 days). A fraction of \punine will produce \puone after a pair of neutron captures, and by the end of the cycle this isotope will contribute a fission fraction comparable to that of \ureight.

As can be seen, all four isotopes contribute a non-neglible fraction of the total thermal power, and since their antinuetrino spectra are fairly similar, they all contribute significantly to the flux. Therefore, it is imperative that each isotope's spectrum be predicted as accurately as possible. This can be done either \emph{ab initio}, by combining the beta spectra of all fission products listed in nuclear databases, or via the \emph{conversion method}, in which the total beta spectrum is measured and then converted into an antineutrino spectrum. The latter method is generally preferred when measurements are available, as nuclear databases are known to be incomplete. In what follows, we discuss the existing predictions produced using both methods, and determine an optimal set of predictions to use in the oscillation analysis.

\subsection{\textit{Ab initio} method}
\label{sec:abinitio}



\subsubsection{Vogel calculation (U238)}
\label{sec:vogel}

\subsubsection{French calculation (U238)}
\label{sec:french238}

\subsection{Conversion method}
\label{sec:conversion}

\subsubsection{ILL $\beta$ spectra measurements}
\label{sec:illmeas}

\subsubsection{Schreckenbach ILL conversion}
\label{sec:schreck}

\subsubsection{French ILL conversion}
\label{sec:frenchconv}

\subsubsection{Huber ILL conversion}
\label{sec:huberconv}

\subsubsection{FRM II U238 measurement and conversion}
\label{sec:u238conv}

\subsection{Off-equilibrium correction}
\label{sec:offeqcorr}

\subsection{Spent nuclear fuel}
\label{sec:snfcorr}

\section{Power, burnup, and isotope fractions}
\label{sec:reacpow}

\section{AD spectrum prediction}
\label{sec:adspectra}

\section{Uncertainties and correlations}
\label{sec:reacunccorr}

