\documentclass[../thesis.tex]{subfiles}

\begin{document}

\chap{Reactor prediction}
\label{chap:reactor}

\conts

In principle, a reactor-based near/far experiment can measure the oscillation parameters without any recourse to reactor physics: Simply measure the spectrum at the near site, ``unoscillate'' it back to the reactor, and then oscillate it out to the far site. The oscillation parameters are then those that give the best fit to the far-site data. No information is needed on reactor power, fission fractions, or theoretical antineutrino spectra.

In practice, the situation is complicated by the fact that Daya Bay features two reactor complexes and two near sites. Instead of two baselines, there are effectively six, and the reactors can differ from each other in terms of power and burnup. These factors must be accounted for when predicting the far site spectra from the near ones.

In the simplest approach, one can just use the known power and baseline information to split each near spectrum into components from each reactor, and then unoscillate/oscillate each component separately to the far site. This can be done separately for each near site, and the two far site predictions can then be averaged. This is sufficient for getting a good measurement of $\theta_{13}$, thanks to the cancellation of absolute efficiency uncertainties and the lack of need for reactor modeling.

However, measuring $\Delta m^2$ involves examining the detailed shapes of the near and far spectra. In the simple approach just described, it is assumed that all reactors are producing the same spectral shape (and ratio of flux to power). This is not necessarily accurate, as the reactors may be in different stages of their fuel cycles. Therefore, to get the most out of a rate/shape analysis, we must perform a full prediction of the spectrum from each reactor separately. This will, naturally, introduce uncertainties from the imperfect nature of the predictions. As long as the predictions are applied consistently, their uncertainties will be largely (but not completely) cancelled in the near/far ratios, so this primary benefit of a near/far measurement is not lost.

This chapter describes the reactor prediction, beginning with the question of predicting the antineutrino spectrum from a single fission of a given isotope, and then proceeding to discuss how these are combined, with the aid of information on power and burnup (i.e., fission fractions), to form a full spectrum prediction.

\section{Spectrum prediction}
\label{sec:specpred}

\def\urfive{$^{235}$U\xspace} \def\punine{$^{239}$Pu\xspace}
\def\puone{$^{241}$Pu\xspace} \def\ureight{$^{238}$U\xspace}

In a conventional nuclear reactor, virtually all of the power originates from the fission of four isotopes: \urfive, \punine, \puone, and \ureight. In a fresh fuel assembly of low-enriched uranium, initially some 92\% of fissions will be of \urfive, while the remaining 8\% will be fast-neutron fissions of \ureight. This \ureight fraction remains nearly constant throughout a fuel cycle. At the same time, some of the \ureight will undergo neutron capture and subsequent beta decay to \punine, whose fission rate rapidly reaches 10-20\% of the total, eventually catching up to the (decreasing) \urfive rate by the end of the cycle ($\sim$450 days). A fraction of \punine will produce \puone after a pair of neutron captures, and by the end of the cycle this isotope will contribute a fission fraction comparable to that of \ureight.

Clearly, all four isotopes contribute a non-neglible fraction of the total thermal power, and since their antineutrino spectra are fairly similar, they all contribute significantly to the flux. Therefore, it is imperative that each isotope's spectrum be predicted as accurately as possible. This can be done either \emph{ab initio}, by summing theoretical spectra based on the individual beta branches of all fission products listed in nuclear databases, or via the \emph{conversion method}, in which the total beta spectrum is measured and then converted into an antineutrino spectrum. The latter method is generally preferred when measurements are available, as nuclear databases are known to be incomplete. In what follows, we discuss the existing predictions produced using both methods, and determine an optimal set to use in the oscillation analysis.

\subsection{\textit{Ab initio} method}
\label{sec:abinitio}

In a beta decay, conservation of energy implies a one-to-one correspondence between electron and antineutrino energy.\footnote{Ignoring nuclear recoil effects, which introduce a negligible smearing of $\mathcal{O}(E_0/(Am_p))\,\sim\,10^{-4}$ [Huber]} For a single beta branch of endpoint energy $E_0$,

\[ E_\nu = E_0 - E_e. \]

Hence, if we know the spectrum of every beta branch of every fission product, we can invert them all and sum them up to derive the total antineutrino spectrum. Unfortunately, this procedure is hampered by two significant issues. First of all, nuclear databases contain the beta decay \emph{endpoints}, not the spectra, and there are theoretical uncertainties involved in calculating the spectra. Secondly, existing databases are known to be missing some 10\% of beta branches, and errors exist in the listed branching ratios for the known branches.

The theoretical difficulties arise because, to properly model the spectrum for a single beta branch, it is not enough to merely know the endpoint energy. One must also know the \emph{type} of decay: Is it a Fermi decay, with antiparallel electron and antineutrino spins? Or a Gamow-Teller (GT) decay, with parallel spins? Does the lepton system carry any orbital angular momentum, making it a \emph{forbidden} (as opposed to an \emph{allowed}) decay? If the decay is forbidden, does the nucleus undergo the maximum possible $\Delta J$ (a \emph{unique} decay), or not (a \emph{non-unique} decay)? The type of decay directly affects the shape of the spectrum.

Traditionally, \emph{ab initio} calculations have assumed that all beta branches are of allowed GT type. Unfortunately, nuclear databases indicate that some $\sim$25\% of fission product beta branches are forbidden. Even if we generously assume that the databases correctly list the type of each decay, there remains a problem: While it is possible to calculate a general shape correction for \emph{unique} forbidden decays, the correction for a \emph{non-unique} decay depends on the exact combination of nuclear matrix elements involved in the decay, and this information is largely unknown.

Furthermore, the idealized Fermi model of beta decay ignores a number of subtle effects that add further corrections to the spectrum. Most of these are well-understood, including the effects of the finite size of the nucleus, charge screening, and radiative corrections. These corrections can be applied with minimal uncertainty. However, there is an additional effect known as \emph{weak magnetism} (WM): Essentially, the total weak current of the nucleus contains a contribution from the spatial distribution of the vector current, and this factor depends on the specific (and typically unknown) details of each nucleus's structure. In \emph{ab initio} calculations, numerous assumptions and simplifications are applied in order to produce a tractable model for the WM correction, which is then fit to measured observables. While this approach is better than nothing, it still results in one of the largest components of the uncertainty produced by the \emph{ab initio} method.

Finally, \emph{ab initio} calculations are hampered by the fact that nuclear databases suffer from missing and incorrect information. This problem is particularly acute for the rarest and/or most unstable isotopes, some of which are entirely missing from the databases. Due to their short livetime, these isotopes are expected to possess high-energy beta branches, well above the inverse beta decay threshold. Among known daughter isotopes, $\sim$6\% lack \emph{any} tabulated data on beta branches [Dwyer-Langford]. In [Mueller], discussed later, their hybrid \emph{ab initio}/conversion procedure suggests that nuclear databases fail to account for some 10\% of the measured beta decay spectrum. Even among isotopes for which data exists, this data may be biased by the \emph{pandemonium effect}, in which low-$E_0$ beta branches are undercounted relative to high-$E_0$ ones for the same isotope, due to the fact that the deexcitation of the daughter nucleus involves low-energy transitions between closely spaced energy levels, and these gammas often evade measurement.

Altogether, while the \emph{ab initio} method is attractive and elegant in principle, in practice it suffers from deficiencies in underlying nuclear databases, as well as a poor understanding of the weak magnetism effect. For this reason it has traditionally been rejected in favor of the conversion method, described next. The exception is \ureight: Since it is only fissioned by fast neutrons, measuring the total beta spectrum of its fission daughters is difficult, and data only became available in the last few years. Given that \ureight only contributes some 10\% of the total fission rate, even a conservative 10\% error on the \emph{ab initio} result will only result in an uncertainty of 1\% on the total spectrum, which is acceptable.

\subsubsection{\ureight calculations}
\label{sec:vogel}

In the 1980s, P.~Vogel carried out a prediction of the \ureight antineutrino spectrum using the nuclear data available at the time. Although this was a very careful analysis, certain approximations were made in order to keep the calculation tractable; for instance, the finite size and weak magnetism effects were parameterized by a single energy-dependent correction applied to the spectrum as a whole, rather than being treated branch-by-branch.

In 2011, the ``French'' collaboration of Mueller \emph{et al.} revisited the problem using the latest available nuclear data, aggregated and curated from multiple sources. This time, all higher-order corrections were applied on each branch individually. With these two improvements, along with an increase in the theoretical IBD cross section, a 9.8\% increase was found in the predicted \ureight contribution to IBD rate, relative to Vogel's result. To this day, the French spectrum remains the state-of-the-art among \ureight \emph{ab initio} predictions, and will be used in this oscillation analysis.

\subsection{Conversion method}
\label{sec:conversion}

In contrast to the \emph{ab initio} method, which depends on thousands of measurements of individual fission daughters and beta branches, the conversion procedure relies on just one measurement: The total beta spectrum from fissions of a given isotope. The total beta spectrum provides a powerful constraint, but it cannot be directly converted to an antineutrino spectrum the way that a single branch can. The traditional solution to this problem is to fit the total spectrum with a series of fictional \emph{virtual} branches, which are then inverted separately and summed to give the total antineutrino prediction. Typically, one starts at the endpoint of the total spectrum, positing a virtual branch (an allowed GT decay, generally) of the same endpoint. This virtual branch is normalized by fitting it to the end of the total spectrum, and then subtracted out. The procedure is repeated at the new endpoint of the subtracted total spectrum, until one finally ends up with a few dozen virtual branches that together fit the entire spectrum.

While this approach avoids the uncertainties caused by the incompleteness of nuclear databases, it gains uncertainty from the arbitrary nature of the virtual branch technique. Fortunately, this uncertainty can be characterized fairly well by varying the procedure and observing the changes in the result. It should be noted that, in spite of their relative independence, both approaches suffer from some of the same theoretical uncertainties involved in inverting single beta branches (whether real or virtual), particularly from weak magnetism. In the end, however, the total uncertainty of the conversion approach is found to be less than that of the \emph{ab initio} approach [XXX by how much?].

\subsubsection{ILL $\beta$ spectra measurements}
\label{sec:illmeas}

For \urfive, \punine, and \puone, all conversion predictions make use of the same measurements of the total beta spectra. These were taken at ILL in 1980s by Schrekenbach \emph{et al.}. Thin foils of each isotope were subjected to a thermal neutron flux from the ILL research reactor, and a small, extremely pure sample of beta decay electrons escaped through a narrow vacuum pipe for measurement by a high-resolution magnetic spectrometer, BILL. For normalization, the total number of fissions was later measured by performing gamma counting on the foils [XXX check]. These measurements remain the most precise total spectra in the literature.

\subsubsection{Schreckenbach ILL conversion}
\label{sec:schreck}

After these beta spectra were measured, Schreckenbach \emph{et al.} proceeded to convert them into antineutrino spectra. Their method was a ``pure'' conversion, based only on virtual branches with no input from nuclear databases. As with Vogel's \ureight \emph{ab initio} prediction, certain corrections (finite size, weak magnetism) were applied in a simplified manner [XXX check] to the total spectrum, providing the dominant uncertainty on the final result. Until 2011, the Schreckenbach conversion was considered canonical.

\subsubsection{French ILL conversion}
\label{sec:frenchconv}

The situation changed in 2011 with the work of Mueller \emph{et al.}. The French team took the attitude that, while nuclear databases are indeed somewhat incomplete, the data they \emph{do} include is still a precious constraint that deserves to be considered in the conversion procedure. Accordingly, they began with an \emph{ab initio} calculation for all four isotopes. For \ureight, as described above, they had to stop at that point. But for the other three isotopes, they proceeded to subtract the \emph{ab initio} spectra (for the electron, not the antineutrino) from the ILL measurements, leaving a $\sim$10\% residual, which was then fit with virtual branches. This ``hybrid'' approach, by making use of \emph{all} available data, with its redundant constraints, resulted in a substantially reduced uncertainty [XXX how much?]. As was mentioned, Mueller \emph{et al.} also applied a more accurate, branch-by-branch correction for weak magnetism and other higher-order effects. Their final result predicted a 3\% total increase in the IBD rate relative to the Schreckenbach/Vogel predictions.

\subsubsection{Huber ILL conversion}
\label{sec:huberconv}

Shortly after the publication of the French prediction, P.~Huber undertook an independent calculation of the antinuetrino spectra from \urfive, \punine, and \puone. Unlike Mueller \emph{et al.}, he avoided using nuclear databases, instead converting each spectrum using only virtual beta branches, as Schrekenbach had. His analysis included careful studies of the variance introduced by the details of the conversion procedure; based on these studies, the procedure was tuned to minimize any introduced bias [XXX check???]. Like the French team, Huber carefully treated the WM and other corrections on each individual virtual branch. With these improvements, Huber obtained a result that was largely consistent with the French one. In this oscillation analysis, we use an equally-weighted average of the Huber and Mueller predictions for these three isotopes [XXX or do we? and how are uncertainties combined?].

\subsubsection{FRM II U238 measurement and conversion}
\label{sec:u238conv}

It is worth noting that in 2013, the \ureight total beta spectrum was finally measured by N.~Haag \emph{et al.} at the FRM~II neutron source in Germany. They exposed foils of natural uranium to both thermal and fast neutrons, measuring the beta spectra with a gamma-suppressing electron telescope, whose efficiency was accurately determined by comparing their \urfive data to the results from ILL. This is an extremely valuable measurement, but conversion results remain preliminary [XXX are they even published?], so for this oscillation analysis we will instead employ the pure \emph{ab initio} result of Mueller \emph{et al.}

\subsection{Off-equilibrium correction}
\label{sec:offeqcorr}

When a reactor is at equilibrium, each unstable isotope is held at a constant concentration (modulo evolution of the fuel fractions), decaying at the same rate as it is generated from its parent. At reactor startup, these concentrations are all effectively zero, and each isotope accumulates until its total decay rate equals its production rate. The longer an isotope's lifetime (or those of its ancestors), the longer this takes.

At ILL, the target foils were irradiated for about a day. However, reactor fuel typically lives in the core for more than a year. As such, the ILL beta spectra do not properly account for isotopes that take more than a day to reach equilibrium. Around 10\% of fission products meet this criterion, so the effect cannot be neglected. As longer-lived isotopes are the ones affected, the spectral distortion is restricted to low energies, up to about 4~MeV. Any correction must obviously be time-dependent, spanning the range from initial irradiation up to the end of a fuel cycle.

Mueller \emph{et al.} provide correction factors for \urfive, \punine, and \puone at five energy values from 2 to 4~MeV, calculated at irradiation times of 100~d, 10$^7$~s ($\sim$115~d), 300~d, and 450~d. The ILL reference spectra provide an anchor at 36~h (except for \urfive, which was measured at 12~h; a 36~h correction is thus provided for the isotope). These corrections, at 450~d, are at most 2\% (5\% for \urfive). They also account for another, subdominant difference between the reactor environment and the ILL apparatus, namely, the presence of epithermal and fast neutrons, which alter the distribution of fission products. The reactor simulation code MURE was used to determine the correction, which was validated against the FISACT code.

In this oscillation analysis, we use Mueller's corrections verbatim, collectively referring to them as the ``off-equilibrium correction''. Interpolation in time is performed between the provided points, and the appropriate value of $t$ is determined from data provided by the power company (discussed in Sec.~\ref{sec:reacpow}). In the Daya Bay cores, one third of the fuel is replaced during each refueling, so each core contains three fuel batches at different levels of burnup. Accordingly, a weighted sum of off-equilibrium corrections is applied. Conveniently, the batched fueling tends to wash out the differences between cores as well as the overall time dependence of the correction.

For our purposes, we assign a 30\% uncertainty to the Mueller correction factors. The correction increases the predicted flux by some 0.5\%, so the ensuing uncertainty on the absolute rate is around 0.2\%.

\subsection{Spent nuclear fuel}
\label{sec:snfcorr}

During refueling, spent fuel is added to water-filled storage pools on site. Long lived fission products in the spent fuel will continue to decay for some time [XXX how long? Check data files], producing an additional low-energy antineutrino flux whose spectrum can be calculated from nuclear data, assuming that the fuel's irradiation history (and time since removal) is known.

As virtually all of the relevant decay activity from a spent fuel batch will have subsided by one year [XXX check], it is reasonable to only consider the most recent spent batches at each core. Using public information on irradiation time and fuel quantity, P.~Jaffke has produced reference spectra for one spent fuel batch at 1~d, 100~d, and 521~d (326~d) after fuel removal at Daya Bay (Ling Ao). By combining these spectra with the refueling history of each core, the spent fuel contribution can be calculated. In total, spent fuel is predicted to account for around 0.3\% of the IBD rate, with an assigned uncertainty of 100\%.

\section{Power, burnup, and fission fractions}
\label{sec:reacpow}

In order to combine the various ingredients above into a final reactor prediction, we must know the thermal power of each core, its fission fractions, and its refueling history. This information is provided by the power company in a number of forms.

Each core's average daily thermal power is provided as a fraction of the full nominal power (2895~MW). This information is crucial for calculating fuel fractions (discussed below), but the Daya Bay detectors often do not manage to produce a full 24~hours of ``good'' data each day. As such, the collaboration periodically supplies the power company with a ``good hour'' list, from which the company produces a livetime-averaged daily power. Combined with the daily livetime, this provides a normalization for the daily predicted spectrum at each detector.

The fission fractions of a core can be expressed as a function of \emph{burnup}. Burnup is defined as the total energy extracted from each unit mass of fuel, typically given in units of MWd/(megaton uranium). The power company has performed simulations to determine the fission fractions versus burnup of each core\footnote{Data is provided for a full fuel cycle; for cycles that haven't yet run their course, nominal projections of thermal power are used for the ``future'' fractions.}, and this information is provided to the collaboration in intervals of $\sim$1000~MWd/MTU. The company also provides the burnup of each core at two-week intervals; using daily thermal power to interpolate between the provided points, the daily burnup, and hence fission fractions, can be calculated.

[XXX How do we deal with the fact that each core contains three batches at different burnups? See Christine's code.]

\section{Final AD spectrum}
\label{sec:adspectra}

Putting it all together, the predicted IBD spectrum at a given detector, from a single core, is

\begin{samepage}
  \[ S_{\mathrm{ibd}}(E, t) = \frac{N_p(t)}{4\pi L^2} \, \epsilon(E, t) \,
    \sigma(E) \left( \sum_i \left( \frac{W(t) \, f_i(t)}{\sum_i f_i(t) \, e_i}
        S_i(E) \, c_{\mathrm{ne},i}(E, t) \right) + S_{\mathrm{snf}}(E, t)
    \right). \]

  Here, $E$ is the \nubar energy, not the IBD prompt energy; $N_p(t)$ is the number of target protons; $L$ is the baseline; $\epsilon(E, t)$ is the detection efficiency; $\sigma(E)$ is the IBD cross section; $W(t)$ is the thermal power; $f_i(t)$ is the fission fraction of isotope $i$, $e_i$ is its energy per fission, $S_i(E)$ is its per-fission spectrum, and $c_{\mathrm{ne},i}(E, t)$ is its non-equilibrium correction; and finally, $S_\mathrm{snf}(E, t)$ is the spectrum from spent fuel. The full prediction at each detector is obtained by adding the contributions from each core.
\end{samepage}

\section{Uncertainties and correlations}
\label{sec:reacunccorr}

To be continued.

\end{document}