% XXX discuss the spill-in efficiency etc.

\documentclass[../thesis.tex]{subfiles}

\begin{document}

\chapter{Oscillation fit}
\label{chap:fitting}

\section{Introduction}
\label{sec:fittingIntro}

At this stage in the analysis, we have a sample of IBD candidates, the background predictions, and the calculated efficiency corrections (for the muon veto and the multipliticy cut). The next step is to take the background-subtracted and efficiency-corrected IBD prompt spectra at the eight ADs, and fit them to the 3-flavor oscillation model in order to extract $\SinSq$ and $\Dmsqee$.

Our analysis is based on the Daya Bay oscillation fitter developed at Lawrence Berkeley National Laboratory (LBNL) \cite{berkeley_shapefit,berkeley_toymc}, which is one of a half-dozen independent fitters used in the production and cross-checking of Daya Bay's official oscillation fits. While it is relatively straightforward to find the oscillation parameters that best fit Daya Bay's data, determining their uncertainties is significant more complicated, given the need to consider a wide range of systematics (e.g., those related to the reactors, the detectors, and the backgrounds). The modeling and methods used by the LBNL fitter have been carefully developed and validated over the years, allowing our analysis to benefit from these efforts. Furthermore, our use of the LBNL fitter ensures that our cut-variation studies are directly applicable to Daya Bay's past and future results.

The LBNL fitter is a \emph{relative} fitter: That is, instead of simultaneously comparing all eight ADs to a model of the reactor spectra and oscillation probability (an \emph{absolute} fit), it uses the measurements of the near ADs to predict the far AD spectra (including oscillation effects). This approach ensures that the results are independent (to first-order) of the modeling used for the reactor spectra. A reactor model is still required in order to decompose each near AD's spectrum into the components from individual cores, but the final result is negligibly affected by the details of the model. Among the Daya Bay fitters developed outside of LBNL, some take the absolute approach, and they too are able to produce results that are stable across reactor models, but this requires using explicit nuisance parameters (or ``pull terms'') to represent the uncertainties in the reactor modeling.

Aside from its use of a relative approach, a second defining characterstic of the LBNL fitter is its use of a covariance matrix to encode all systematic (as well as statistical) uncertainties. This ensures that finding the best-fit oscillation parameters is a simple two-dimensional minimization problem, with an objective function of the form
\begin{equation}
  \chi^2 = (x_i - \widebar x_i) V_{ij}^{-1} (x_j - \widebar x_j).
\end{equation}
where the index $i$ runs over the energy bins in the far hall, the $x_i$ are the measured far-hall spectra, the $\widebar x_i$ are the predicted far-hall spectra based on the near-hall observations, and $V_{ij}$ is the total covariance matrix (including both systematic and statistical components). All of the complexity lies in the determination of $V_{ij}$ (which we discuss shortly). In contrast to the covariance matrix approach, it is possible to parameterize each systematic with a pull term $\eta_j$ (and a corresponding nominal value $\widebar \eta_j$ and variance $\zeta_j^2$), resulting in an objective function akin to
\begin{equation}
  \chi^2 = \sum_i \frac{(x_i - \widebar x_i)^2}{\sigma_i^2} + \sum_j \frac{(\eta_j - \widebar \eta_j)^2}{\varsigma_j^2}.
\end{equation}
In this case, the predictions $\widebar x_i$ are implicitly dependent on the pull terms $\eta_j$. Performing the fit then involves finding both the oscillation parameters \emph{and the pull terms} that minimize $\chi^2$. This is a high-dimensional minimization problem, with an attendant computational cost and need to avoid false minima, but it avoids the calculation of a complex covariance matrix. Both methods, covariance-matrix-based and pull-term-based, can be used in either a relative or an absolute approach, leading to a total of four options. Each option has benefits and drawbacks from the standpoint of implementation complexity; however, assuming the use of consistent modeling, all four should give consistent final results, as has indeed been demonstrated repeatedly during the Collaboration's internal cross-checks. We do not make any claim of superiority for the LBNL fitter's relative, covariance-matrix approach; it is simply what we have chosen to use. Further discussion of these different methodologies can be found in \autoref{sec:fitoverview}.

As was mentioned, the price of a simple two-dimensional minimization is that we must first calculate the covariance matrix $V_{ij}$. This matrix encodes the variance in each energy bin, and its covariances with other bins, as the combination of contributions from \emph{all} sources of uncertainty, systematic and statistical. For any reasonably complex experiment such as Daya Bay, this cannot be done analytically, or even (deterministically) numerically. Instead, Monte Carlo (MC) must be used. The LBNL fitter includes a toy MC for this purpose. The toy MC generates sets of predicted prompt spectra at the eight ADs. Each set (``toy sample'') is generated under randomized assumptions regarding detector and background uncertainties. From a large ensemble of toy samples, the covariance between each pair of energy bins can be calculated, giving a partial covariance matrix that accounts only for the detector and background systematics. This is then combined with two other matrices: a preexisting covariance matrix that accounts for reactor-related systematics, and an analytically calculated statistical covariance matrix, giving the full matrix used in this fit.

In the remainder of this chapter, we provide basic overviews of the reactor model, the toy MC, and the fitter, as well as a summary of our inputs and results. Further details can be found in Appendices~\ref{chap:reactor} and \ref{chap:fittingDetails}.

\section{Reactor model}
\label{sec:fittingReactor}

Although the details of the reactor model have minimal bearing on the final result, a model must still be chosen in order to decompose each near AD's spectra into individual reactor contributions, for extrapolation to the far hall. In keeping with the official practice of the Collaboration, we use the results of Huber \cite{PhysRevC.84.024617} and Mueller \cite{PhysRevC.83.054615}. In particular, for fissions of $^{235}$U, $^{239}$Pu, and $^{241}$Pu, we use Huber's conversions of the total $\beta$ spectra measured at Institut Laue-Langevin (ILL), while for $^{238}$U, we use Mueller's \emph{ab initio} calculations based on nuclear data. These predictions include time-dependent corrections for each reactor's departure from equilibrium, as well as additional contributions from the spent nuclear fuel (SNF) pools near each core.

The four individual isotope spectra must then be appropriately scaled and combined to obtain the total prediction for each core. This is accomplished using data provided by the power plant on each core's weekly average power and fission fractions. This allows the total spectrum at each core to be determined on a weekly basis. Livetime-weighted summing of these spectra then gives the predictions for the entire data period. The toy MC can then feed these spectra into its model of propagation/oscillation, the IBD cross-section, and the detector response, producing a toy sample of the predicted prompt spectrum at each AD.

Aside from the predicted core spectra, the reactor model is also responsible for producing the reactor-related covariance matrix. This is done by quantifying all systematic uncertainties, include those on the weekly power, the fission fractions, the isotope spectra, and the off-equilibrium and SNF corrections. The model is then used to generate samples of core spectra under variations of these systematics. These core spectra are fed to the LBNL toy MC to produce the AD prompt spectra, from which the covariance matrix is then calculated. This is only performed once, given the fixed nature of the reactor prediction. In contrast, the detector, background, and statistical covariance matrices are recalculated by the toy MC for each fit, as discussed next.

\end{document}
