\documentclass[../thesis.tex]{subfiles}

\begin{document}

\chapter{Background subtraction}
\label{chap:bkg}

While Daya Bay's design ensures a fairly pure sample of antineutrino events, a
small contamination of backgrounds (at the percent level) is unavoidable. These
backgrounds can be subdivided into \emph{correlated} and \emph{uncorrelated}
backgrounds. The uncorrelated backgrounds consist entirely of accidental
coincidences between singles events, and the rate and spectrum can be easily
estimated from that of the singles. On the other hand, the correlated
backgrounds are so named because the prompt and delayed pulses are correlated in
time, as they both originate from a single underlying process. The correlated
background in Daya Bay correspond to four distinct processes, and each one
requires its own technique for determining the rate and spectrum.

In this chapter, we discuss the measurement of each of these backgrounds. Once
they have been measured, their scaled prompt spectra can be subtracted from that
of IBD candidates, allowing the oscillation fit to proceed with more pure prompt
spectrum, albeit one with an additional uncertainty stemming from the imprecise
nature of the background measurements.

Before considering these double-coincidence backgrounds, we begin by discussing
a background defined at the level of \emph{individual} triggers, namely, the
so-called ``flashing'' of PMTs. Reduction of these ``flashers'' is necessary in
order to minimize the rate of the uncorrelated backgrounds.

\section{PMT light emission (``flashers'')}
\label{sec:bkgFlashers}

During detector commissioning, some PMTs were found to occasionally emit light
due to arcing in their bases. At any given time, a dozen or two PMTs in each AD
will have the tendency to flash brightly enough to trigger the detector. Some
flashers can produce as much as 100~MeV of reconstructed energy. Within the
delayed energy region of 6-12~MeV, the flasher rate has averaged at around
0.7~Hz for each AD. These ``delayed-like'' flashers, if included in the
analysis, would significantly increase the rate of backgrounds caused by the
accidental coincidence of two uncorrelated signals. As discussed in
\autoref{sec:accbkg} and \autoref{chap:accDMC}, the rate of such ``accidentals''
is proportional to the rate of delayed-like signals, and this rate (excluding
flashers) ranges from around 0.05~Hz at EH3 to 1~Hz at EH1. While this would
merely (roughly) double the 1\% accidental background in the near halls, in the
far hall it would increase this background by an order of magnitude to the 10\%
level, counter to Daya Bay's goal of perecent-level background contamination.

Fortunately, flashers are easily distinguished from ``physical'' singles due to
their unique pattern of light emission, enabling them to be removed from the
analysis with high efficiency while minimally affecting true IBDs. This light
pattern is characteristized by two ``hot spots'' on opposite sides of the
AD. When a PMT base emits light, much of the light is absorbed by the black
radial shield and conical magnetic shield. The remainder escapes within a
conical profile; some of the photons will strike the flasher's photocathode
(resulting in the flashing PMT having the highest charge), and others will
primarily illuminate the PMTs across the AD from the flasher, especially the one
that lies directly opposite to it. In addition, the time distribution of PMT
hits is broadened for flashers due to the geometry of light propagation across
the AD. By taking advantage of these telltale distributions of charges and
times, it is possible to achieve excellent discrimination of flashers from
physical events.

The flasher identification criteria were developed in a somewhat ad-hoc fashion,
by defining quantities that could conceivably serve as discriminators, and then
further defining combinations of these quantities, and finally plotting the
distributions of these (combined) quantities until a clean separation between
flashers and non-flashers was apparent. It is far more important to minimize
(IBD) signal inefficiency instead of maximizing flasher rejection, because a
small amount of unrejected flashers will simply slightly increase the rate of
accidental backgrounds (which can be easily quantified), whereas a signal
inefficiency could vary among the ADs and thereby bias the oscillation fit.

\newcommand\fmax{f_{\mathrm{max}}} \newcommand\fquad{f_{\mathrm{quad}}}
\newcommand\fID{f_{\mathrm{ID}}} \newcommand\fPSD{f_{\mathrm{PSD}}}

Early in the experiment, this prolonged and interative process eventually gave
rise to the \emph{ellipse cut} (based on the charge distribution) and the
\emph{PSD cut} (based on the time distribution), which demonstrated excellent
performance, and these cuts continue to be used in this analysis. The ellipse
cut is based on two quantities, termed $\fmax$ and $\fquad$. The first, $\fmax$,
is simply the ratio of $Q_{\mathrm{max}}$ (the maximum individual PMT charge
across all PMTs) over the total charge $Q_{\mathrm{tot}}$:
\begin{equation*}
  \fmax = \frac{Q_{\mathrm{max}}}{Q_{\mathrm{tot}}}.
\end{equation*}
For flasher events, $Q_{\mathrm{max}}$ belongs to the flashing PMT itself, and
$\fmax$ is typically higher for flashers than for physics events. However,
physical events near the edge of the detector can exhibit high $\fmax$, so this
variable alone is insufficient to cleanly discriminate flashers. As such, we
also consider $\fquad$, which is based on dividing the AD into four quadrants:
``Quadrant 1'' (q1) is the one that is centered on the highest-charge PMT; q3 is
the one across from q1; and q2 and q4 are the two ``to the side.'' $\fquad$ than
captures the conical nature of the light emission:
\begin{equation*}
  \fquad = \frac{Q_{\mathrm{q3}}}{Q_{\mathrm{q2}} + Q_{\mathrm{q4}}}.
\end{equation*}
Like $\fmax$, $\fquad$ alone is not a good discriminator, due to overlap between
flashers and physics events. However, their combination
\begin{equation*}
  \fID = \log_{10} \left[ \fquad^2 + \left( \frac{\fmax}{0.45} \right)^2 \right]
\end{equation*}
turns out to be an excellent discriminator. Indeed, as shown by (XXX Fig. 21 of
the long paper), requiring $\fID < 0$ reduces the flasher rate to a negligible
level, and many analyses have relied on $\fID$ alone to identify flashing 8"
PMTs. Still, even further flasher reduction can be achieved by incorporating
timing information. To capture the broadening of the time distribution shown by
flashers, we use the variable(s) $f_{\mathrm{t}1}$ ($f_{\mathrm{t}2}$), defined
as the ratio of the number of hits in the first 200 (150)~ns of the signal, over
the number of hits in the first 400~ns. The discriminator $\fPSD$ is then
defined as
\begin{equation*}
  \fPSD = \log_{10} [4 \cdot (1 - f_{\mathrm{t}1})^2 + 1.8 \cdot (1 - f_{\mathrm{t}2})^2].
\end{equation*}
By requiring both $\fID < 0$ and $\fPSD < 0$, we eliminate virtually all 8" PMT
flashers from the analysis. However, in addition to the 192 8" PMTs, there are
six 2" PMTs located at the top and bottom of each AD along the calibration axes,
and these can also flash. Such events were easily identified as those in which a
2" PMT saw an extreme amount of charge, with a cut of 100 PE providing
essentially perfect separation between 2" PMT flashers and other events.

The exact efficiency of these cuts, in terms of identifying flashers, is
unimportant, as long as it is high enough. Any residual flashers will
automatically be counted in the singles rate, and thus so will their
contribution to the accidental background rate.\footnote{There is a small
  second-order correction due to the fact that an accidental cannot be formed by
  two flashers in the same PMT, given that it takes on the order of a second for
  a PMT to ``recharge'' after flashing. However, this correction is negigible
  given the extremely low rate of residual flashers.} On the other hand, it
\emph{is} important to study the signal inefficiency, i.e. the probability of
improperly rejecting an IBD prompt or delayed trigger. If this inefficiency
differs significantly among the ADs, it could bias the oscillation result. The
inefficiency was estimated by taking a sample of IBD-like events (without the
flasher cut) (XXX Xin's flasher slides), and making a 2D histogram of
$(f_{\mathrm{prompt}},f_{\mathrm{delayed}})$, where $f$ is $\fID$ or
$\fPSD$. The IBD-like sample consists of a small number of accidentals that
contain a flasher (or two), and a much larger number of flasher-free pairs
(including true IBDs, accidentals, and correlated backgrounds). The use of
IBD-like events has the effect of diluting the presence of flashers in the
sample, since a coincident pair is much more likely to come from an IBD than an
accidental (which furthermore is more likely to have two physical singles rather
than a flasher). The inefficiency was then determined by the degree to which the
flasher-free ``blob'' extended into the regions rejected by the
discriminator. These studies found that $\fID$ and $\fPSD$ each introduce an
inefficiency of 0.02\%, with uncertainties of 0.01\% correlated and 0.01\%
uncorrelated. Since we use both of them in this analysis, our estimated
inefficiency is 0.04\%, with an uncertainty of 0.02\% correlated and 0.02\%
uncorrelated.

XXX show 2D plot of ellipse cut from doc-7143.

\section{Accidental coincidences}
\label{sec:accbkg}

As previously stated, the accidental background can be straightforwardly
measured based on the characteristics of singles events. The singles spectrum is
first measured by searching for prompt-like events that satisfy the usual muon
vetos but are separated from other prompt-like events by at least
400~$\mu$s. The total integral of this spectrum gives the prompt-like rate
$R_p$, while the integral above 6~MeV gives the delayed-like rate $R_d$. The
accidental background rate is then simply
\[ R_\mathrm{acc} = R_d(1 - e^{-R_p\Delta t})e^{-2R_p\Delta t}, \] where the
factor in parentheses is the probability for a prompt-like single to fall within
the $\Delta t$~=~200~$\mu$s preceding a delayed-like single, and the final
factor is the probability that the event is \emph{not} rejected by the decoupled
multiplicity cut, which disallows any additional prompt-like single in the
400~$\mu$s preceding the delayed event. Once the rate has been determined this
way, the spectrum is trivial: It is simply the singles spectrum itself.

\begin{comment}
  Mention IHEP's cross-check, and the additional uncertainty stemming from the
  difference between it and the nominal result?
\end{comment}

\section{Cosmogenic $^9$Li/$^8$He}
\label{sec:bkgCosmo}

\newcommand\linine{$^9$Li}

The dominant correlated background in Daya Bay comes from the short-lived
isotopes $^9$Li and $^8$He, which are produced as spallation products of carbon
when muons traverse the AD. These two isotopes\footnote{For the sake of brevity,
  we will henceforth collectively refer to the two isotopes as \linine; this is
  by far the dominant of the two, as will be shown later.} have livetimes on the
order of hundreds of milliseconds; thus, while the majority of them will decay
within the $\sim$1~s veto window that follows AD muons, a non-negligible
fraction will survive past it. When they undergo beta decay, the daughter nuclei
break apart, and neutrons are contained among the debris. The combination of the
beta decay and the subsequent nGd capture produce the characteristic
double-coincidence signature of an IBD event.

At the statistical level, the key difference between IBDs and \linine\ decays is
that the latter are correlated in time with AD muons. This correlation can be
exploited to measure the total rate of \linine\ decays. To do so, the IBD
candidate sample is used to construct a histogram of the time since the last AD
muon. In this histogram, true IBDs will exhibit an exponential distribution with
a characteristic time $\tau$ corresponding to the muon rate, while \linine\
decays will be similarly distributed but with a much shorter $\tau$
corresponding to the \linine\ lifetime. By fitting the histogram with a sum of
these exponential distributions, it is possible to extract the \linine\ rate.

Although this method is simple in theory, a significant challenge arises from
the fact that a minimum muon energy must be defined when calculating the time
between each event and its most recent preceding muon. If this cut is too low,
then the time between muons will be comparable to the \linine\ livetime, and
with finite statistics, it will be difficult to reliably distringuish between
the two components in the fit. Conversely, if the muon cut is too high, then
some fraction of \linine-producing muons will be discarded, and those \linine\
events will not appear to be muon-correlated, leading to an underestimation of
the rate.

To resolve this issue, one can repeat the \linine\ fit for a variety of muon
cuts, and then attempt to extrapolate the results down to a muon cut of (nearly)
zero. More concretely, one can obtain good fits at muon cuts of above $\sim
2\times10^5$~p.e., and marginal fits down to $1\times10^5$ p.e., and then
perform this extrapolation. Due to the absence of data points below
$1\times10^5$~p.e., and the lack of knowledge regarding the manner in which the
\linine-production rate scales with muon energy, this extrapolation dominates
the final systematic uncertainty of the result.

Improving this extrapolation requires the addition of data points at lower muon
cuts. This can be accomplished by modifying the IBD selection in order to enrich
it in \linine, resulting in an improved ability to distinguish the components in
the time-since-last-muon fit. Two methods have been developed to achieve this:
Neutron tagging, and restriction of the prompt energy cut. The first method
takes advantage of the fact that \linine\ decays are more likely than true IBDs
to occur in association with an additional neutron (from spallation), and the
second uses the fact that the \linine\ spectrum is significantly harder than the
IBD spectrum. By applying these cuts, performing the fits, and scaling the
results by the additional efficiency of these extra criteria, data points can be
obtained at lower muon cut energies.

Neutron tagging allows greater enrichment of \linine, at the cost of increased
uncertainty in the additional efficiency. Hence, it is better suited for the
lowest of muon cuts, where the prompt energy cut technique provides insufficient
enrichment, whereas the latter approach is preferable at higher muon cuts, due
to the avoidance of the neutron tagging efficiency. Below, we describe a
measurement that uses both techniques, applied where each is the optimal choice.

\subsection{Spectrum}
\label{sec:bkgLi9Spectrum}

The \LiHe\ spectrum can be either extracted from data or predicted from nuclear
tables. The two approaches give consistent results, and we briefly describe them
both here, although our analysis makes exclusive use of the predicted spectrum.

An extraction of the spectrum from Daya Bay data was performed by Marshall in
[XXX]. The approach takes advantage of the fact that \LiHe\ are essentially the
only IBD-like events that are correlated with muons on the 100~ms timescale. A
\LiHe-enriched sample was obtained by taking IBD-like events within 2---200~ms
of a ``shower'' muon, here defined as one producing at least $2\times10^5$
photoelectrons. This sample contained various muon-uncorrelated ``backgrounds'',
such as true IBDs and accidentals. In order to remove this contamination, a
\LiHe-depleted sample was obtained by looking for IBD candidates with no
preceding shower muons within 1.5~s. Before subtracting the two spectra, an
appropriate normalization for the depleted sample had to be determined. This was
done by performing the time-to-last-muon fit for the enriched sample, which
indicated the number of true \LiHe\ events in the sample, in turn implying the
number of non-\LiHe\ events. The depleted sample was thus normalized to this
latter count, and the subtraction was performed, giving the results shown in
Fig.~XXX.

The prediction of the \LiHe\ spectrum was carried out by Ochoa in [XXX docs
8772, 8860]. Three types of reference tables were consulted: nuclear structure,
branching ratios, and measured spectra (of neutrons, alphas, and gammas). Given
the number of decay pathways involved, a purely analytic approach was
infeasible, so a toy Monte Carlo was used to produce random decay events. The
discussion here uses the example of $^9$Li, but $^8$He was essentially treated
the same way. The initial $\beta$ decay (into any of four $^9Be^*$ states) was
simulated using the Fermi theory and the published energy levels and branching
fractions. For the decays that produce a $2\alpha+n$ final state (i.e. the only
decays of interest to us), $^9Be^*$ disintegrates via two consecutive two-body
decays (via either $^8$Be or $^5$He). Since there is no angular distribution to
consider in a two-body decay, the disintegration was treated using basic
kinematics, with the width of each state modeled with a Breit-Wigner
function. The result was a collection of simulated events, each one recording
the (true) energies of the electron, the neutron, and the two alphas (or, in the
case of $^8$He, the electron, the neutron, and the gamma).

In order to benchmark this simulation, its output was compared against published
measurements of \LiHe\ spectra of neutrons, alphas, and gammas. Based on this
comparison, one (particularly broad) level of $^8$He had to be augmented with a
Gaussian density of states (a complication not considered for the other
levels). Since the published branching ratios were relatively imprecise, they
were hand-tuned (within limits) in the simulation so as to achieve satisfactory
agreement with the published spectra.

Finally, to obtain the predicted spectrum in terms of prompt (i.e. reconstructed
energy), the simulated events were passed through a model of the detector
nonlinearity. Given that the \LiHe\ spectrum prediction was performed in 2013,
it is based on an older nonlinearity model than the one discussed in
\autoref{sec:fitToyDetResponse}.\footnote{The differences between models are not
  significant enough to warrant concern, especially in light of the uncertainty
  we assign to the \LiHe\ spectrum, as discussed in XXX ref toy MC.} Crucially,
this model, produced by the BCW analysis group,\footnote{Brookhaven, Caltech,
  and William \& Mary.} includes nonlinearity curves for alphas and neutrons,
whose kinetic energies were also considered when determining the prompt energy
for each event. The resulting prompt energy spectra could then be combined
according to the best estimate of the relative proportions of $^9$Li and
$^8$He. For the sake of this analysis, a nominal 5.5\% fraction of $^8$He was
used.\footnote{The measured spectrum and the rate fit both give results that are
  consistent with zero $^8$He, while rough predictions (XXX cite?) indicate that
  the $^8$He proportion should not exceed 20\%. Meanwhile, the predicted
  spectrum does not change very significantly when the fraction is varied from 0
  to 20\%, in comparison to the other sources of uncertainty (neutron and alpha
  quenching). Hence, 5.5\%, an ``inherited'' feature of this analysis, is as
  good a guess as any.} \autoref{sec:fitToyBackgrounds} discusses the
uncertainty assigned to the \LiHe\ spectrum.

XXX show plot overlaying measured and predicted spectra.

\section{Cosmogenic fast neutrons}

XXX ref doc-10948. Fig. 25 from long paper.

As cosmic muons travel through the rock and other materials surrounding the ADs,
they can eject fast neutrons from the medium. If a fast neutron of the
appropriate energy thermalizes and stops inside the AD, it will produce an
IBD-like coincidence pair, in which the prompt signal consists largely of
scintillation from scattered protons, and the delayed signal results from nGd
capture. This process leads to a significant correlated background, amounting to
some 20-30\% of that produced by cosmogenic isotopes. Two methods have been
developed for estimating this background, the so-called \emph{extrapolation} and
\emph{scaling} methods.

In the extrapolation method, the prompt energy cut of the IBD selection is
extended past 12 MeV to 100~MeV or beyond, where true IBDs are completely absent
and the (largely flat) spectrum consists almost entirely of fast neutrons. This
spectrum is then extrapolated below 12~MeV to estimate the fast neutron
component of the IBD sample, using a fit to a well-motivated model of the fast-n
spectrum.

In the scaling method, a search is performed for ``muon-tagged'' IBD-like events
in the immediate aftermath of ``peripheral'' muons that only trigger the outer
water pool and/or RPC. As with the extrapolation method, the prompt energy cut
is significantly extended. In the region above 12 MeV, where true IBDs are
absent, the muon-tagging efficiency can be determined from the ratio of
muon-tagged to untagged events. Then, below 12 MeV, the tagged spectrum (which
contains very few true IBDs due to the short post-muon time window searched) is
rescaled according to the tagging efficiency, yielding an estimate of the fast
neutron spectrum within in the sub-12~MeV region.

The two methods are consistent to within 1--3\% (an order of magnitude smaller
than the estimated uncertainty of each method), providing a high level of
confidence in the estimation. In the following sections, we describe these
methods, and their results (obtained by Hu, Ji, and Treskov, whom we abbreviate
as HJT), in further detail.

\subsection{Event selection}
\label{sec:fastn_sel}

Two event samples are used in the fast neutron analysis. The first, which we
shall refer to as the ``untagged'' sample is obtained using the standard IBD
selection, as described in \autoref{chap:selection}, with the modification that
the upper limit on prompt energy is extended to 300~MeV instead of the usual
12~MeV. In this sample, the prompt spectrum below 12~MeV is essentially the same
as the one used in the oscillation fit (i.e., dominated by true IBDs), whereas
the high-energy region almost exclusively contains fast neutrons.

The other, ``tagged,'' sample contains IBD-like events that occur right after a
muon that triggers \emph{only} the outer water pool. When a muon only strikes
the OWS, as opposed to the IWS or AD, most of the muon-generated debris is
unable to penetrate into the GdLS, but fast neutrons are an exception. This
tagging therefore provides a highly pure sample of fast neutrons for analysis.

The tagged sample is obtained by extending the upper cut to 300~MeV (as in the
untagged sample) while disabling the standard muon veto. An additional
requirement is that the prompt signal be timestamped within (-300, 600)~ns of an
OWS trigger (defined by NHit > 15, in this case). Furthermore, the delayed
signal must occur at least 15 $\mu$s after the muon (to eliminate Michel
electrons), and there must be no AD or IWS muons within 600 $\mu$s of the OWS
muon. This selection results in fairly low statistics (amounting to a few
hundred events in the near halls), but the size of the sample is still
sufficient to provide strong constraints on the fast-neutron background.

\subsection{Scaling method}
\label{sec:fastn_scaling}

In the scaling method, we assume that the shape of the tagged spectrum is an
accurate representation of the shape of the fast-neutron background. In other
words, we assume that, for any given fast neutron, the probability of an
associated OWS trigger is independent of the neutron's energy. Previous studies
within the collaboration have supported the validity of this assumption.

\def\emax{\ensuremath{E_\mathrm{max}}} \def\ntag{\ensuremath{N_\mathrm{tag}}}
\def\nuntag{\ensuremath{N_\mathrm{untag}}}

We define the scaling factor $F$ as \[ F(\emax) = \frac{\nuntag[12,
    \emax]}{\ntag[12, \emax]}, \] i.e., the ratio of the integral of the two
samples between 12~MeV and \emax. Essentially, $F$ represents the efficiency of
the OWS-tagging procedure. The dependence on \emax\ reflects the arbitrary
choice of the upper energy limit, which contributes to the uncertainty on the
result (primarily due to stastical fluctuations in the small sample of tagged
neutrons). HJT quantify this uncertainty by comparing the results for \emax\ of
80, 100, 120, and 150~MeV. In addition, for a \emph{fixed} \emax, there is a
purely statistical uncertainty,
\[ \sigma_F(\emax) = F \cdot \sqrt{\nuntag^{-1}[12, \emax] + \ntag^{-1}[12,
    \emax]},
\]
which also contributes to the final uncertainty.

\def\nfn{\ensuremath{N_\mathrm{fid}}} \def\rfn{\ensuremath{R_\mathrm{FN}}}

In the untagged spectrum, the number of fast neutrons within the fiducial energy
range of [0.7, 12]~MeV is determined simply as
\[ \nfn = F \cdot \ntag[0.7, 12]. \] Its statistical uncertainty, in turn, is
\begin{equation}
  \label{eq:fastn_scal_unc}
  \sigma_\mathrm{FN} = \sqrt{\ntag^2[0.7, 12]
    \cdot \sigma_F^2 + F^2 \cdot \ntag[0.7, 12]}.
\end{equation}
Finally, the normalized daily fast neutron rate is
\begin{equation}
  \label{eq:fastn_rate}
  \rfn = \frac{\nfn}{T_\mathrm{DAQ} \cdot \epsilon_\mu \cdot \epsilon_m},
\end{equation}
where $T_\mathrm{DAQ}$, $\epsilon_\mu$, and $\epsilon_m$ are the DAQ livetime,
muon veto efficiency, and multiplicy cut efficiency for the untagged
sample. This value is then combined with the extrapolation result, described
next, and a total uncertainty is assigned to the combination.

\subsection{Extrapolation method}
\label{sec:fastn_extrap}

Previous simulation studies have found that the fast neutron spectrum can be
accurately described by the p.d.f.
\[ f(E) = A \cdot \left( \frac{E}{E_0} \right)^{-a-E/E_0}, \] where $E_0$ and
$a$ are fitted shape parameters, and $A(E_0, a)$ normalizes the p.d.f. When we
float an additional normalization factor $N$ and perform a fit to some portion
of the fast neutron spectrum, the best-fit $N$ represents the number of events
``under the curve'' from 0 to $\infty$~MeV.

In turn, if we have a (hypothetical) pure fast-neutron spectrum containing
$N_\mathrm{fid}$ events in the fiducial region of [0.7, 12]~MeV, then the full
spectrum ([0, $\infty$]~MeV) should contain an event count equal to
\[ N_\mathrm{tot} = \frac{N_\mathrm{fid}}{\int_{0.7}^{12} f(E; E_0, a)\,dE }. \]
This $N_\mathrm{tot}$ is, by definition, the same as the $N$ introduced in the
previous paragraph. Therefore, when we fit a portion of the fast neutron
spectrum to the form
\begin{equation}
  \label{eq:fastn_extrap_form}
  S(E) = \frac{N_\mathrm{fid} \left( \frac{E}{E_0} \right)^{-a-E/E_0}}
  {\int_{0.7}^{12} \left( \frac{E}{E_0} \right)^{-a-E/E_0} },
\end{equation}
(in which $A$ cancels out) the best fit $N_\mathrm{fid}$ indicates the number of
events in the fiducial region. The key to the extrapolation method is that this
fit is performed outside the fidicial region, where the fast-n sample is
uncontaminated.

Prior to carrying out the extrapolation procedure, HJT verified the validity and
robustness of \eqref{eq:fastn_extrap_form} by fitting it to their WP-tagged
samples. They used four fitting ranges, all starting at 0.7~MeV, and ending at
80, 100, 120, and 150~MeV. In all cases, they obtained satisfactory
goodness-of-fit and consistent values of $E_0$. The disabling of the $a$
parameter was found to introduce negligible differences.

Finally, the fit was performed on the untagged sample, using the same four upper
limits as before, but with the lower limit set to 12~MeV. The spread between the
resulting four values was incorporated into the total uncertainty, as described
in the next section. As with the scaling method, each value of $\nfn$ was
converted to a livetime- and efficiency-normalized daily rate according to
\eqref{eq:fastn_rate}.

\subsection{Final result and total uncertainty}
\label{sec:fastn_comb}

In total, Hu, Ji, and Treskov each obtained a total of eight estimates of \rfn\
for each hall, derived from four scaling ranges in the scaling method, and four
fitting ranges in the extrapolation method. We use Hu's results, as her
selection cuts were essentially identical to ours. She arbitrarily chose the
12-100~MeV scaling method as the source her nominal fast neutron rates, and it
is those numbers that we employ in our own analysis.

Six uncertainties were added in quadrature to obtain the total. The first is the
statistical uncertainty on the scaling factor $F$, described by
\eqref{eq:fastn_scal_unc}. The second is the uncertainty from the choice of
scaling range, which was determined from the difference between the highest and
lowest fast-n rates across the four ranges used. The third is the analogue for
the choice of fitting range. The fourth is the statistical uncertainty in the
fitted value of $N_\mathrm{fid}$. The fifth is the systematic uncertainty
resulting from the dependence of fit results on the choice of binning, which was
determined by varying the bin widths and repeating the fits. Finally, the sixth
component was obtained from the difference in results between the scaling and
extrapolation methods. These uncertainties are summarized in Table~XXX, and the
final results given in Table~XXX.

\newcommand\AmC{$^{241}$Am-$^{13}$C}

\section{AmC source}

To study the response of the detectors to neutrons, each AD was initially
configured with a low-intensity ($\sim$0.7~Hz) \AmC\ neutron source in each of
the three automated calibration units (ACUs) housed on the detector's lid. The
\AmC\ sources are used, for instance, in determining the ratio of visible
energies between $^{60}$Co and n-Gd events; this ratio is a necessary input in
the calibration of the AdScaled reconstruction. In contrast to more traditional
neutron sources such as $^{252}$Cf and $^{241}$Am-$^{9}$Be, the \AmC\ source was
designed to avoid multi-neutron and gamma-neutron cascades (a potential
correlated background). Furthermore, when not in use, the location and shielding
of the sources ensures that neutrons do not infiltrate the GdLS region,
protecting against correlated backgrounds from proton recoils followed by
neutron capture.

In spite of these precautions, a rare mechanism can still produce correlated
backgrounds: A neutron may scatter inelastically in the stainless steel against
Fe, Cr, Mn, or Ni, producing prompt gammas (generally totalling less than 3
visible MeV), before thermalizing and being captured either by one of these four
elements or by Gd in the GdLS overflow tank (producing a signal between 6 and 12
visible MeV). Although many of these gammas do not reach the scintillator, some
do, and when this happens for both the prompt and delayed gammas, the resulting
pair can sometimes pass the IBD selection criteria.

The first experimental suggestion of this background came from an observed
excess of neutron-like (i.e. 6--12 MeV) \emph{uncorrelated} events in the top
half of the detector. The expected neutron-like events (primarily fast neutrons
and decays of cosmogenic $^{12}$B) are predicted to display vertical symmetry
(as was explicitly confirmed for a high-purity sample of $^{12}$B candidates),
so the asymmetric nature of this excess implicated the \AmC\ source as the
origin. Subsequent MC simulations showed that these uncorrelated events (from
neutron capture in the SS or Gd overflow tank) are associated with a correlated
background. The evaluation of this so-called AmC background is detailed in
\cite{Gu_2016}; here we briefly summarize the procedure and its results.

At first glance, the AmC background could be measured simply by removing the AmC
sources from one detector and comparing the IBD candidate rate to an adjacent
detector. However, given the low rate of this process (0.2--0.3 events/day/AD
when all three AmC sources are installed) and the substantial ``background''
from IBDs, it would take an impractical amount of time to accumulate sufficient
statistics, and the AdScaled calibration would be impaired in the meantime. On
the other hand, it \emph{is} possible to directly measure the rate
($\sim$230/day, initially) of \emph{uncorrelated} events from neutron capture in
the stainless steel. Furthermore, MC simulations can be used to relate the rates
of uncorrelated and correlated events. Finally, a high-activity AmC source (HAS)
can be used to benchmark the MC. These three insights together enable a
relatively precise (i.e. to within 45\%) determination of the AmC background.

The following is the fundamental relationship used in the AmC background
estimation:
\begin{equation*}
  R_{\mathrm{corr}} = R_{\mathrm{uncorr}} \times \xi = R_{\mathrm{uncorr}} \times \int_{0.7}^{\SI{12}{MeV}} f(E)\,dE.
\end{equation*}
Here, $R_{\mathrm{uncorr}}$ is the rate of uncorrelated neutron-like events
produced by the AmC source, as measured directly from ordinary data. Meanwhile,
$\xi$ is the ratio of correlated to uncorrelated events, as determined from a
combination of simulations and HAS data. $f(E)$ is simply the differential value
of $\xi$ as a function of prompt energy; i.e. $f(E)\,dE$ is the number of
correlated events with $E_{\mathrm{p}}$ in [$E$, $E + dE$], per uncorrelated
neutron-like event.

$R_{\mathrm{uncorr}}$ was measured trivially, simply by taking the difference in
the number of neutron-like events between the top and bottom halves of each
AD. All of the complexity lay in the determination of $\xi$. In principle, $\xi$
could be extracted directly from a MC simulation. Although the Daya Bay MC
contains a detailed and accurate modeling of the detector materials and geometry
(including the ACUs and the AmC sources themselves), any inaccuracies in the
simulated physics could bias the obtained value of $\xi$. Accordingly, the HAS
was designed and deployed in order to validate the simulations.

Compared to the AmC calibration source, with a rate of $\sim$0.7~Hz, the HAS was
much more intense, producing $\sim$59 neutrons/s. Furthermore, the enclosure of
the HAS consisted of a nearly solid cylinder of stainless steel, in order to
maximize the rate of neutron captures and inelastic scatters. The HAS was placed
on the lid of EH3-AD1 (AD4) in the summer of 2012, and data was collected for
ten days. The number of uncorrelated HAS-induced neutron-like events was
determined by subtracting the neutron-like samples between AD4 and the adjacent
AD5, which observed $\sim$50,000 and $\sim$4,000 events, respectively.

Meanwhile, the number of \emph{correlated} AmC events was measured by taking the
spectrum of IBD candidates in AD4, subtracting the accidental background, and
then subtracting the (background-subtracted) IBD sample measured by
AD5.\footnote{This procedure doesn't account for other correlated backgrounds in
  AD4, such as $^9$Li and fast neutrons, but their rates of $< 0.2$/d are
  insignificant compared to the 63 correlated events per day produced by the
  HAS.} Relating the uncorrelated and correlated rates gave a value of $\xi$,
\emph{for the HAS}, of $(1.5\pm0.3)\times10^{-3}$. The Geant4 MC, on the other
hand, returned a $\xi$ of $(1.2\pm0.1)\times10^{-3}$ for the HAS. The 25\%
difference versus data was then assigned as an uncertainty of the MC. With the
addition of the 20\% statistical uncertainty on the data, a total uncertainty of
30\% was assigned to $\xi$.

Compared to the HAS, the ordinary, i.e. low-intensity, AmC source (LAS) is
expected to have a lower $\xi$, since it lies farther from the AD and has a
lower density of surrounding stainless steel. For the LAS, the MC predicted a
$\xi$ of $0.9\times10^{-3}$. Based on the MC/data comparison for the HAS, this
value was scaled up by 25\% to $1.125\times10^{-3}$, with an uncertainty of
30\%.

In addition to the rates, the prompt spectrum of the AmC background also
required determination. Fortunately, excellent agreement was found between the
data and MC prompt spectra for the HAS. Furthermore, similar agreement was found
between the MC HAS and MC LAS prompt spectra, in spite of the differences in
geometry and material between the HAS and the LAS. As such, any one of these
spectra could have been chosen as a reference. The choice was made to use the
measured HAS spectrum, which was fit to an exponential function,
\begin{equation*}
  f(E) = p_0 \times e^{-E/p_1}.
\end{equation*}
The fit gave $p_1 = \SI{0.783}{MeV}$ with a 10\% statistical uncertainty,
compared to \SI{0.794}{MeV} and \SI{0.830}{MeV} for the LAS and HAS MC samples,
respectively. This 5\% spread, in combination with the 10\% statisical
uncertainty, gave a conservative total uncertainty of 15\% on $p_1$ (essentially
a shape uncertainty). Meanwhile, $p_0$ was fixed by the normalization condition
$\int f(E)\,dE = \xi$, giving (for $\xi = 1.125\times10^{-3}$) $p_0 =
\SI[parse-numbers = false]{3.606\times10^{-3}}{/MeV}$. Conservatively combining
the 30\% uncertainty on $\xi$ with the 15\% uncertainty on $p_1$ gave a total
uncertainty of 45\% on the AmC background. Given the identical design of the
ADs, identical behavior was assumed with respect to the AmC background, and no
attempt was made to calculate AD-specific quantities.

After the determination of $\xi$, the prompt spectrum (i.e. $p_1$), and the
uncertainty, evaluation of each AD's AmC background then amounted to the simple
task of measuring $R_{\mathrm{uncorr}}$ (after correcting for the muon veto
efficiency) and multiplying it by $\xi$, resulting in the final values used in
this analysis. It should be noted that the ACU-B and ACU-C AmC sources were
removed from the EH3 ADs in 2012, during installation of EH2-AD2 and
EH3-AD4.\footnote{In principle, the effective value of $\xi$ could vary between
  the three-sources and one-source scenarios, but this subtlety is not discussed
  in \cite{Gu_2016}. Presumably, any such effects fall within the 45\%
  uncertainty.} This significantly reduced the AmC background at the far site
from 0.3\% to 0.1\% of the IBD rate. Furthermore, over the first two years of
data, a 50\% decline was observed in the rate from each AmC source, in all three
halls, likely due to leakage of scintillator into the source enclosures. This
led to an ultimate background rate of only 0.05\% and 0.03\%, near and far
(although the mean rate over the entire data sample is higher, due to the fact
that earlier rates were higher.)

\newcommand\alphN{(\alpha,\mathrm{n})} \newcommand\CanO{^{13}\mathrm{C}(\alpha,
  \mathrm{n})^{16}\mathrm{O}}

\section{$\CanO$}

A final and relatively minor ($\lesssim$ 0.01\%/0.07\% near/far) correlated
background arises from $\alphN$ reactions initiated by natural radioactivity
within the detector. In these reactions, an alpha from natural radioactivity is
captured by a nucleus, which then emits a neutron. A prompt signal arises from a
number of sources of energy deposition, including the kinetic energy of the
alpha, gammas from nuclear deexcitation (including potentially those from
inelastic scattering of the neutron), and proton recoils caused by the
neutron. This prompt signal is then followed by capture of the neutron,
mimicking the signature of an IBD.

Based on the chemical composition of the scintillator and the known cross
sections of $\alphN$ reactions, it was determined that $\CanO$ is the only such
reaction to occur in the ADs at any significant rate. Meanwhile, there are three
natural decay chains that can lead to alpha activity in the AD: The so-called
uranium, thorium, and actinium chains, which begin, respectively, with the
long-lived isotopes $^{238}$U, $^{232}$Th, and $^{235}$U\footnote{In practice,
  the rate of the thorium chain is determined by the concentration of the
  shorter-lived $^{228}$Th ($t_{1/2}$ = 1.9~yr), and likewise, for the actinium
  chain by $^{227}$Ac ($t_{1/2}$ = 21.8~yr).}. Given that U, Th, and Ac all have
similar chemical properties to Gd, a small amount of contamination is difficult
to avoid during the Gd-doping process. In addition to these three decay chains,
additional alpha activity comes from the decay of $^{210}$Po, a moderately
stable ($t_{1/2}$ = 138~d) daughter of $^{222}$Rn (which itself comes from the
uranium chain). $^{210}$Po was deposited on detector surfaces by $^{222}$Rn
during detector construction, and is essentially the only significant alpha
emitter outside the GdLS region.

Quantifying the $\CanO$ background consists of two parallel tasks. One task is
to determine the level of alpha activity produced by the three decay chains and
by $^{210}$Po. The other task is to determine, for the set of alphas produced by
a given chain, the probability and prompt spectrum of $\CanO$ events. These two
pieces of knowledge can then be combined to yield a predicted rate and spectrum
for the $\CanO$ background.

The three chains all share a fortuitious property that enables a straightforward
estimation of their rates. Namely, they each contain a rapid $\alpha$-$\alpha$
or $\beta$-$\alpha$ cascade whose time correlation and energy distribution allow
for clean extraction from the data. For the uranium, thorium, and actinium
chains, these cascades are, respectively, $^{214}$Bi $\to$ $^{214}$Po $\to$
$^{210}$Pb, $^{212}$Bi $\to$ $^{212}$Po $\to$ $^{208}$Pb, and $^{219}$Rn $\to$
$^{215}$Po $\to$ $^{211}$Pb, with Po half-lives of \SI{164.3}{\micro s},
\SI{0.3}{\micro s}, and \SI{1.781}{ms}.

To extract these events, time coincidence windows of [10, 400]~\us, [1, 3]~\us,
and [1, 4]~ms were used, respectively. Accidentals (most significant for the
actinium chain's Po cascade) could be subtracted via the usual procedure (XXX
check that this was actually done, at least for actinium), and for the uranium
chain's Po cascade, contamination with nH IBDs was not an issue given that the
(quenched) delayed alpha energy for these events is around 1-1.5~MeV,
significantly below the 2.2~MeV nH peak. For the thorium chain, the prompt
spectrum had to be extrapolated below 0.5~MeV in order to determine the total
rate; otherwise, there were no major complications. Under the assumption that
each chain is in equilibrium\footnote{Up to $^{228}$Th and $^{227}$Ac for the
  thorium and actinium chains, rather than all the way up to $^{232}$Th and
  $^{235}$U, as noted previously.}, the rate of the polonium cascade gives the
rate of the entire chain.

For the first two years of data, this procedure determined rates of 0.009, 0.16,
and 0.2~Bq for U, Th, and Ac. Since the U chain is initiated by $^{238}$U in the
AD, its rate is essentially constant, given the $^{238}$U half-life of
4.5~Gyr. On the other hand, the Th and Ac rates do decrease over time, since the
parent half-lives (i.e. those of $^{228}$Th and $^{227}$Ac) are 1.9 and
21.8~yr. To determine the average Th and Ac rates for the 5+ year dataset used
in this analysis, we account for this decrease and obtain XXX.

For $^{210}$Po, a single decay instead of a chain, time correlations could not
be exploited. Instead, 5.3~MeV alphas produced by this isotope were, after
quenching, visible as a peak around 0.5~MeV in the singles spectrum. Fitting
this peak gave an average rate, for the first two years, of 0.7~Hz. Based on the
$^{210}$Po half-life of 138~d, this was extrapolated to give an average rate in
our dataset of XXX.

For a given decay chain (or $^{210}$Po), the set of emitted alphas is known. For
each of these alphas, in turn, simulations can be used to determine the rate and
prompt spectrum of $\CanO$ events. At each step in the simulation, the alpha
loses some energy and travels some distance according to its $dE/dx$ profile in
the LS. With some probability (i.e. cross section), during this step the alpha
may be captured, producing one of the excited states of $^{17}$O. If this
happens, the $^{17}$O will emit a neutron, whose energy depends on both the
initial excited state of $^{17}$O and the final (excited or ground) state of
$^{16}$O. The neutron produces prompt energy through proton recoils and, if it
is sufficiently energetic, may scatter inelastically on $^{12}$C to produce a
$\sim$5~MeV gamma. Additional prompt energy will come from deexcitation gammas
if the $^{16}$O had been produced in an excited state. Repeating this procedure
in a large Monte Carlo sample then gives the desired rate and prompt spectrum of
$\CanO$ events for each alpha source.

Uncertainties in the $\CanO$ prediction arise from a number of sources. The
uncertainty coming from the $\alphN$ cross section was estimated by repeating
the MC procedure using two different cross section tables, JENDL and EXFOR. This
suggested an uncertainty ranging from 6.6\% (for $^{210}$Po) up to 27.5\% (for
$^{232}$Ac) \cite{Zhao_2014}. Additional uncertainty could come from the
fundamentals of the MC simulation, i.e., the $dE/dx$ table and the numerical
integration of discrete steps. This was evaluated by comparing the results of
GEANT4 and SRIM, which differed at a negligible level of less than a
percent. Finally, the assumption of decay chain equilibrium, and the efficiency
of the cascade selection, both could introduce additional uncertainty. These are
difficult to explicitly quantify, and given the relative unimportance of the
$\CanO$ background, the decision was made to simply assign a conservative 50\%
total uncertainty to the $\CanO$ estimation

XXX refer to CPC alpha-n paper.

\end{document}
