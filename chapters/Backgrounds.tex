\documentclass[../thesis.tex]{subfiles}

\begin{document}

\chapter{Background subtraction}
\label{chap:bkg}

\section{Accidental coincidences}

\section{Cosmogenic $^9$Li/$^8$He}

\newcommand\linine{$^9$Li}

The dominant correlated background in Daya Bay comes from the short-lived isotopes $^9$Li and $^8$He, which are produced as spallation products of carbon when muons traverse the AD. These two isotopes\footnote{For the sake of brevity, we will henceforth collectively refer to the two isotopes as \linine; this is by far the dominant of the two, as will be shown later.} have livetimes on the order of hundreds of milliseconds; thus, while the majority of them will decay within the $\sim$1~s veto window that follows AD muons, a non-negligible fraction will survive past it. When they undergo beta decay, the daughter nuclei break apart, and neutrons are contained among the debris. The combination of the beta decay and the subsequent nGd capture produce the characteristic double-coincidence signature of an IBD event.

At the statistical level, the key difference between IBDs and \linine\ decays is that the latter are correlated in time with AD muons. This correlation can be exploited to measure the total rate of \linine\ decays. To do so, the IBD candidate sample is used to construct a histogram of the time since the last AD muon. In this histogram, true IBDs will exhibit an exponential distribution with a characteristic time $\tau$ corresponding to the muon rate, while \linine\ decays will be similarly distributed but with a much shorter $\tau$ corresponding to the \linine\ lifetime. By fitting the histogram with a sum of these exponential distributions, it is possible to extract the \linine\ rate.

Although this method is simple in theory, a significant challenge arises from the fact that a minimum muon energy must be defined when calculating the time between each event and its most recent preceding muon. If this cut is too low, then the time between muons will be comparable to the \linine\ livetime, and with finite statistics, it will be difficult to reliably distringuish between the two components in the fit. Conversely, if the muon cut is too high, then some fraction of \linine-producing muons will be discarded, and those \linine\ events will not appear to be muon-correlated, leading to an underestimation of the rate.

To resolve this issue, one can repeat the \linine\ fit for a variety of muon cuts, and then attempt to extrapolate the results down to a muon cut of (nearly) zero. More concretely, one can obtain good fits at muon cuts of above $\sim 2\times10^5$~p.e., and marginal fits down to $1\times10^5$ p.e., and then perform this extrapolation. Due to the absence of data points below $1\times10^5$~p.e., and the lack of knowledge regarding the manner in which the \linine-production rate scales with muon energy, this extrapolation dominates the final systematic uncertainty of the result.

Improving this extrapolation requires the addition of data points at lower muon cuts. This can be accomplished by modifying the IBD selection in order to enrich it in \linine, resulting in an improved ability to distinguish the components in the time-since-last-muon fit. Two methods have been developed to achieve this: Neutron tagging, and restriction of the prompt energy cut. The first method takes advantage of the fact that \linine\ decays are more likely than true IBDs to occur in association with an additional neutron (from spallation), and the second uses the fact that the \linine\ spectrum is significantly harder than the IBD spectrum. By applying these cuts, performing the fits, and scaling the results by the additional efficiency of these extra criteria, data points can be obtained at lower muon cut energies.

Neutron tagging allows greater enrichment of \linine, at the cost of increased uncertainty in the additional efficiency. Hence, it is better suited for the lowest of muon cuts, where the prompt energy cut technique provides insufficient enrichment, whereas the latter approach is preferable at higher muon cuts, due to the avoidance of the neutron tagging efficiency. Below, we describe a measurement that uses both techniques, applied where each is the optimal choice.

\section{Cosmogenic fast neutrons}

As cosmic muons travel through the rock and other materials surrounding the ADs, they can eject fast neutrons from the medium. If a fast neutron of the appropriate energy thermalizes and stops inside the AD, it will produce an IBD-like coincidence pair, in which the prompt signal consists largely of scintillation from scattered protons, and the delayed signal results from nGd capture. This process leads to a significant correlated background, amounting to some 20-30\% of that produced by cosmogenic isotopes. Two independent methods have been developed for estimating this background, the so-called \emph{extrapolation} and \emph{scaling} methods.

In the extrapolation method, the prompt energy cut of the IBD selection is extended well past 12 MeV, where true IBDs are completely absent and the (flat) spectrum consists almost entirely of fast neutrons. This spectrum is then extrapolated below 12~MeV to estimate the fast neutron component of the IBD sample, using a fit to a well-motivated model of the fast-n spectrum.

In the scaling method, a search is performed for IBD-like events in the immediate aftermath of water pool muons. As with the extrapolation method, the prompt energy cut is significantly extended. In the region above 12 MeV, where true IBDs are absent, the tagging efficiency can be determined from the ratio of WP-tagged to untagged events. Then, below 12 MeV, the tagged spectrum (which contains very few true IBDs due to the short post-muon time window) is rescaled according to the tagging efficiency, yielding an estimate of the fast neutron spectrum within in the sub-12~MeV region.

The two methods are consistent to within 1--3\% (an order of magnitude smaller than the estimated uncertainty of each method), providing a high level of confidence in the estimation. In the following sections, we describe these methods, and their results, in further detail.

\section{AmC source}

\section{$^{13}\mathrm{C}(\alpha, \mathrm{n})^{16}\mathrm{O}^*$}

\end{document}