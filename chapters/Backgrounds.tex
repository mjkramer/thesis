\documentclass[../thesis.tex]{subfiles}

\begin{document}

\chapter{Background subtraction}
\label{chap:bkg}

\section{Accidental coincidences}

\section{Cosmogenic $^9$Li/$^8$He}

\section{Cosmogenic fast neutrons}

As cosmic muons travel through the rock and other materials surrounding the ADs, they can eject fast neutrons from the medium. If a fast neutron of the appropriate energy thermalizes and stops inside the AD, it will produce an IBD-like coincidence pair, in which the prompt signal consists largely of scintillation from scattered protons, and the delayed signal results from nGd capture. This process leads to a significant correlated background, amounting to some 20-30\% of that produced by cosmogenic isotopes. Two independent methods have been developed for estimating this background, the so-called \emph{extrapolation} and \emph{scaling} methods.

In the extrapolation method, the prompt energy cut of the IBD selection is extended well past 12 MeV, where true IBDs are completely absent and the (flat) spectrum consists almost entirely of fast neutrons. This spectrum is then extrapolated below 12~MeV to estimate the fast neutron component of the IBD sample, using a fit to a well-motivated model of the fast-n spectrum.

In the scaling method, a search is performed for IBD-like events in the immediate aftermath of water pool muons. As with the extrapolation method, the prompt energy cut is significantly extended. In the region above 12 MeV, where true IBDs are absent, the tagging efficiency can be determined from the ratio of WP-tagged to untagged events. Then, below 12 MeV, the tagged spectrum (which contains very few true IBDs due to the short post-muon time window) is rescaled according to the tagging efficiency, yielding an estimate of the fast neutron spectrum within in the sub-12~MeV region.

The two methods are consistent to within 1--3\% (an order of magnitude smaller than the estimated uncertainty of each method), providing a high level of confidence in the estimation. In the following sections, we describe these methods, and their results, in further detail.

\section{AmC source}

\section{$^{13}\mathrm{C}(\alpha, \mathrm{n})^{16}\mathrm{O}^*$}

\end{document}