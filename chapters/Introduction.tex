\documentclass[../thesis.tex]{subfiles}

\begin{document}

\chapter{Neutrino oscillations}
\label{chap:intro}

The Standard Model (SM) of particle physics has proven to be an enormous success. From a handful of ingredients---three gauge groups, three generations of leptons and quarks, and the Higgs field---one can make predictions that agree extraordinarily well with the thousands of experimental measurements that have been made of the behavior of fundamental particles.

However, even if we ignore quantum field theory's inability to incorporate gravity and its reliance on \emph{ad hoc} parameters, there are clear signs of cracks in the Standard Model. It fails to explain dark matter, dark energy, cosmic inflation, the lightness of the electroweak scale, and the matter/antimatter asymmetry of the Universe. These are difficult problems which may take decades to resolve, and there is currently no clarity on how the SM will need to be extended as part of that resolution. On the other hand, the last half-century has produced overwhelming evidence of another Beyond the Standard Model (BSM) effect, one that admits a relatively successful quantitative description: neutrino oscillations.

In the Standard Model, the neutrino is treated as a massless particle. The observed kinematics of nuclear beta decay, for instance, have historically indicated that the neutrino has no discernable mass, at least not in comparison to the masses of other elementary particles. A small but nonzero mass would seem to create a fine-tuning problem; hence, a mass of zero has traditionally held greater appeal. In addition, massless neutrinos avoid the need to consider either introducing right-handed neutrinos (never observed) or Majorana mass terms (not present for any other particle) to the theory.

\section{Neutrinos and the Standard Model}
\label{sec:neuAndSM}

\section{History of neutrino oscillation}
\label{sec:history}

Half a century ago, puzzling experimental results began to appear in the neutrino sector. Deep in South Dakota's Homestake mine, Ray Davis had filled a large tank with tetrachloroethylene and waited as solar neutrinos interacted with chlorine-37 atoms via the reaction
\[ \mathrm{\nu_e + \ ^{37}Cl \longrightarrow \ ^{37}Ar + e^-.} \]
The argon, bubbled out and counted every few weeks, was by the early 1970s clearly indicating a reaction rate that was one-third of predictions based on the Standard Solar Model (SSM). This ``solar neutrino problem'' was interpreted to mean that either the SSM or the experiment was in error, and neutrinos remained, according to the wisdom of the day, massless.

Further cracks arose in the mid-1980s, when the IMB and Kamiokande experiments observed a deficit in the number of charged-current atmospheric $\mathrm{\nu_\mu}$ events relative to expectations, while also confirming the solar neutrino problem, which was yet again confirmed around 1992 by the gallium-using SAGE and GALLEX radiochemical experiments. Four years later, the massive Super-Kamiokande (SK) experiment came online, and in 1998 it published measurements of the zenith angle dependence of the atmospheric neutrino deficit. This geometric dependence was in qualitative agreement with models in which massive neutrinos oscillate between flavors due to mixing of the mass and flavor eigenstates. Other models, such as neutrino decay or decoherence, were disfavored by the data. The best-fit oscillation model provided initial indications of $\theta_{23}$ and $\Delta m^2_{32}$. 

In 2002, the Sudbury Neutrino Observatory (SNO) published definitive evidence of flavor oscillation caused by neutrino mass. Owing to its use of heavy water as the target material, SNO had unprecedented sensitivity to neutral current (NC) interactions, which are undergone by all three neutrino flavors. This was in addition to the usual charged current (CC) sensitivity, which only provides detection of electron neutrinos. As such, SNO was capable of independently measuring both the total and the electron neutrinos fluxes. The total flux was in excellent agreement with the SSM, demonstrating that the ``missing'' neutrinos underlying the solar neutrino problem were, in fact, merely hiding in the form of muon and tau neutrinos.

At this point, neutrino oscillations were no longer a hypothesis to disprove but instead were a phenomenon to be quantified. The precision era had begun, and KamLAND was one experiment that had gotten a head start. The experiment measured the disappearance of reactor antineutrinos over long baselines. These oscillations are controlled by $\theta_{12}$ and $\Delta m^2_{21}$, the former of which had previously been constrained by the solar neutrino problem. KamLAND's measurements were in agreement with the solar neutrino observations, and furthermore provided the most precise measurement of $\Delta m^2_{21}$ to date.

Meanwhile, the atmospheric results of SK and others on $\Delta m^2_{32}$ had set off a flurry of successful long-baseline accelerator experiments, such as K2K, T2K, MINOS, and NO$\nu$A, optimized for this mass splitting and designed to narrow down its value and that of $\theta_{23}$. One mixing angle, however, remained elusive: $\theta_{13}$.

The disappearance of electron neutrinos is controlled by $\theta_{12}$ and $\Delta m^2_{21}$ at longer baselines (such as those employed in solar experiments and KamLAND) and by $\theta_{13}$ and $\Delta m^2_{21}$ at shorter baselines. As luck would have it, at the energies of reactor antineutrinos, the oscillation length is $\sim$1~km, which is a sufficiently short distance that a relatively modest target volume of $\sim$100~tons will provide ample statistics from a typical commercial power reactor. The CHOOZ experiment in France had attempted to observe $\theta_{13}$, but due to uncertanties in the reactor flux normalization, it was only able to set an upper limit of 0.17 on $\sin^22\theta_{13}$.

In order to mitigate this uncertainty, the next generation of reactor experiments were designed using multiple identical detectors at different baselines. This would enable the near detectors to measure the flux while the far detectors measure any oscillation; uncertainties on the absolute flux thus largely cancel in the far/near ratio. The Double CHOOZ, RENO, and Daya Bay experiments embarked on this effort in parallel. This thesis describes Daya Bay's analysis in detail.


\section{Neutrino oscillation physics}
\label{sec:oscPhysics}

\section{Reactor neutrinos}
\label{sec:introReactor}

\section{Relevance of $\theta13$ to future research}
\label{sec:futureRelevance}

\end{document}