\documentclass[../thesis.tex]{subfiles}

\begin{document}

\section{Neutrinos and the Standard Model}
\label{sec:neuAndSM}

The Standard Model (SM) of particle physics has proven to be an enormous success. From a handful of ingredients---three gauge groups, three generations of quarks and leptons, the Higgs field, and 18 free parameters---the SM provides a succinct and precise description of Nature that agrees incredibly well with the bulk of experimental observations.

However, even if we ignore the glaring absence of gravity in the theory (a general problem for the whole class of QFTs), there are clear and exciting signs that new physics must lie beyond the Standard Model. For instance, the SM fails to explain dark matter, dark energy, cosmic inflation, the lightness of the electroweak scale, and the matter/antimatter asymmetry of the Universe, among other puzzles.

These are difficult problems which may take decades to resolve, and there is currently no clarity on how the SM will need to be extended along the way. On the other hand, the last half-century has produced overwhelming evidence of another Beyond the Standard Model (BSM) effect, one that admits a relatively successful quantitative description: neutrino oscillations. Before discussing this phenomenon, it is worthwhile to review the story of the neutrino itself. 

More than a century ago, measurements of nuclear beta decay gave the surprising result that the energy of the outgoing electron was continuously distributed, in stark contrast to the discrete lines observed in alpha and gamma decay. If, like alpha and gamma decay, beta decay were a two-body process, then a continuous spectrum would seem to imply the violation of energy and momentum conservation. Furthermore, it had been observed that nuclear spin is either integral (for even mass numbers) or half-integral (for odd mass numbers), implying that the nuclear spin can only change by an integer during beta decays, which conserve the mass number. And yet, the electron has a spin of 1/2, so a two-body process would also imply the nonconservation of angular momentum. Did all of this mean that, alas, it was necessary to discard the most sacred conservation laws of physics?

An alternative resolution, one that would avoid violations of the conservation laws, was proposed in 1930 by Pauli, who postulated the existence of an unobserved, light, neutral particle, or ``neutron'', contained in the nucleus and emitted along with the electron in beta decay. Chadwick's 1932 discovery of the actual neutron led Fermi to redub Pauli's hypothetical particle as the \emph{neutrino}. Fermi's 1933 theory of beta decay, which incorporated the neutrino, was successful in reproducing the measured electron spectra, but the neutrino itself would not be directly observed for another two decades.

In 1956, antineutrinos from a nuclear reactor were detected by the Cowan-Reines experiment, marking the first direct confirmation of the neutrino's existence. Over the decades that followed, it was found that neutrinos come in three distinct ``flavors''---electron, muon, and tau, corresponding to their charged lepton partners---and that neutrinos have zero (or nearly zero) mass.

The second half of the 20th century saw remarkable progress in the experimental and theoretical elucidation of elementary particle behavior, culminating in the 1970s with the formulation of the Standard Model (SM). In the SM, the three neutrinos are massless left-handed Weyl spinors that interact only via the W and Z bosons. Masslessness is theoretically appealing, as it avoids the need to imbue the theory with right-handed neutrinos (which have never been observed) or Majorana mass terms (which are not present for any other SM particle); in addition, it avoids the need to explain why the neutrino mass is unobservably small in comparison with the other particles of the theory. Although some puzzling neutrino observations, discussed next, had been known since the late 1960s, massive neutrinos were seldom given serious consideration as the explanation. But as experimental anomalies continued to mount, this wall would eventually have to crumble, leading to the revolution that has transpired over the last few decades.

\section{History of neutrino oscillation}
\label{sec:history}

Half a century ago, deep in South Dakota's Homestake mine, Ray Davis filled a large tank with tetrachloroethylene, a common dry-cleaning agent, and waited as solar neutrinos interacted with chlorine-37 atoms via the reaction
\[ \mathrm{\nu_e + \ ^{37}Cl \longrightarrow \ ^{37}Ar + e^-.} \]
The argon, bubbled out and counted every few weeks, was by the early 1970s clearly indicating a reaction rate that was one-third of predictions based on the Standard Solar Model (SSM). This ``solar neutrino problem'' was interpreted to mean that either the SSM or the experiment was in error, and neutrinos remained, according to the wisdom of the day, massless.

Further cracks arose in the mid-1980s, when the IMB and Kamiokande experiments observed a deficit in the number of charged-current atmospheric $\mathrm{\nu_\mu}$ events relative to expectations, while also confirming the solar neutrino problem, which was yet again confirmed around 1992 by the gallium-using SAGE and GALLEX radiochemical experiments. Four years later, the massive Super-Kamiokande (SK) experiment came online, and in 1998 it published measurements of the zenith angle dependence of the atmospheric neutrino deficit. This geometric dependence was in qualitative agreement with models in which massive neutrinos oscillate between flavors due to mixing of the mass and flavor eigenstates. Other models, such as neutrino decay or decoherence, were disfavored by the data. The best-fit oscillation model provided initial indications of $\theta_{23}$ and $\Delta m^2_{32}$. 

In 2002, the Sudbury Neutrino Observatory (SNO) published definitive evidence of flavor oscillation caused by neutrino mass. Owing to its use of heavy water as the target material, SNO had unprecedented sensitivity to neutral current (NC) interactions, which are undergone by all three neutrino flavors. This was in addition to the usual charged current (CC) sensitivity, which only provides detection of electron neutrinos. As such, SNO was capable of independently measuring both the total and the electron neutrinos fluxes. The total flux was in excellent agreement with the SSM, demonstrating that the ``missing'' neutrinos underlying the solar neutrino problem were, in fact, merely hiding in the form of muon and tau neutrinos.

At this point, neutrino oscillations were no longer a hypothesis to disprove but instead were a phenomenon to be quantified. The precision era had begun, and KamLAND was one experiment that had gotten a head start. Around the time of SNO's announcement, KamLAND published results on the disappearance of reactor antineutrinos over long baselines. These oscillations are controlled by $\theta_{12}$ and $\Delta m^2_{21}$, the former of which had previously been constrained by the solar neutrino problem. KamLAND's measurements were in agreement with the solar neutrino observations, and furthermore provided the most precise measurement of $\Delta m^2_{21}$ to date.

Meanwhile, the atmospheric results of SK and others on $\Delta m^2_{32}$ had set off a flurry of successful long-baseline accelerator experiments, such as K2K, T2K, MINOS, and NO$\nu$A, optimized for this mass splitting and designed to narrow down its value and that of $\theta_{23}$. One mixing angle, however, remained elusive: $\theta_{13}$.

The disappearance of electron neutrinos is controlled by $\theta_{12}$ and $\Delta m^2_{21}$ at longer baselines (such as those employed in solar experiments and KamLAND) and by $\theta_{13}$ and $\Delta m^2_{21}$ at shorter baselines. As luck would have it, at the energies of reactor antineutrinos, the oscillation length is $\sim$1~km, which is a sufficiently short distance that a relatively modest target volume of $\sim$100~tons will provide ample statistics from a typical commercial power reactor. The CHOOZ experiment in France had attempted to observe $\theta_{13}$, but due to uncertanties in the reactor flux normalization, it was only able to set an upper limit of 0.17 on $\sin^22\theta_{13}$.

In order to mitigate this uncertainty, the next generation of reactor experiments were designed using multiple identical detectors at different baselines. This would enable the near detectors to measure the flux while the far detectors measure any oscillation; uncertainties on the absolute flux thus largely cancel in the far/near ratio. The Double CHOOZ, RENO, and Daya Bay experiments embarked on this effort in parallel. This thesis describes Daya Bay's analysis in detail.

\section{Neutrino oscillation physics}
\label{sec:oscPhysics}

\section{Reactor neutrinos}
\label{sec:introReactor}

\section{Relevance of $\theta13$ to future research}
\label{sec:futureRelevance}

\end{document}