\documentclass[../thesis.tex]{subfiles}

\begin{document}

\chapter{Introduction}
\label{chap:intro}

\section{Neutrinos and the Standard Model}
\label{sec:neuAndSM}

The Standard Model (SM) of particle physics has proven to be an enormous success. From a handful of ingredients---three gauge groups, three generations of quarks and leptons, the Higgs field, and 18 free parameters---the SM provides a succinct and precise description of Nature that agrees incredibly well with the bulk of experimental observations.

However, even if we ignore the glaring absence of gravity in the theory (a difficulty inherent to quantum field theory itself), there are clear and exciting signs that new physics must lie beyond the Standard Model. For instance, the SM fails to explain dark matter, dark energy, cosmic inflation, the lightness of the electroweak scale, and the matter/antimatter asymmetry of the Universe, among other puzzles.

These are difficult problems which may take decades to resolve, and in most cases there is little clarity on how the SM will need to be extended along the way. On the other hand, the last half-century has produced overwhelming evidence of another Beyond the Standard Model (BSM) effect, one that admits a relatively successful quantitative description: neutrino oscillations. Before discussing this phenomenon, it is worthwhile to review the story of the neutrino itself. 

More than a century ago, measurements of nuclear beta decay gave the surprising result that the energy of the outgoing electron was continuously distributed, in stark contrast to the discrete lines observed in alpha and gamma decay. If, like alpha and gamma decay, beta decay were a two-body process, then a continuous spectrum would seem to imply the violation of energy and momentum conservation. Furthermore, it had been observed that nuclear spin is either integral (for even mass numbers) or half-integral (for odd mass numbers), implying that the nuclear spin can only change by an integer during beta decays, which conserve the mass number. And yet, the electron has a spin of 1/2, so a two-body process would also imply the nonconservation of angular momentum. Did all of this mean that, alas, it was necessary to discard the most sacred conservation laws of physics?

An alternative resolution, one that would avoid violations of the conservation laws, was proposed in 1930 by Pauli, who postulated the existence of an unobserved, light, neutral particle, or ``neutron'', contained in the nucleus and emitted along with the electron in beta decay. Chadwick's 1932 discovery of the actual neutron led Fermi to redub Pauli's hypothetical particle as the \emph{neutrino}. Fermi's 1933 theory of beta decay, which incorporated the neutrino, was successful in reproducing the measured electron spectra, but the neutrino itself would not be directly observed for another two decades.

In 1956, antineutrinos from a nuclear reactor were detected by the Cowan-Reines experiment, marking the first direct confirmation of the neutrino's existence. Over the decades that followed, it was found that neutrinos come in three distinct ``flavors''---electron, muon, and tau, corresponding to their charged lepton partners---and that neutrinos lack any discernable mass. These qualities would eventually be incorporated into the Standard Model, which crystallized in the 1970s after a period of remarkable theoretical and experimental progress in particle physics.

In the SM, the three neutrinos are massless left-handed Weyl spinors that interact only via the W and Z bosons. Masslessness is theoretically appealing, as it avoids the need to imbue the theory with right-handed neutrinos (which have never been observed) or Majorana mass terms (which are not present for any other SM particle); in addition, because beta decay observations indicate that the neutrino mass must be \emph{very small} if it is nonzero, masslessness avoids the need to explain such unnatural smallness in comparison with the other particles of the theory. Although some puzzling neutrino observations, discussed next, had been known since the late 1960s, massive neutrinos were seldom given serious consideration as the explanation. But as experimental evidence continued to mount, this wall would eventually have to crumble, leading to the revolution that has transpired over the last few decades.

\section{History of neutrino oscillations}
\label{sec:history}

Half a century ago, deep in South Dakota's Homestake mine, Ray Davis filled a large tank with tetrachloroethylene, a common dry-cleaning agent, and waited as solar neutrinos interacted with chlorine-37 atoms via the reaction
\[ \mathrm{\nu_e + \ ^{37}Cl \longrightarrow \ ^{37}Ar + e^-.} \]
The argon, bubbled out and counted every few weeks, was by the early 1970s clearly indicating a reaction rate that was one-third of the prediction derived from the Standard Solar Model (SSM). This ``solar neutrino problem'' was interpreted to mean that either the SSM or the experiment was in error, and neutrinos remained, according to the wisdom of the day, massless.

Despite the prevailing belief in masslessness, the possibility of massive neutrinos, and the resulting potential for oscillations, was considered as early as 1962, after the discovery that neutrinos come in separate electron and muon flavors. That year, an obscure and then-unacknowledged paper by Maki, Nakagawa, and Sakata (MNS) introduced the idea. Some five years before MNS, in 1957, Pontecorvo had been the first to propose the idea of neutrino oscillations, but in the form of neutrino-antineutrino oscillations, in analogy with the recently discovered phenomenon of neutral kaon mixing. It was a decade later, in 1967, that Pontecorvo independently raised the prospect of flavor oscillations, going so far as to suggest that solar neutrinos could oscillate, well before the first experimental hints of such by the Davis experiment. Still, it would take decades of further obervations to prove that these four theorists were correct.

The mid-1980s delivered additional anomalies, when the IMB and Kamiokande experiments observed a deficit in the number of charged-current atmospheric $\mathrm{\nu_\mu}$ events relative to expectations, while also confirming the solar neutrino problem, which was yet again confirmed around 1992 by the gallium-based SAGE and GALLEX radiochemical experiments. In 1996, the massive Super-Kamiokande (SK) water Cerenkov detector came online, and in 1998 the SK collaboration published measurements of the zenith angle dependence of the atmospheric neutrino deficit. This geometric dependence was consistent with mass-induced flavor oscillations. Other models, such as neutrino decay or decoherence, were disfavored by the SK data. SK's best-fit oscillation model provided initial indications of $\theta_{23}$ and $\Delta m^2_{32}$, assuming that the model was indeed correct, which would require further confirmation. 

This confirmation came in 2002, thanks to the Sudbury Neutrino Observatory (SNO). Owing to its use of heavy water as the target material, SNO had unprecedented sensitivity to neutral current (NC) interactions, which are undergone by all three neutrino flavors. This NC sensitivity stood in addition to SNO's customary sensitivity to charged current (CC) interactions, which (at solar energies) only provide detection of electron neutrinos. As such, SNO was capable of independently measuring both the total and the electron neutrinos fluxes. The total flux was in excellent agreement with the SSM, demonstrating that the ``missing'' neutrinos underlying the solar neutrino problem were, in fact, merely hiding in the form of muon and tau neutrinos.

At this point, the existence of neutrino oscillations was no longer a question. The precision era had begun, and KamLAND was one experiment that had gotten a head start. Around the time of SNO's announcement, KamLAND published results on the disappearance of reactor antineutrinos over long ($\sim$100~km) baselines. These oscillations are controlled by $\theta_{12}$ and $\Delta m^2_{21}$, the former of which had previously been constrained by the solar neutrino problem. KamLAND's measurements were in agreement with the solar neutrino observations, and furthermore provided what remains the most precise measurement of $\Delta m^2_{21}$ ever performed.

Meanwhile, the atmospheric results of SK and others on $\Delta m^2_{32}$ had set off a flurry of successful long-baseline accelerator experiments, such as K2K, T2K, MINOS, and NO$\nu$A, optimized for this mass splitting and designed to narrow down its value and that of $\theta_{23}$. As the 2010s approached, however, one mixing angle remained elusive: $\theta_{13}$.

The disappearance of electron neutrinos is controlled by $\theta_{12}$ and $\Delta m^2_{21}$ at longer baselines (such as those employed in solar experiments and KamLAND) and by $\theta_{13}$ and $\Delta m^2_{31}$ at shorter baselines. As luck would have it, at the energies of reactor antineutrinos, the latter oscillation length is $\sim$1~km, which is a sufficiently short distance that a relatively modest target mass of $\sim$100~tons will provide ample statistics from a typical commercial power reactor. The CHOOZ experiment in France had attempted to observe $\theta_{13}$, but due to uncertanties in the reactor flux normalization, it was only able to set an upper limit of 0.17 on $\sin^22\theta_{13}$.

In order to mitigate this uncertainty, the subsequent generation of reactor experiments were designed using multiple identical detectors at different baselines. This would enable the near detectors to measure the flux while the far detectors measure any oscillation. Uncertainties on the absolute flux thus largely cancel in the far/near ratio. The Double CHOOZ, RENO, and Daya Bay experiments embarked on this effort in parallel. This thesis describes, in detail, an analysis of Daya Bay's data in order to extract $\theta_{13}$ and the mass splitting.

\section{Neutrino oscillation physics}
\label{sec:oscPhysics}

Neutrino oscillations are the consequence of two facts: First, that the flavor eigenstates are not the same as the mass eigenstates (\emph{mixing}), and second, that the mass eigenvalues aren't fully degenerate (implying that at least one is nonzero). As a result, a flavor eigenstate is a superposition of mass eigenstates which each undergo phase rotation at their own rates; the mass components thus interfere to produce different flavor compositions over time, leading to the observation of oscillations.

\newcommand\upmns{\ensuremath{%
    U_\mathrm{PMNS}}}
\newcommand\flavcol{\ensuremath{%
    \begin{pmatrix} {\nu_e} \\ {\nu_\mu} \\ {\nu_\tau} \end{pmatrix}}}
\newcommand\masscol{\ensuremath{%
    \begin{pmatrix} \nu_1 \\ \nu_2 \\ \nu_3 \end{pmatrix}}}
\newcommand\upmnsSimple{\ensuremath{%
    \begin{pmatrix} U_{e 1} & U_{e 2} & U_{e 3} \\ U_{\mu 1} & U_{\mu 2} & U_{\mu 3} \\ U_{\tau 1} & U_{\tau 2} & U_{\tau 3} \end{pmatrix}}}
% \newcommand\cee[1]{\ensuremath{%
%      \cos\theta_{#1}}}
% \newcommand\ess[1]{\ensuremath{%
%      \sin\theta_{#1}}}
\newcommand\cee[1]{\ensuremath{%
     c_{#1}}}
\newcommand\ess[1]{\ensuremath{%
     s_{#1}}}
\newcommand\upmnsFactored{\ensuremath{%
    \begin{pmatrix} 1 & 0 & 0 \\ 0 & \cee{23} & \ess{23} \\ 0 & -\ess{23} & \cee{23} \end{pmatrix}%
    \begin{pmatrix} \cee{13} & 0 & \ess{13}e^{-i\delta_\text{CP}} \\ 0 & 1 & 0 \\ -\ess{13}e^{i\delta_\text{CP}} & 0 & \cee{13} \end{pmatrix}%
    \begin{pmatrix} \cee{12} & \ess{12} & 0 \\ -\ess{12} & \cee{12} & 0 \\ 0 & 0 & 1 \end{pmatrix}}}
\newcommand\upmnsFull{\ensuremath{%
    \begin{pmatrix} c_{12}c_{13} & s_{12} c_{13} & s_{13}e^{-i\delta_\text{CP}} \\%
      -s_{12}c_{23} - c_{12}s_{23}s_{13}e^{i\delta_\text{CP}} & c_{12}c_{23} - s_{12}s_{23}s_{13}e^{i\delta_\text{CP}} & s_{23}c_{13}\\%
      s_{12}s_{23} - c_{12}c_{23}s_{13}e^{i\delta_\text{CP}} & -c_{12}s_{23} - s_{12}c_{23}s_{13}e^{i\delta_\text{CP}} & c_{23}c_{13} \end{pmatrix}}}
\newcommand*\mathcmd[2]{\newcommand#1{\ensuremath{#2}}}
\mathcmd{\tAB}{\theta_{12}}
\mathcmd{\tBC}{\theta_{23}}
\mathcmd{\tAC}{\theta_{13}}
\mathcmd{\dcp}{\delta_{CP}}

The neutrino fields in the flavor and mass bases are related by the Pontecorvo-Maki-Nakagawa-Sakata (PMNS) matrix, \upmns\ (or simply $U$):
\[\nu_\alpha = U_{\alpha i} \nu_i,\]
that is,
\[ \flavcol = \underbrace{\upmnsSimple}_{\upmns} \masscol. \]
It can then be shown that the \emph{states} (as opposed to the fields) are related by the complex conjugate of $U$:
\begin{equation}
  \label{eq:nuStateRel}
\ket{\nu_\alpha} = U^*_{\alpha i}\ket{\nu_i}.
\end{equation}
\upmns\ can be parameterized in terms of the three mixing angles, \tAB, \tBC, and \tAC, along with a CP-violating complex phase \dcp:\footnote{Although a general 3x3 unitary matrix includes six complex phases, in this case five of them can be absorbed into the definitions of the neutrino and charged lepton fields, leaving only one remaining. It could be inserted anywhere in the factorization of $U$ as long as unitary is preserved, but by convention the definition in (\ref{eq:upmnsParam}) is what is used.}
\begin{align}
  \label{eq:upmnsParam}
  \begin{split}
  U &= \upmnsFactored \\\\
    &= \upmnsFull.
  \end{split}
\end{align}
where $c_{12} \equiv \cos\theta_{12}$, etc.    It is these mixing angles (or trigonometric functions thereof), and not the matrix elements themselves, that are directly measured by experiments.

Physically speaking, neutrinos are produced in processes that are localized in space and time. A fully microscopic treatment of neutrino oscillation would therefore involve modeling each neutrino as a wave packet of definite initial flavor. However, given a bulk flux of neutrinos from a spatially extensive source, the oscillation probability for neutrinos of energy $E$ can be calculated by simply considering a spacetime-filling plane wave. This approach ignores effects caused by interference and decoherence of wave packet components; however, these effects were shown to be completely insignificant under the experimental conditions at Daya Bay [XXX wave packet paper].

\newcommand\nuF{\nu^{\mathrm{F}}}
\newcommand\nuM{\nu^{\mathrm{M}}}

To calculate the oscillation probability over baseline $L$ for a neutrino of energy $E$ and initial flavor $\alpha$, we first redefine the problem slightly and consider a neutrino plane wave $\ket{\nu(t)}$ of initial flavor $\alpha$ and well-defined \emph{momentum} $\vec{p}$, such that $E = |\vec{p}|$.\footnote{One could instead assume a well-defined \emph{energy}, or even \emph{velocity}, and the end result would be the same to leading order. The advantage of fixing the momentum is that it provides a spatially uniform flavor composition at all times, so we need only worry about the time dependence.} This state contains components of three very slightly different energies, one for each mass eigenstate. For clarity, flavor and mass eigenstates will be labeled with the superscripts F and M, respectively. Following [XXX Giunti], we have
\[\ket{\nu(0)} = \ket{\nuF_\alpha} = U^*_{\alpha i} \ket{\nuM_i}. \]
This state evolves in time like
\[\ket{\nu(t)} = e^{-i E_i t} U^*_{\alpha i} \ket{\nuM_i}.\]
Given that we've fixed the momentum $\vec{p}$, the energy $E_i$ of the $i$th mass eigenstate component of $\ket{\nu}$ is
\[E_i = \sqrt{p^2 + m_i^2} \approx p + \frac{m_i^2}{2p} \equiv E + \frac{m_i^2}{2E}.\]
This gives, after we discard the global phase evolution $e^{-iEt}$,
% \[\ket{\nu(t)} = e^{-i\frac{m^2}{2E} t} U_{\alpha i} \nuM_i\]
% \[\ket{\nu(t)} = e^{-i(m^2/2E) t} U_{\alpha i} \nuM_i\]
% \[\ket{\nu(t)} = e^{-itm^2/2E} U_{\alpha i} \nuM_i\]
\begin{align*}
  \ket{\nu(t)} &= \exp\left(-i\frac{m^2t}{2E}\right) U^*_{\alpha i} \ket{\nuM_i} \\
               &\approx \exp\left(-i\frac{m^2L}{2E}\right) U^*_{\alpha i} \ket{\nuM_i},
\end{align*}
since, for a relativistic neutrino, $t \approx L$. The probability of measuring this state to be of flavor $\beta$ is then
\begin{align*}
  P(\alpha \rightarrow \beta)
  &= \left|\Braket{\nuF_\beta|\nu(t)}\right|^2
    = \left| \Bra{\nuM_j} U_{\beta j} \exp\left( -i\frac{m_i^2L}{2E} \right) U^*_{\alpha i}\Ket{\nuM_i}\right|^2 \\
  &=  \left| U^*_{\alpha i} U_{\beta i} \exp\left( -i \frac{m_i^2L}{2E} \right)\right|^2 \\
  % &= U^*_{\beta j} U_{\alpha j} U_{\beta i} U^*_{\alpha i} \exp\left( -i \frac{(m_j^2 - m_i^2)L}{2E} \right)
  &= U^*_{\alpha j} U_{\beta j} U_{\alpha i} U^*_{\beta i} \exp\left( -i \frac{\Delta m^2_{ji}L}{2E} \right),
\end{align*}
where $\Delta m^2_{ji} \equiv m^2_j - m^2_i$. This can be rewritten by separating the terms for $i = j$ and $i \neq j$, giving
\begin{equation}
  \label{eq:oscSepSums}
P(\alpha \rightarrow \beta) = \sum_{j} |U_{\alpha j}|^2 |U_{\beta j}|^2
  + 2 \Re \sum_{j>i} U^*_{\alpha j} U_{\beta j} U_{\alpha i} U^*_{\beta i}
  \exp\left( -i \frac{\Delta m^2_{ji}L}{2E} \right),
\end{equation}
where, for clarity, we are explicitly indicating the summations. Going further, we can employ the unitarity relation
\[U_{\alpha j} U^*_{\beta j} = \delta_{\alpha \beta}\]
which, upon squaring, gives
\[ \sum_j |U_{\alpha j}|^2 |U_{\beta j}|^2 = \delta_{\alpha \beta} -
  2 \sum_{j > i} \Re(U^*_{\alpha j} U_{\beta j} U_{\alpha i} U^*_{\beta i}).\]
Substituting this into (\ref{eq:oscSepSums}), and using Euler's identity, along with the trigonometric identity $1 - \cos 2\varphi = 2\sin^2 \varphi$, we finally get
\begin{align}
  \label{eq:oscReIm}
  \begin{split}
    P(\alpha \rightarrow \beta) = \delta_{\alpha \beta}
    &- 4\sum_{j > i} \Re(U^*_{\alpha j} U_{\beta j} U_{\alpha i} U^*_{\beta i})
    \sin^2 \left( \frac{\Delta m^2_{ji}L}{4E} \right) \\
    &+ 2\sum_{j > i} \Im(U^*_{\alpha j} U_{\beta j} U_{\alpha i} U^*_{\beta i})
    \sin \left( \frac{\Delta m^2_{ji}L}{2E} \right). \\
  \end{split}
\end{align}
Note that this result applies to \emph{neutrinos,} not antineutrinos. For the latter, we have, analogously to (\ref{eq:nuStateRel}),
\[\ket{\nubar_\alpha} = U_{\alpha i}\ket{\nubar_i}\]
(note the lack of complex conjugation on the matrix element). The preceding derivation then produces (\ref{eq:oscReIm}) with the complex conjugations swappped.

In an experiment like Daya Bay, where we are looking for the \emph{disappearance} of (anti)neutrinos of a given flavor, our interest is in $P(\alpha \rightarrow \alpha)$. In this case, the behavior of neutrinos and antineutrinos is the same, and (\ref{eq:oscReIm}) simplifies to
\[P(\alpha \rightarrow \alpha) = 1 - 4 \sum_{j>i} |U_{\alpha j}|^2 |U_{\alpha i}|^2
\sin^2\left( \frac{\Delta m^2_{ji} L}{4E} \right). \]
The matrix elements from (\ref{eq:upmnsParam}) can then be inserted in order to derive the disappearance probability for a particular flavor. For electron antineutrinos, as at Daya Bay, the survival probability is
\begin{align}
  \label{eq:survProbDybFull}
  \begin{split}
    P(\nubar_e \rightarrow \nubar_e) =
    1 &- \cos^4\theta_{13}\sin^2 2\theta_{12} \sin^2 \Delta_{21} \\
    &- \sin^2 2\theta_{13}\left( \cos^2\theta_{12} \sin^2 \Delta_{31}
    + \sin^2\theta_{12} \sin^2 \Delta_{32}\right),
  \end{split}
\end{align}
where we've introduced the notation
\begin{align*}
  \Delta_{ji} &\equiv \frac{\Delta m^2_{ji}L}{4E}
                \approx \frac{1.267\, \Delta m^2_{ji}\,\mathrm{[eV^2]}\;
                L\,\mathrm{[m]}}{E\,\mathrm{[MeV]}}.
\end{align*}

At Daya Bay, $\sin^2 \Delta_{21}$ is so small that the experiment has no ability to constrain $\theta_{12}$ or $\abs{\Delta m^2_{21}}$. In this analysis, then, their values are fixed to those measured by KamLAND (XXX global?). This leaves $\theta_{13}$, $\abs{\Delta m^2_{31}}$, and $\abs{\Delta m^2_{32}}$. However, given that the difference between $\abs{\Delta m^2_{31}}$ and $|\Delta m^2_{32}|$ is less than one part in thirty, distinguishing between their corresponding phases would require some combination of a high-resolution detector and a baseline long enough to stretch out the phase difference. At Daya Bay, the baseline of $\sim$1~km is approximately one oscillation length, where the detector resolution is insufficient for resolving the difference. Therefore, Daya Bay cannot measure the two splittings jointly. On the other hand, since $\Delta m^2_{21}$ is well constrained and known to be positive, we can tightly relate $\abs{\Delta m^2_{31}}$ and $\abs{\Delta m^2_{32}}$:
\[ \abs{\Delta m^2_{32}} =
  \begin{cases*}
    \abs{\Delta m^2_{31}} - \Delta m^2_{21} & normal hierarchy ($\Delta m^2_{31} > 0$), \\
    \abs{\Delta m^2_{31}} + \Delta m^2_{21} & inverted hierarchy ($\Delta m^2_{31} < 0$), \\
  \end{cases*}
\]
Thus, by using this relation to eliminate one parameter, it is possible to to perform a two-parameter fit of (\ref{eq:survProbDybFull}) directly, \emph{provided that the mass hierarchy is specified}. Since the mass hierarchy is currently unknown, two sets of results must be reported, and they are subject to change if an improved determination of $\Delta m^2_{21}$ is ever published.

Alternatively, we can recast (\ref{eq:survProbDybFull}) in a form that refers only to an empirical \emph{effective} mass splitting $\Delta m^2_{ee}$:
\begin{equation}
  \label{eq:survProbDybEE}
  P(\nubar_e \rightarrow \nubar_e) \approx 1 - \cos^4\theta_{13}\sin^2 2\theta_{12} \sin^2 \Delta_{21}
  - \sin^2 2\theta_{13}\sin^2\Delta_{ee},
\end{equation}
where
\[ \Delta_{ee} \equiv \frac{\Delta m^2_{ee}L}{4E}. \]
It must be noted that (\ref{eq:survProbDybEE}) is not an exact reparameterization of (\ref{eq:survProbDybFull}), since the former contains only two frequencies rather than three. Again, however, Daya Bay cannot resolve $\Delta_{31}$ and $\Delta_{32}$, so there is no practical loss in sensitivty with this approach. The advantage of it is that it produces a single value that is independent of the mass hierarchy and immune to changes wrought by updates to $\Delta m^2{21}$.

$\Delta m^2_{ee}$, as used here, has an \emph{operational,} rather than a physical, definition: It is simply the value that, when inserted into (\ref{eq:survProbDybEE}), gives the best fit to the data. Note that, if Daya Bay only measured antineutrinos at a single $L/E$, we could instead have made a truly physical definition of $\Delta m^2_{ee}$ by declaring that
\[ \sin^2 \Delta_{ee} \equiv \cos^2\theta_{12} \sin^2 \Delta_{31}
  + \sin^2\theta_{12} \sin^2 \Delta_{32}. \]
However, the righthand side of this definition depends on $L/E$, so it cannot be used in broadband analyses such as Daya Bay's.



\section{Reactor neutrinos}
\label{sec:introReactor}

\section{Relevance of $\theta_{13}$ to future research}
\label{sec:futureRelevance}

\end{document}