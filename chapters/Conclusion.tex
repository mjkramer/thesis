\documentclass[../thesis.tex]{subfiles}

\begin{document}

\chapter{Conclusion}
\label{chap:conclusion}

% XXX mention my contributions to Li9.

In \autoref{eq:fitResults}, we gave the best-fit oscillation parameters obtained when using the nominal IBD selection cuts. These values form the basis of our final result. However, in \autoref{sec:cutVaryJoint}, we assessed an additional systematic of 0.0005 ($2\times10^{-5}$~eV$^2$) on $\SinSq$ ($\Dmsqee$), based on the degree of variation observed when the IBD selection cuts were randomly fluctuated. We therefore add this systematic (in quadrature) to the fitter's reported (statistical + systematic) uncertainty of 0.0029 ($7.2\times10^{-5}$~eV) from \autoref{eq:fitResults}, giving our final result:
\begin{equation}
  \SinSq = 0.0853 \pm 0.0030 \qquad \Dmsqee = 2.47\times10^{-3} \pm 7.5\times10^{-5}\;\text{eV}^2
\end{equation}
We regard these uncertainties as comprehensive: They include the effects of statistics (from the toy MC's covariance matrix), signal/background systematics (again, from the toy MC), and the systematic uncertainty (added manually) arising from the flexible nature of the IBD selection criteria. This work represents the most detailed assessment to-date of this final uncertainty, demonstrating that the Daya Bay experiment and oscillation analysis are of sufficiently high quality as to be largely insensitive to cut variations. In the worst case ($\Dmsqee$), the total uncertainty is increased by only 4\% when we account for cut variations. Nevertheless, it would be worthwhile for future Daya Bay results to be verified as stable against changes in IBD cuts; this work thus provides a foundation for further increasing the community's high confidence in the results of Daya Bay.

% In summary, we have presented a detailed description of the Daya Bay oscillation analysis, going step-by-step from raw data to the final oscillation fit. Many individuals have contributed to the various steps in the analysis. Without their efforts, this work would not have been possible. The author's own contributions have focused primarily on calibration, reconstruction, data quality, and event selection. In particular, we have implemented an entirely new event selection framework, which was not only used for the IBD selection in this analysis, but which has also been adopted by at least one other analysis group for their own research. The high performance of our event selection enabled the analysis to be repeated for hundreds of variations of the IBD selection cuts. This allowed for a novel contribution to Daya Bay's results. Namely, we have shown two findings: First, that the best-fit oscillation parameters are stable under variations of the selection criteria, and second, that the error bars are also stable, implying that there is no compelling reason to modify the ad-hoc cuts that have been used in Daya Bay's publications. It is hoped that this work may bolster confidence in Daya Bay's measured oscillation parameters.

% Having measured $\theta_{13}$ (and $\Dmsqee$), we now conclude this thesis by discussing the significance of this measurement in the context of resolving a number of outstanding questions involving the neutrino and our Universe as a whole.


\end{document}