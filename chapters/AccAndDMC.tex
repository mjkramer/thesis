\documentclass[../thesis.tex]{subfiles}

\begin{document}

\chapter{Accidentals rate and DMC efficiency}
\label{chap:accDMC}

These two quantities depend on the rate of uncorrelated physics events, or \emph{singles.} The accidentals rate is determined by the probability of two singles occurring in the same coincidence window, while the efficiency of the decoupled multiplicy cut (DMC) is similarly based on the chance of one or more singles occurring within a certain distance in time from the delayed event.\footnote{IBDs and correlated backgrounds also contribute to the inefficiency of the DMC, but the effect is negligible given the vast disparity in rates between singles and correlated events.}

\def\Emin{\ensuremath{E_\mathrm{min}}}
\def\Emax{\ensuremath{E_\mathrm{max}}}
\def\Rs{\ensuremath{R_\mathrm{s}}}

Let $\Rs(\Emin, \Emax)$ be the rate of singles whose reconstructed energy lies between \Emin\ and \Emax. To be precise, \Rs\ is the \emph{true physical rate} of all \emph{muon-uncorrelated} processes that produce such singles. Naively, one could attempt to calculate \Rs\ by counting all non-muon-vetoed triggers in $(\Emin, \Emax)$ and dividing by the veto-corrected DAQ livetime. However, the rate will then be overestimated due to the inclusion of triggers from correlated events, and, likewise, the predicted accidentals spectrum will be distorted.

Instead, the correct approach is to apply an isolation cut (in time) to ensure a clean sample of true singles. A correction must then be applied for the efficiency of this cut. Once \Rs\ has been obtained in this way, calculation of the accidentals rate and DMC efficiency is a straightforward application of Poisson statistics. We now describe the calculation from end to end.

\section{Singles selection}
\label{sec:singsel}

We begin with a sample of singles, consisting of events that meet the following cuts:

\begin{comment}
  This is the current state of the code:
\begin{enumerate}
\item Not a flasher or forced trigger.
\item Not a muon.
\item AdSimple energy of at least 0.7~MeV.
\item No other events passing cuts 1-3 within time window $\pm t$.
\item Not in a muon veto window.
\end{enumerate}
Below is what it should be. Why the difference? XXX Gotta correct for the lack-of-muon-vetoing-of-other in dmcEffSingles??? Or enhance the SinglesSelector to also pull events from ClusterTree, in case the ``extra'' is vetoed (YES, THIS)? Then our cuts look like:
\end{comment}

\begin{enumerate}
\item Not a flasher or forced trigger.
\item Not a muon, and not in a muon veto window.
\item AdSimple energy of at least 0.7~MeV.
\item No other events passing cuts 1-3 within time window $\pm t$.
\end{enumerate}

The muon veto cuts need not be the same as those used in the IBD selection, as long as they are sufficiently stringent so as to remove muon-correlated backgrounds. (In practice, we use the nominal LBNL muon veto.) Likewise, $t$ can be chosen arbitrarily, provided that it is long enough to remove correlated triggers and short enough to provide sufficient statistics. As currently implemented, a value of $t = 1$ms is used. This is a conservative choice; other analysis groups have successfully used values more than 50\% shorter.

\def\Rplu{\ensuremath{R_\mathrm{+}}}
\def\Rpro{\ensuremath{R_\mathrm{p}}}
\def\Rdel{\ensuremath{R_\mathrm{d}}}
\def\Nplu{\ensuremath{N_\mathrm{+}}}
\def\Npro{\ensuremath{N_\mathrm{p}}}
\def\Ndel{\ensuremath{N_\mathrm{d}}}
\def\eisol{\ensuremath{\epsilon_\mathrm{i}}}
\def\emu{\ensuremath{\epsilon_\mathrm{\mu}}}

\section{Isolation cut efficiency}
\label{sec:isolcuteff}

Let us define a \emph{prompt-plus} single as an uncorrelated trigger with a reconstructed energy of at least 0.7~MeV, but not enough charge/energy to be considered an AD muon. The sample described in the preceding section consists of prompt-plus singles. Similarly, a \emph{prompt-like} single has an energy of between 0.7 and 12~MeV, while a \emph{delayed-like} single has an energy of 6--12~MeV.

For convenience, we define the \emph{plus-like rate} \Rplu\ as $\Rs(0.7,\infty)$ (again, excluding muons), and likewise the \emph{prompt-like rate} \Rpro\ as $\Rs(0.7, 12)$ and the \emph{delayed-like rate} \Rdel\ as $\Rs(6, 12)$. Similarly, the raw event counts in our sample are \Nplu, \Npro, and \Ndel. To complete this round of definitions, let \eisol\ be the isolation cut efficiency, \emu\ the muon cut efficiency (for the singles selection), and $T$ be the raw DAQ livetime of the sample.

As a starting point, we have the following identity:
\begin{equation}
  \label{eq:ident0}
  \Nplu = \emu \eisol \Rplu T,
\end{equation}
in which \emu\ and $T$ are known, and \eisol\ and \Rplu\ are not.

In addition, the Poisson distribution implies that
\begin{equation}
  \label{eq:ident1}
  \eisol = e^{-2\Rplu t}.
\end{equation}
This is simply the probability of observing zero plus-like events in a time window of length $t$. The factor of 2 arises from the fact that an empty window must exist both \emph{before and after} the event.

Combining these two equations and eliminating \eisol, we have:
\begin{equation}
  \Nplu = \emu e^{-2\Rplu t} \Rplu T,
\end{equation}
which, after some rearrangement, gives:
\begin{equation}
  (-2 \Rplu t) e^{-2 \Rplu t} = - \frac{2 \Nplu t}{\emu T}.
\end{equation}
This equation takes the form $we^w = z$, which cannot be solved for $w$ (or, in our case, \Rplu) using elementary functions. Instead we employ the Lambert $W$ function, defined as the inverse of the map $w \mapsto we^w$ (that is, $W(we^w) = w$). As shown in Fig.~\ref{fig:lambertW}, the $W$ function has two branches. We know that, for a 1~ms isolation window, $\Rplu t \ll 1$, implying that the correct choice is the upper branch $W_0$. Then
\begin{equation}
  \Rplu = -\frac{1}{2t}\, W_0 \left(-\frac{2\Nplu t}{\emu T}\right)
\end{equation}

\begin{figure}
  \centering
  \includegraphics[scale=0.7]{../images/lambertW.png}
  \caption{The two branches of the Lambert $W$ function}
  \label{fig:lambertW}
\end{figure}

Finally, inserting this into Eq.~\ref{eq:ident0}, we obtain the isolation cut efficiency:
\begin{equation}
  \eisol = \exp W_0 \left( -\frac{2\Nplu t}{\emu T} \right).
\end{equation}

\end{document}