\documentclass[../thesis.tex]{subfiles}

\begin{document}

\chapter*{Accidentals rate and DMC efficiency}
\stepcounter{chapter}
\label{chap:accDMC}

\def\Emin{\ensuremath{E_\mathrm{min}}} \def\Emax{\ensuremath{E_\mathrm{max}}}
\def\Rs{\ensuremath{R_\mathrm{s}}} \def\Rplu{\ensuremath{R_\mathrm{+}}}
\def\Rpro{\ensuremath{R_\mathrm{p}}} \def\Rdel{\ensuremath{R_\mathrm{d}}}
\def\Rsub{\ensuremath{R_\mathrm{\lambda}}} \def\Nplu{\ensuremath{N_\mathrm{+}}}
\def\Npro{\ensuremath{N_\mathrm{p}}} \def\Ndel{\ensuremath{N_\mathrm{d}}}
\def\eisol{\ensuremath{\epsilon_\mathrm{i}}}
\def\emu{\ensuremath{\epsilon_\mathrm{\mu}}}
\def\etot{\ensuremath{\epsilon_\mathrm{tot}}}
\def\Racc{\ensuremath{R_\mathrm{acc}}}

These two quantities depend on the rate of uncorrelated physics events, or
\emph{singles.} The accidentals rate is determined by the probability of two
singles occurring in the same coincidence window, while the efficiency of the
decoupled multiplicy cut (DMC) is similarly based on the chance of one or more
singles occurring within a certain distance in time from the delayed
event.\footnote{IBDs and correlated backgrounds also contribute to the
  inefficiency of the DMC, but the effect is negligible given the vast disparity
  in rates between singles and correlated events.}

Let $\Rs(\Emin, \Emax)$ be the rate of singles whose reconstructed energy lies
between \Emin\ and \Emax. To be precise, \Rs\ is the \emph{true physical rate}
of all \emph{muon-uncorrelated} processes that produce such singles. Naively,
one could attempt to calculate \Rs\ by counting all non-muon-vetoed triggers in
$(\Emin, \Emax)$ and dividing by the veto-corrected DAQ livetime. However, the
rate will then be overestimated due to the inclusion of triggers from correlated
events, and, likewise, the predicted accidentals spectrum will be distorted.

Instead, the correct approach is to apply an isolation cut (in time) to ensure a
clean sample of true singles. A correction must then be applied for the
efficiency of this cut. Once \Rs\ has been obtained in this way, calculation of
the accidentals rate and DMC efficiency is a straightforward application of
Poisson statistics. We now describe the calculation from end to end.

\section{Event classes}
\label{sec:accdmcevtcls}

Let us define a \emph{muon-like} event as an AD trigger with charge of at least
3,000 p.e. (about 18~MeV).\footnote{It would be simpler to define a muon-like
  event directly in terms of energy instead of charge, but presently we use
  3,000 p.e. for consistency with the historical LBNL analysis. This may change.
  Also, we don't call these events ``muons'', as 3,000 p.e. is not necessarily
  the definition of an AD muon used in the IBD selection.} Then a
\emph{sub-muon} event is one with reconstructed energy of at least 12~MeV, but
not enough charge to be muon-like. A \emph{prompt-like} event has energy of
0.7--12~MeV, a \emph{delayed-like} event 6--12~MeV, and, finally, a
\emph{prompt-plus} event is one that is either prompt-like or a sub-muon. The
sample described in \autoref{sec:singsel} consists of prompt-plus events.

Similarly, we define the \emph{prompt-like rate} \Rpro\ as $\Rs(0.7, 12)$, the
\emph{delayed-like rate} \Rdel\ as $\Rs(6, 12)$, the \emph{sub-muon
  rate}\footnote{$\lambda$ precedes $\mu$ in the Greek alphabet... I think.}
\Rsub\ as $\Rs(12, \mu)$ (i.e., up to the muon-like charge threshold), and the
\emph{prompt-plus rate} \Rplu\ as $\Rpro + \Rsub$. Likewise, the raw event
counts in our sample are \Npro, etc. To complete this round of definitions, let
\eisol\ be the isolation cut efficiency, \emu\ the muon cut efficiency (for the
singles selection), and $T$ be the raw DAQ livetime of the sample.

\section{Singles selection}
\label{sec:singsel}

We begin with a sample of singles, consisting of events that meet the following
cuts:

\begin{comment}
  This is the current state of the code:
  \begin{enumerate}
  \item Not a flasher or forced trigger.
  \item Not a muon.
  \item AdSimple energy of at least 0.7~MeV.
  \item No other events passing cuts 1-3 within time window $\pm t$.
  \item Not in a muon veto window.
  \end{enumerate}
  Below is what it should be. Why the difference? XXX Gotta correct for the
  lack-of-muon-vetoing-of-other in dmcEffSingles??? Or enhance the
  SinglesSelector to also pull events from ClusterTree, in case the ``extra'' is
  vetoed (YES, THIS)? We really need to think about whether the events we're
  ``isolating from'' are in the same class as ``this event''. If so thend our
  cuts look like(?):
\end{comment}

\begin{enumerate}
\item Not a flasher or forced trigger.
\item Not in a muon veto window.
\item AdSimple energy of at least 0.7~MeV.
\item Charge of less than 3,000 p.e. (i.e., not muon-like)
\item No other events passing the above cuts within time window $\pm t$.
\end{enumerate}

The muon veto conditions need not be the same as those used in the IBD
selection, as long as they are sufficiently stringent (as explained in
\autoref{sec:avoidmuoncorr}). Likewise, $t$ can be chosen arbitrarily, provided
that the window is large enough to remove correlated triggers and short enough
to provide sufficient statistics. As currently implemented, a value of $t =
1$~ms is used. This is a conservative choice; other analysis groups have
successfully used values more than 50\% shorter.

\section{Muon veto considerations}
\label{sec:muonventoconsider}

\subsection{Avoiding muon-correlated events}
\label{sec:avoidmuoncorr}

In the selection criteria described in \autoref{sec:singsel}, the final one (the
isolation cut) requires ``no other events passing the above cuts within time
window $\pm t$''. Since ``the above cuts'' includes ``not in a muon veto
window'', this implies that an extra \emph{muon-correlated} trigger will
\emph{not} lead to event rejection by the isolation cut.

This formulation of the isolation cut is necessary for the mathematical
consistency of the calculation of its efficiency: We will be assuming that the
events we are measuring---muon-uncorrelated singles---belong to the exact same
class as the ``extra'' events that enter the definition of the isolation cut.

However, at first glance this poses a problem: Our goal is to measure
muon-uncorrelated events, but if we allow a potential ``single'' to be preceded
by a muon-correlated trigger, then the ``single'' may itself be the product of
muon activity.

Fortunately, this problem can be easily eliminated by an appropriate choice of
the muon veto window. In particular, we need to ensure that, if there is a
muon-correlated event lying inside the isolation window preceding a singles
candidate, the same muon \emph{will veto our candidate}. Previous studies have
shown that the majority of muon-induced activity occurs within 400~$\mu$s of the
muon. Therefore, if we veto at least $t$ + 400~$\mu$s after a muon (where $t$ is
the length of the isolation cut window), the presence of a muon-correlated
trigger guarantees that the candidate will get rejected, albeit by the muon
veto, not by the isolation cut. During singles selection, therefore, we are
careful to use muon veto windows that are sufficiently long.

\subsection{Decoupling of efficiencies}
\label{sec:effdecoup}

\def\Pa{\ensuremath{P_\mathrm{a}}} \def\Pb{\ensuremath{P_\mathrm{b}}}
\def\Pab{\ensuremath{P_\mathrm{ab}}}

In the calculations that follow, we assume that
\begin{equation}
  \label{eq:effdecoup}
  \etot = \emu \eisol,
\end{equation}
that is, that we can calculate the muon veto and isolation cut efficiencies
independently, and get the total efficiency by multiplying the two. This
assumption is valid only if the two occurrences---either a muon, or an ``extra''
single---are statistically independent. As was explained above, extra
muon-correlated triggers are \emph{not} considered when applying the isolation
cut. Therefore, the presence of a muon has no effect on the probability of
passing the isolation cut, and, conversely, the presence of an extra
uncorrelated single has no effect on the probability of being inside a muon veto
window. The requirements for statistical independence are thus satisfied.

As a final verification of \eqref{eq:effdecoup}, let us divide rejected events
into three non-overlapping categories: (a), events rejected only by the muon
veto, (b), events rejected by the isolation cut, and (ab), events rejected by
both. Since the two cuts are independent, $\Pab = \Pa\Pb$. We have:
\begin{align*}
  \etot &= 1 - \Pa - \Pb - \Pab \\
        &= 1 - (1-\emu)\,\eisol - \emu\,(1-\eisol) - (1-\emu)(1-\eisol) \\
        &= \emu \eisol
\end{align*}
Clearly, \eqref{eq:effdecoup} indeed applies. The muon veto efficiency is
determined simply by keeping track of the total vetoed livetime, while the
isolation cut efficiency is described next.

% XXX 200 us or 400 us or...?

\section{Isolation cut efficiency}
\label{sec:isolcuteff}

As a starting point, we have the following identity:
\begin{equation}
  \label{eq:ident0}
  \Nplu = \emu \eisol \Rplu T,
\end{equation}
in which \Nplu, \emu, and $T$ are known, and \eisol\ and \Rplu\ are not. In
addition, the Poisson distribution implies that
\begin{equation}
  \label{eq:ident1}
  \eisol = e^{-2\Rplu t}.
\end{equation}
This is simply the probability of observing zero prompt-plus events in two time
windows (one preceding, one following) each of length $t$. Combining these two
equations and eliminating \eisol, we have:
\begin{equation}
  \Nplu = \emu e^{-2\Rplu t} \Rplu T,
\end{equation}
which, after some rearrangement, gives:
\begin{equation}
  (-2 \Rplu t) e^{-2 \Rplu t} = - \frac{2 \Nplu t}{\emu T}.
\end{equation}
This equation takes the form $we^w = z$, which cannot be solved for $w$ (or, in
our case, \Rplu) using elementary functions. Instead we employ the Lambert $W$
function, defined as the inverse of the map $w \mapsto we^w$ (that is, $W(we^w)
= w$). As shown in \autoref{fig:lambertW}, the $W$ function has two branches.
We know that, for a 1~ms isolation window, $\Rplu t \ll 1$, implying that the
correct choice is the upper branch $W_0$. Then
\begin{equation}
  \Rplu = -\frac{1}{2t}\, W_0 \left(-\frac{2\Nplu t}{\emu T}\right).
\end{equation}
Finally, inserting this into \eqref{eq:ident1}, we obtain the isolation cut
efficiency:
\begin{equation}
  \eisol = \exp \, W_0 \left( -\frac{2\Nplu t}{\emu T} \right).
\end{equation}

TODO: Describe generalization to case where different time periods and/or energy
ranges are used before/after the event. Hint: Take the two coupled equations,
move all exponentials to the left side, add the two equations.

\begin{figure}
  \centering \includegraphics[scale=0.7]{../images/lambertW.png}
  \caption{The two branches of the Lambert $W$ function}
  \label{fig:lambertW}
\end{figure}

\section{Singles rates}
\label{sec:singratescalc}

Once the isolation cut efficiency is known, it is trivial to calculate the
prompt- and delayed-like singles rates:
\begin{equation}
  \Rpro = \frac{\Npro}{\eisol \emu T},\qquad
  \Rdel = \frac{\Ndel}{\eisol \emu T},
\end{equation}
and so on.

\section{DMC efficiency}
\label{sec:dmceffcalc}

\def\edmc{\ensuremath{\epsilon_\mathrm{m}}}

In order for an IBD candidate to satisfy the decoupled multiplicity cut, there
must be no delayed-like triggers in the 200~$\mu$s following the delayed event,
and no extra prompt-plus events\footnote{Formerly, in the LBNL analysis, only
  extra \emph{prompt}-like events would lead to a DMC rejection. In order to
  eliminate the double neutron background, ``prompt-like'' was later changed to
  ``prompt-plus''.} in the 400~$\mu$s prior to the delayed event. More formally,
\begin{align}
  \edmc &= P(0; \Rplu \cdot 2\tau)\;P(0; \Rdel \cdot \tau) \nonumber \\
        &= \exp [-(2\Rplu\tau + \Rdel\tau)],
\end{align}
where $\tau$ = 200~$\mu$s.

\section{Accidentals rate}
\label{sec:accratecalc}

An accidental event, in order to enter the IBD selection, must pass the DMC, as
with any other IBD candidate. Since the DMC is ``centered'' on the delayed event
of the pair, it is simplest to calculate the accidentals rate by similarly
centering the calculation on the delayed event.

Given an uncorrelated delayed-like trigger, we can calculate the probability
that it forms the delayed half of an accidental IBD candidate. Letting all time
intervals be defined in relation to the delayed-like trigger, there are four
conditions:
\begin{enumerate}
\item Exactly one prompt-like event in [-200, 0]~$\mu$s.
\item No sub-muon events in [-200, 0]~$\mu$s.
\item No prompt-plus (prompt-like or sub-muon) events in [-400, -200]~$\mu$s.
\item No delayed-like events in [0, 200]~$\mu$s.
\end{enumerate}
The accidentals rate is then simply the product of these probabilities with the
delayed-like singles rate:
\begin{align}
  \Racc &= \Rdel \cdot P(1; \Rpro \tau)\cdot P(0; \Rsub \tau)
          \cdot P(0; \Rplu \tau)\cdot P(0; \Rdel \tau) \nonumber \\
        &= \Rdel\cdot\Rpro\tau e^{-\Rpro\tau}\cdot e^{-\Rsub\tau}
          \cdot e^{-\Rplu\tau} \cdot e^{-\Rdel\tau}
\end{align}

Note that this calculation incorporates the inefficiency of the DMC (but not of
the muon veto cut). By convention, the oscillation fitter expects background
rates to be provided as ``theoretical'' rates, that is, the rate expected if all
cuts were perfectly efficient. Internally, the fitter multiplies these rates by
the DMC and veto efficiencies. Therefore, we must divide out the DMC efficiency
when providing the accidentals rate to the fitter:
\begin{equation}
  \Racc' = \frac{\Racc}{\edmc}.
\end{equation}

\section{Conclusions}
\label{sec:accdmcconcl}

The most subtle part of this calculation is the efficiency of the isolation cut.
I hope that there isn't any circular logic lurking in my derivation.

\end{document}
