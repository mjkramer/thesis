\documentclass[../thesis.tex]{subfiles}

\begin{document}

\chapter{Variation of IBD selection cuts}
\label{chap:cutVary}

\section{Introduction}
\label{sec:cutVaryIntro}

The IBD selection described in \autoref{chap:selection} involves a number of numerical parameters, or \emph{cuts,} which must be specified. For the official IBD selection, these cuts were essentially chosen arbitrarily at an early stage in the experiment's history, although heuristic arguments were used to justify the claim that the chosen values would provide reasonable signal statistics and background levels. Some of the cuts differed between different analyses, which nonetheless produced consistent results, thus providing some indication of the robustness of the oscillation analysis with respect to variations in the cuts. Still, prior to this work, a systematic investigation of cut variations had never been undertaken, leaving open the question: Are Daya Bay's reported values of $\SinSq$ and $\Dmsqee$ sensitive to the precise values of the IBD selection cuts, and if so, to what extent? Naturally, any such sensitivity would need to be included as an additional source of systematic uncertainty. In this chapter, which constitutes a novel contribution to the high-level Daya Bay analysis, we demonstrate that Daya Bay's result is indeed insensitive to reasonable variations of the cuts.

Aside from the stability of the best-fit oscillation parameters as functions of the cuts, an additional consideration is the size of the uncertainty reported by the fitter. In principle, as the cuts are varied, there will be changes in the balance of efficiencies, raw statistics, and background rates, all of which can influence the size of the final error bars. It is therefore possible that the official IBD selection may be suboptimal in terms of minimizing the final uncertainty, and thus a secondary goal of this study is to determine whether an alternative ``optimal'' set of cuts can be found. If so, the final uncertainty will be improved, and if not, we will have shown a further aspect of the analysis's robustness.

In total, there are 12 cut parameters that enter the IBD selection:

\begin{itemize}
\item Minimum and maximum prompt energy
\item Minimum and maximum delayed energy
\item Minimum and maximum prompt-delayed time separation
\item Water pool muon charge threshold and veto time
\item AD muon charge threshold and veto time
\item Shower muon charge threshold and veto time
\end{itemize}

We do not separately consider the parameters of the decoupled multiplicity cut (DMC), as these are fixed by the prompt/delayed energy and time separation cuts. Among the 12 cuts listed above, most will, upon varying, have minimal impact on the analysis. The maximum prompt/delayed energy of 12~MeV is well above the endpoint of both spectra. Likewise, there are very few events at the time-separation limits of 1 and 200~\us. The shower muon veto window is three orders of magnitude longer than that of the WP and AD muon vetos; thus, as long as the latter two vetos can identify muons with $\sim$100\% efficiency while vetoing sufficiently long to remove prompt cosmogenic events, they are ``good enough''. This leaves four cuts which may have a significant effect upon the analysis: The shower muon threshold and window, the minimum delayed energy, and the minimum prompt energy. In subsequent sections, we discuss the reasons to expect these four cuts to be potentially impactful. Aside from these four cuts, we will also explore the effects of applying a vertex (i.e., position) cut, a nonstandard addition to the analysis which can provide a further demonstration of its overall robustness.

In order to carry out this study, significant modifications needed to be made to the LBNL oscillation analysis. The original LBNL IBD selection was intended to be run just once per data set, without any variation of the cut parameters, which were hardcoded and scattered throughout the code. As such, a new, general-purpose event processing framework was written from the ground up. The IBD selection was then implemented on top of this framework, with all cut parameters specified just once in an external text file, eliminating any need to recompile the code for each new set of cuts. As opposed to the previous IBD selection, which ran in a single pass, the new one takes a two-stage approach, with an initial, cut-independent pre-selection stage that performs a data reduction on the full processed Daya Bay data files, extracting only those events and variables of potential interest to the IBD selection. (Flashers and nonphysical triggers are removed at this stage, along with the vast majority of event variables, leaving only the reconstructed charge and energy, trigger time, and reconstructed vertex.) The second stage then applies a given set of IBD cuts to perform the selection. Computational efficiency was a primary consideration in the development of this code, in order to enable the reprocessing of data with many different cuts.

In addition to the IBD selection, the fitter also required a significant amount of work, although in this case a ground-up rewrite was not necessary. The only consideration here was computational efficiency, since each IBD selection must be followed by its own fit. Originally, the full fitting chain (including the toy Monte Carlo) took a couple of hours to run on NERSC's Cori cluster. After aggressively parallelizing the code, both by running independent steps in parallel processes, and by using OpenMP to harness multithreaded data parallelism within each process, the full chain was reduced to a runtime of around ten minutes. These changes, combined with the rewrite of the IBD selection, are what made this study possible.

\section{Muon veto efficiency calculation}
\label{sec:cutVaryMuVetoEff}

In order to use the LBNL toy Monte Carlo for the purpose of predicting and validating the data-driven optimization of the shower muon veto, we must provide the overall muon veto efficiency as an input to the toy MC (and subsequently to the fitter). During the actual data-driven optimization, the veto efficiency is essentially obtained ``for free'', as one of the outputs of the IBD selection. In principle, then, for the toy MC study we could simply run the IBD selection in order to obtain the efficiency for a given definition of the veto. However, it would be ideal for the toy MC cross check to be as decoupled as possible from the IBD selection and its associated data. Thus, we seek an independent method of determining the muon veto efficiency.

The veto efficiency is, naturally, a function of the muon rate. In particular, it depends on the rates of the three types of muons: water pool, AD, and shower. The first step, then, is to measure these rates, while taking care to avoid double counting due to retriggers and multi-detector muons. Once the rates have been obtained, the next step is to calculate the efficiency, which requires a proper treatment of the possible overlap of veto windows between closely spaced muons. In what follows we describe these two steps in detail and demonstrate that the results agree sufficiently well with the ``true'' veto efficiency obtained from the IBD selection.

\subsection{Muon rate measurement}
\label{sec:cutVaryMuRate}

The objective of the muon rate measurement is to obtain, for each AD, two results: First, the rate of \emph{water pool-only} muons (i.e., those muons which produced a ``WP muon'' trigger without an associated ``AD muon'' trigger), and second, the spectrum (in terms of charge or energy) of those muons that triggered the AD. Given that the AD/shower veto windows are longer than the WP window, any AD+WP muon should be regarded as a single AD muon. The measurement of the AD muon \emph{spectrum}, rather than the total rate, is important because it enables a breakdown into ``normal'' and ``shower'' AD muons, which carry different veto windows, and it makes this breakdown possible while varying the shower muon definition, without requiring a re-run of the muon selection.



\begin{comment}
  XXX local slides from mid-late Oct for retrigger plots. See misc_ana/MuonVetoEff/condenser4retrig.
\end{comment}

\end{document}
