\documentclass[../thesis.tex]{subfiles}

\begin{document}

\chapter{Event selection}
\label{chap:selection}

From the sequence of reconstructed triggers in the ADs, we are primarily interested in extracting IBD candidates, in order to obtain the antineutrino rate and spectrum. The tight time correlation of the prompt and delayed triggers, as well as the relatively high 8~MeV energy of the nGd capture peak, enable the extraction of a $\sim$98\% pure sample of IBDs, from which the independently estimated backgrounds can then be subtracted.

Aside from the IBD selection, this analysis also employs an extraction of \emph{singles,} that is, those events that produce only a single trigger, uncorrelated in time with any others. The purpose of the singles sample is to enable determination, firstly, the rate and spectrum of backgrounds produced by accidental coincidences, and secondly, the efficiency of the multiplicity cut (discussed in \autoref{sec:pairSel}).

Both selections are implemented using a two-stage approach. In the first stage, the \emph{pre-selection,} the reconstructed Daya Bay DAQ files are scanned, unimportant events are ignored, and of the remaining events, only the minimum required data fields are stored in the output. This process reduces $\sim$600,000 reconstructed DAQ files (each representing $\sim$10 minutes), totaling some 600~TB, down to about 5,500 files (each representing one hall $\times$ day), totaling one terabyte. In the second stage, the \emph{final selection,} the full set of selection criteria are applied to the pre-selected data, producing samples of IBDs and singles for use in the oscillation fit. This two-stage approach significantly reduces the amount of time needed to generate new IBD/singles samples after modifying the selection criteria, since the pre-selection does not need to be re-run. When the NERSC cluster is not under severe disk I/O load, the two-stage approach provides a speed improvement of 3 to 4; during disk overload, the improvement can be greater still.

\section{IBD selection}
\label{sec:selIBDs}

We begin by discussing the IBD selection. The singles selection proceeds similarly, with minor differences in the final steps, as discussed in \autoref{sec:selSingles}.

\subsection{Pre-selection}
\label{sec:selPreSel}

\subsubsection{Input data}
\label{sec:selInputData}

The processed (i.e. calibrated and reconstructed) Daya Bay DAQ files (in ROOT format) serve as the input to the pre-selection. Although these files contain basic taggings of muon-like events and coincidence clusters, this information is not used here; our event selection is a completely independent implementation.

Two ROOT TTrees are read in parallel: the \texttt{AdSimple} tree, which contains the reconstructed energy, and the \texttt{CalibStats} tree, which contains the nominal charge (used, in some cases, for pre-muon identification), the number of hit PMTs (used for identifying pre-muons in the water pool), and various quantities that are used for removing instrumental backgrounds. Both trees are of the same length, with one entry per trigger, including triggers in the water pools and RPCs (for which AD-specific quantities are left blank). Being the same length, the two trees can be ``friended'' together (in ROOT parlance) and scanned as one. Other fields loaded from this combined tree are the detector ID, the trigger type, the trigger ID, and the trigger time.

\subsubsection{Trigger type restriction}
\label{sec:selTrigType}

The very first criterion applied in the pre-selection is a restriction on the type of triggers saved. In particular, six types of triggers are excluded: manual triggers, cross triggers (issued when a trigger occurs in a different detector subsystem), periodic and random triggers (used for subthreshold and noise measurements), pedestal triggers (used in special runs to measure the ADC pedestals), and calibration triggers (issued, for instance, when a calibration LED is pulsed). The remaining events consist solely of \texttt{NHit} and \texttt{ESum} triggers, issued (respectively) when the number of hit channels or the total measured charge, within the previous $\sim\mu$ s, is above a configured threshold. The NHit threshold is 45, while the ESum threshold (in units roughly, but not exactly, analogous to the nominal charge) is 100, 107, and 130 in EH1, EH2, and EH3. These thresholds were determined during commissioning in order to ensure perfect trigger efficiency at the IBD prompt energy threshold of 0.7~MeV, without overwhelming the trigger rate.

\subsubsection{Pre-muon selection}
\label{sec:selPreMuons}

After removing unwanted trigger types, the next step in the pre-selection is to extract muon-like events to an output tree of \emph{pre-muons.} The actual definition of a muon (for the purpose of applying the muon veto) is applied in stage two; the pre-muon criteria are thus designed to be loose enough to encompass any final muon definition. An event in the water pool is considered a pre-muon if the number of hit PMTs is more than 12, and an AD pre-muon is defined as having a reconstructed energy of more than 20~MeV (XXX).\footnote{In previous versions of the pre-selection, AD pre-muons were defined in terms of nominal charge rather than energy, with a cut at 3,000 p.e. This was done for the sake of consistency with the original LBNL IBD selection, which defined muons in terms of charge rather than energy. However, the uniform use of energy simplifies matters somewhat, so this departure was made from the original selection.} Each pre-muon was stored with its trigger time and its \emph{strength}, either its energy (for AD triggers) or its hit multiplicity (for WP triggers).

\subsubsection{Flasher removal}
\label{sec:selFlashers}

Among the remaining non-pre-muon events, some are \emph{flashers,} instrumental backgrounds produced by arcing within the base of the PMTs. The light from these arcs can illuminate the detector, resulting in a trigger. As detailed in \autoref{sec:bkgFlashers}, these events can be easily distinguished by their characteristic conical pattern of light emission. The three cuts described in \autoref{sec:bkgFlashers} (the \emph{ellipse,} \emph{PSD,} and \emph{2" PMT} cuts) are used to identify and remove flashers from the output.

\subsubsection{Saving and merging}
\label{sec:selMergingOne}

Finally, the non-pre-muon, non-flasher triggers are saved in their own tree (one for each AD), separate from the pre-muon tree. For each event, the run number, file number, trigger time, trigger number, and energy are saved. A minimum reconstructed energy of 0.7~MeV (the threshold for prompt-like triggers) is applied here to further reduce the data volume, since lower-energy events are not considered in this analysis.

The pre-selection files are initially produced in one-to-one correspondence with the reconstructed DAQ files, resulting in $\sim$600,000 small files. To reduce the file count and improve IO performance, these files are merged (using ROOT's \texttt{hadd}) into files that each represent one calendar day in one hall, a total of $\sim$5,500 files. Finally, these daily files are pre-loaded into an SSD \emph{burst buffer} at NERSC, to ensure that the performance of the final selection will be minimally impacted by disk load conditions at the facility.

\subsection{Final selection}
\label{sec:selFinalSel}

The specific thresholds for the cuts described in this section are \emph{nominal}; they are defined somewhat arbitrarily based on qualitative observations and notions of reasonableness, intended to give a satisfactory ratio of signal to background. Later, in \autoref{chap:cutOptim}, we study the effects of varying these cuts, with the aim of jointly optimizing both the uncertainty on the oscillation parameters as well as the stability of the fit with respect to variations in the cuts. Doing so will eliminate the arbitrariness inherent in the cuts described here.

\subsubsection{Muon veto}
\label{sec:selMuonVeto}

When a muon passes through or near the AD, it can produce triggers in the aftermath. These can include instrumentally-induced triggers (caused by PMT afterpulsing and electronics ringing in the 20~\us following the muon), as well as physical events. The physical events can include spallation neutrons, whose thermalization and subsequent capture can mimic an IBD, isotopes such as $^9$Li, and $^8$He, which produce neutrons when they decay, and various uncorrelated decays that can form accidental IBD-like pairs.

For this reason, it is essential to veto the time period immediately following a muon. Although the mean neutron capture time in the GdLS is only 28~\us, neutrons can be produced outside the GdLS and slowly diffuse into it, necessitating a significantly longer veto window. In the case of WP muons, a veto time of 600~\us was shown to effectively remove all such neutrons. Only relatively energetic, i.e., fast, neutrons would have the ability to reach the GdLS; slow diffusion, meanwhile, is not a significant possibility for WP muons. Meanwhile, for AD muons, neutrons \emph{can} diffuse slowly from the LS or mineral oil, leading to the requirement of a veto window closer to a millisecond. The nominal window for this case is 1.4~ms, a factor of $\sim$7 larger than the mean neutron capture time (mainly on hydrogen) in the LS region. AD muons are (XXX) nominally defined as those triggers having an energy of at least 20~MeV.

Muons that deposit an especially high amount of energy in the AD are termed \emph{shower} muons. Compared to lower-energy (i.e., minimum ionizing) muons, shower muons have a much higher probability of producing the two cosmogenic isotopes $^9$Li and $^8$Be, discussed further in \autoref{sec:bkgCosmo}. These isotopes undergo beta decay, producing a prompt-like trigger, and then break up to produce neutrons which are captured, producing a delayed-like trigger. Given that the lifetime of these isotopes is of order 100~ms, the ordinary AD muon veto of around one millisecond would fail to significantly reduce these backgrounds. Accordingly, a much longer veto window, on the order of one second, is needed after a shower muon. The nominal window here is 1~s, with shower muons defined as having an energy of at least 2~GeV. At this threshold energy, the rate of $^9$Li/$^8$He production is low enough to avoid backgrounds from sub-threshold muons, while the rate of such muons themselves is also low enough to avoid too large of a loss in effective detector livetime. These qualititative statements will be made quantitative in \autoref{chap:cutOptim}, where optimal thresholds and windows will be determined.

An additional veto window is applied, spanning 2~\us \emph{before} each muon, common to WP, AD, and shower muons. Given that trigger latencies can vary, it is possible for a WP muon trigger to receive a trigger timestamp that comes \emph{after} the timestamps for muon-induced events in the AD. The 2~\us pre-veto eliminates this possibility. There is no particular need to veto the 2~\us preceding an \emph{AD} muon, but the original LBNL IBD selection did so anyway, and we honor its legacy by following suit.

During application of the muon veto, the total amount of vetoed time is tracked, accounting for overlaps. This value is used in order to calculate the muon veto efficiency, determined (on a daily basis) simply as the ratio of unvetoed DAQ livetime to total DAQ livetime. It should be noted that the muon veto is only applied to the \emph{delayed} trigger of a coincidence pair, and this is what enables the efficiency to be calculated so simply. Otherwise, we would require a complex calculation involving the prompt-delayed time distribution and the muon rate. Given (as described below) that the maximum time difference between the prompt and delayed event is 200~\us, it is possible for the prompt trigger to lie 200~\us before the end of the veto window. The windows are thus made large enough to ensure that this time period is free of muon-correlated activity.

\subsubsection{Pair selection}
\label{sec:pairSel}

A potential IBD candidate may be lurking whenever there is a non-vetoed delayed-like trigger (i.e., one lying between 6 and 12~MeV). Specifically, the following conditions (known as the \emph{decoupled}\footnote{The meaning here of ``decoupled'' is explained further in \autoref{chap:accDMC}.} \emph{multiplicity cut}, or DMC) determine the existence of an IBD candidate:

\begin{enumerate}
\item There is a prompt-like (i.e. 0.7--12~MeV) trigger between 1 and 200~\us before this delayed-like trigger
\item There are no other triggers of more than 0.7~MeV between 1 and 400~\us before this delayed-like trigger.\footnote{Originally, this condition was framed in terms of \emph{prompt-like}, rather than $>$0.7~MeV triggers, but the permitted ``extra'' events above 12~MeV (i.e. low-energy muons) were found to be correlated with backgrounds consisting of pairs of neutron captures. The modified condition eliminates this background.}
\item There are no delayed-like triggers within 200~\us after this one.
\end{enumerate}

The purpose of the latter two conditions is to avoid the ambiguity that can arise, for instance, in the contrived example of three 7~MeV events spaced 100~\us apart. Here there are three possible ways to form a pair. There are other possible ways to define cuts that would avoid this ambiguity, for instance, by defining ``empty'' windows relative to the prompt trigger, but the DMC allows for a simple calculation of the efficiency as well as an avoidance of correlations with the muon veto efficiency, as described in \autoref{chap:accDMC}.

IBD candidates that pass the DMC are stored in an output tree containing the run and file number, the prompt-delayed time difference, and the IDs and energies of the two triggers.

\subsubsection{Merging and post-processing}
\label{sec:selMergingTwo}

After the $\sim$5,500 hall-daily files have been fully produced, they are merged with \texttt{hadd} into nine files, the product of three halls and three periods (the 6AD, 8AD, and 7AD periods, whose names reference the number of operating ADs). The splitting into periods is done for convenience is preparing the input files required by the fitter, which expects separate files for each period. The fitter's input files, which consist of the IBD spectra, the accidentals spectra, and textual tables of rates, efficiencies, backgrounds, uncertainties, etc., are prepared by a simple script from these nine hall-period files.

\section{Singles selection}
\label{sec:selSingles}

The singles selection proceeds in a similar manner to the IBD selection, except that pairs are no longer being selected. Instead, when a non-muon trigger of at least 0.7~MeV is found, an \emph{isolation cut} is applied, eliminating those events for which another 0.7~MeV trigger lies within specified windows before and after the ``singles candidate''. As implemented, the windows used are 400~\us before the trigger and 200~\us after, chosen to resemble the DMC as closely as possible. In principle, the exact sizes of these windows should not matter (provided they are wide enough to eliminate correlated multiplets), as the efficiency of the isolation cut is corrected for in calculating the DMC efficiency and accidental background rate. In practice, due to correlated low-energy processes such as alpha-alpha cascades, the singles sample does not fully consists of ``true'' singles, so the choice of DMC-like time windows is a naive attempt to minimize any resulting biases. Ultimately, the uncertainty of the accidental background rate is inflated beyond the statistical uncertainty in order to account for this problem.

\end{document}