\documentclass[../thesis.tex]{subfiles}

\begin{document}

\chapter{Event reconstruction}
\label{chap:recon}

\section{Introduction}

\autoref{chap:calib} discussed the process of taking the raw ADC and TDC values
of PMT hits, as measured by the front-end electronics, and converting those
values, channel-by-channel into the more useful quantities of hit charge (in
photoelectrons) and hit time (in, e.g., nanoseconds). The next step is to
combine information from all of the channels in order to derive properties of
the event as a whole, such as the amount of deposited energy and the approximate
location of the vertex. This is the purpose of \emph{reconstruction.}

Reconstruction begins with the calculation of the total observed charge
(i.e. photoelectron count) by summing all (XXX?) relevant hits across all
channels, with a correction for the presence of any inactive channels. This
\emph{nominal charge} is then converted into \emph{visible energy}, in MeV,
according to an energy scale determined using regular (weekly or more)
calibrations. At the same time, the distribution of charge across PMTs is used
to estimate the location within the AD of the event. The position is used to
apply a \emph{nonuniformity} adjustment to the visible energy, to correct for
the position-dependent response of the detector. This gives the
\emph{reconstructed energy}.

The reconstructed energy should not be regarded as a best estimate of the true
energy deposited by the event, given the complexities involved in the
nonlinearity of the scintillator and its varying reponses to different particle
types. Rather, reconstructed energy should be considered a
\emph{position-corrected measure of the total observed amount of light}. Due to
the calibration methods used, however, reconstructed energy agrees (by
construction) with deposited energy for the 8~MeV gamma cascade from nGd
capture., which is used in most subsequent analysis stages.

Daya Bay has developed multiple independent reconstruction algorithms. The two
that have been widely used in published results are known as AdSimple and
AdScaled. They differ primarily in their calibration procedures, their vertex
reconstruction algorithms, and their methods of correcting for
nonuniformity. Both will be detailed in this chapter, but only AdSimple will be
used in our analysis.

\section{Energy reconstruction}
\label{sec:reconEnergy}

\subsection{Event charge determination}
\label{sec:reconEnergyChage}

Active PMT correction, etc.

\subsection{Energy scale calibration}
\label{sec:reconEnergyScale}

\subsection{Nonuniformity correction}
\label{sec:reconEnergyNU}

\subsubsection{Time-dependent nonuniformity}
\label{sec:reconEnergyTDNU}

\subsection{Nonlinearity correction}
\label{sec:reconEnergyNL}

\section{Vertex reconstruction}
\label{sec:reconVertex}

\end{document}
