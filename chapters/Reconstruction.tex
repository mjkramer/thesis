\documentclass[../thesis.tex]{subfiles}

\begin{document}

\chapter{Event reconstruction}
\label{chap:recon}

\section{Introduction}
\label{sec:reconIntro}

\autoref{chap:calib} discussed the process of taking the raw ADC and TDC values of PMT hits, as measured by the front-end electronics, and converting those values, channel-by-channel, into the more useful quantities of hit charge (in photoelectrons) and hit time (in, e.g., nanoseconds). The next step is to combine information from all of the channels in order to derive properties of the event as a whole, such as the amount of deposited energy and the approximate location of the vertex. This is the purpose of \emph{reconstruction.}

Reconstruction begins with the calculation of the total observed charge (i.e. photoelectron count) by summing hits across all channels, with a correction for the presence of any inactive channels. This \emph{nominal charge} is then converted into \emph{visible energy}, in MeV, according to an energy scale determined using regular (weekly or more) calibrations. At the same time, the distribution of charge across PMTs is used to estimate the location within the AD of the event. The position is then used to apply a \emph{nonuniformity} adjustment to the visible energy, to correct for the position-dependent response of the detector. This gives the \emph{reconstructed energy}, which is used in most subsequent analysis stages.

The reconstructed energy should not be regarded as a best estimate of the true energy deposited by the event, given the complexities involved in the nonlinearity of the scintillator and its varying responses to different particle types. Rather, reconstructed energy should be considered a \emph{position-corrected measure of the total observed amount of light}, and hence should be regarded as proportional to the total amount of light produced in the scintillator. Due to the calibration methods used, reconstructed energy \emph{does} agree (by construction) with deposited energy for the 8~MeV gamma cascade from nGd capture, but this is only a special case.

Daya Bay has developed multiple independent reconstruction algorithms. The two that have been widely used in published results are known as AdSimple and AdScaled. They differ primarily in their calibration procedures, their vertex reconstruction algorithms, and their methods of correcting for nonuniformity. Both give consistent results in the oscillation analysis, and both will be detailed in this chapter, but only AdSimple will be used in our analysis.

\section{Energy reconstruction}
\label{sec:reconEnergy}

\subsection{Event charge determination}
\label{sec:reconEnergyCharge}

\subsubsection{Hit selection}
\label{sec:reconHitSelection}

The first step in the energy reconstruction is to estimate the total \emph{charge}, i.e., number of photoelectrons, observed from the underlying interaction. Here, the main consideration is the choice of hits to include in the sum. Based on the design of the trigger electronics, a trigger will be issued about \SI{1550 \pm 50}{ns} after the hits are registered by the discriminator/TDC circuit. This would imply that a window of around [-1650, -1450]~ns would be reasonable. In practice, Daya Bay actually uses a window of [-1650, -1250]~ns. The justification for this wider window is related to the properties of the liquid scintillator itself.

When an interaction deposits ionization energy in the LS, various molecular excited states decay stochastically, emitting light in the process. In the Daya Bay LS, the light emission can be accurately modeled with three components: a fast one ($\sim$5~ns time constant), a medium one ($\sim$30~ns), and a slow one ($\sim$150~ns). The time for light to propagate, directly or via reflections, adds a position-dependent delay of a few dozen ns. Altogether, 5\% of PMT hits occur some 50-150~ns after the primary peak (XXX doc-8732). In order to include this ``late'' light, and thereby hopefully improve the energy resolution, Daya Bay uses the widened hit selection window of [-1650, -1250]~ns.

With a window defined for hit selection, the next question is which hits to use from inside this window. Based on the measures discussed in \autoref{sec:calibHitCharge} for correcting the biases in closely-spaced hits, in principle every hit should be trustworthy. In practice, hits that arrive within 100~ns of each other will produce a single shaped peak, and hence only the first hit will have a nonzero calibrated charge. Since most primary light hits \emph{do} in fact arrive within 100~ns of each other, there is usually no difference between taking all hits and taking only the first hit. The \emph{default} or \emph{nominal} charge is accordingly defined as \emph{the sum across channels of the earliest hit in the time window of [-1650, -1250]~ns (relative to the trigger time).}

The nominal charge will generally account for all of the fast/medium light, but will omit \emph{some} of the slow light \emph{unless there is no fast/medium light seen by the channel}. As such, high-energy events will miss a greater proportion of slow light compared to low-energy events, since in the latter case there will be more channels seeing no fast/medium light. This introduces a degree of nonlinearity in the overall detector response. If, instead, one were to take \emph{all} hits in [-1650, -1250]~ns, instead of just the earliest hit, the sum would in principle accurately include all of the components, without the aforementioned nonlinearity. This does not appear to have ever been proposed; the reasons are unknown, but may be related to the fact that this method is more sensitive to the details of the corrections for closely-spaced hits.

However, there \emph{was} an alternative to the nominal charge that was discussed for a time in the earlier days of the experiment. The \emph{peak charge} was defined as the sum across channels of the earliest hit in [-1650, -1480]~ns. This time window effectively excludes the late light, and thus mitigates the associated nonlinearity found in the nominal charge. One (perhaps insignificant) downside of the peak charge is that it includes slightly less light, but late light accounts for only 5\% of the total, and the nominal charge misses some of it anyway, so the overall loss of photon statistics is only on the order of a couple percent. Most likely, in the author's opinion, is that the nominal charge was retained primarily due to inertia. In any case, it is possible to measure and correct for any nonlinearity inherent in the charge calculation, as discussed in \autoref{sec:reconEnergyNL}, so there is a degree of latitude in choosing from among these various methods.

\subsubsection{Active channel correction}
\label{sec:reconActiveChan}

At any given time, there may be dead or malfunctioning channels in an AD. As described in \autoref{sec:calibCQ}, these are recorded in the channel quality (CQ) database according to a number of criteria. If, at the time of a given trigger, a channel is marked as ``bad'', then its charge is \emph{not} included in the total nominal charge. This, naturally, will result in a downward bias on the total. In principle, the size of the effect depends on the position of the event: The effect is larger if the event is closer to the PMT, and vice versa. In practice, however, Daya Bay uses a simple, position-independent correction of $192/N$, where $N$ is the number of active channels. Given that the Daya Bay ADs almost always have fewer than two bad channels, this correction was found to reliably correct the bias, with negligible impact on the resolution.

\subsubsection{Summary}
\label{sec:reconChargeSummary}

In summary, the nominal charge is computed as follows: For every active channel, take the calibrated charge of the earliest hit in the window of [-1650, -1250]~ns pre-trigger. Sum these up, and then apply a correction of $192/N$, where $N$ is the number of active channels. In subsequent stages of the energy reconstruction, the nominal charge (in PE) is scaled by a time-dependent energy scale to give the \emph{visible energy} (in MeV), then adjusted by a time- and position-dependent nonuniformity correction to give the \emph{reconstructed energy} and, finally, at the highest levels of analysis, adjusted again to correct for electronics nonlinearity, scintillator nonlinearity, and IBD kinematics to give the \emph{true neutrino energy}. These steps are discussed below.

\subsection{Energy scale calibration}
\label{sec:reconEnergyScale}

The nominal charge produced by a given interaction can vary over time due to, for instance, degradation or contamination of the scintillator. Furthermore, for the purpose of physics analysis, we would prefer to speak of the energy (in, e.g., MeV) deposited in the scintillator, rather than the amount of light observed. Accordingly object of the energy scale calibration is to fix the definition of a ``visible'' MeV, and to ensure that any given event will yield the same visible energy in every AD, regardless of changes over time in the behavior of the scintillator.

Given that the response of the ADs (i.e. the nominal charge) depends on the type of interaction and is nonlinear with respect to the deposited energy, the energy scale (in charge per MeV) will depend on the choice of interaction used to calibrate the scale. Daya Bay's two main reconstruction algorithms, AdSimple and AdScaled, both define the energy scale such that a neutron capture on gadolinium will yield approximately 8 MeV (XXX precision). However, there are significant differences between the methodology of the two calibrations.

For AdSimple, the calibration uses Gd captures of spallation neutrons produced by high-energy cosmic muons traversing the AD. Since this thesis's analysis is based on AdSimple, we give a detailed description of its calibration procedure in the section that follows. One of the advantages of using spallation neutrons is that they are distributed uniformly throughout the target volume, similarly to IBD neutrons. A disadvantage is that the ensuing energy scale is slightly biased (upward), relative to that of IBD neutrons, due to PMT afterpulsing resulting from the large amount of charged produced by the parent muon. In the end, however, this is accounted for in the nonlinearity model (\autoref{sec:reconEnergyNL}); as long as the energy scale calibration provides consistency in time, space, and between ADs, it is sufficient.

In comparison to AdSimple, AdScaled uses a significantly different method of calibrating the energy scale. We only discuss it briefly, since AdScaled is not used in this analysis. Essentially, the method is based on using weekly $^{60}$Co calibrations to monitor the time variation of the light yield, and occasional $\sim$40-hour AmC calibrations (which produce nGd captures) to measure the nonlinearity between the $^{60}$Co and nGd peaks. Every Friday, the $^{60}$Co source is deployed at the center of each AD for 10 minutes, and the charge peak is fit to a Gaussian (XXX). The peak nominal charge is then multiplied by the ratio between the nGd and $^{60}$Co charge peaks, as determined by the nearest long AmC run, and this scaled light yield is stored in the database for use by the reconstruction. This method works because the \emph{ratio} of the nGd and $^{60}$Co peaks is quite stable, even when the peaks themselves are varying. (Omitting $^{60}$Co, and using AmC alone, would avoid the need for this scaling, but the rate of neutrons from the AmC source is insufficient to provide the necessary statistics.) It is worth noting that the resulting energy scale is defined in terms of events at the \emph{center} of the AD, rather than uniformly distributed throughout the GdLS (as in AdSimple). This leads to consistent a $\sim$5\% difference in the energy scale calibration constants between the two algorithms. Essentially, this is only a difference in conventions (i.e., defining the energy scale based on uniformly distributed vs. centered events), which is accounted for at the event-by-event level by the nonuniformity correction, as discussed in \autoref{sec:reconEnergyNU}.

\begin{comment}
  A sample enriched in such neutrons is obtained by selecting events in a time window (XXX define) immediately after AD muons (XXX of what minimum energy?). These captures are distributed uniformly throughout the GdLS, much like IBDs. The nGd capture peak in the charge distribution is fit to a Gaussian (XXX crystal ball?), and the location of the peak is defined as corresponding to 7.95 MeV (XXX) 8.0 MeV according to doc-7334 (AdSimple). This energy scale is stored in the offline database, valid for the period in which the neutrons were collected. In the near (far) halls, it takes XXX (YYY) days to obtain the necessary statistics; this is thus the time-resolution of the energy scale, which is sufficient, given that the light yield changes very slowly, declining by some 1\% to 1.5\% per year.
  
24 hours
\end{comment}

\begin{comment}
  Figure out exactly what energy is pegged by AdSimple and AdScaled. 7.95 MeV? Discuss differences (e.g. due to muon afterpulsing?)
  5x15min Co60
  4x10hour AmC
\end{comment}

\subsubsection{AdSimple calibration procedure}
\label{sec:reconEnergyAdSimpleCalib}

The AdSimple energy calibration begins with the selection of a sample of spallation neutron candidates. These are defined based on their proximity in time to a preceding AD muon, where an AD muon is regarded as any event that produces more than 3,000 photoelectrons of nominal charge. Non-muon events are filtered through a simplified cut to remove instrumental backgrounds (``flashers''); specifically, the \emph{ellipse cut} described in \autoref{sec:bkgFlashers} is employed. For any surviving event with a nominal charge of more than 100 PE (roughly 0.6~MeV), the energy is added either to a \emph{signal} histogram, if the time since the previous muon is between 20 and \SI{1000}{\micro s}, or to a \emph{background} histogram if $\Delta t$ is between 1020 and \SI{2000}{\micro s}. Given that the characteristic nGd capture time is $\sim$\SI{30}{\micro s}, the latter histogram provides a clean sideband measurement of the background spectrum from muon-uncorrelated events.

\begin{comment}
  Note: For AdSimpleNL, in reconstruction, a (AD-specific?) scale constant is applied to the non-NL energy scale constant. See line 209 of QsumEnergyTool.cc. Discuss this?
\end{comment}

\subsection{Nonuniformity correction}
\label{sec:reconEnergyNU}

\subsubsection{Time-dependent nonuniformity}
\label{sec:reconEnergyTDNU}

\subsection{Nonlinearity correction}
\label{sec:reconEnergyNL}

\section{Vertex reconstruction}
\label{sec:reconVertex}

\end{document}