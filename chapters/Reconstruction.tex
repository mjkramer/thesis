\documentclass[../thesis.tex]{subfiles}

\begin{document}

\chapter{Event reconstruction}
\label{chap:recon}

\section{Introduction}

\autoref{chap:calib} discussed the process of taking the raw ADC and TDC values
of PMT hits, as measured by the front-end electronics, and converting those
values, channel-by-channel, into the more useful quantities of hit charge (in
photoelectrons) and hit time (in, e.g., nanoseconds). The next step is to
combine information from all of the channels in order to derive properties of
the event as a whole, such as the amount of deposited energy and the approximate
location of the vertex. This is the purpose of \emph{reconstruction.}

Reconstruction begins with the calculation of the total observed charge
(i.e. photoelectron count) by summing hits across all channels, with a
correction for the presence of any inactive channels. This \emph{nominal charge}
is then converted into \emph{visible energy}, in MeV, according to an energy
scale determined using regular (weekly or more) calibrations. At the same time,
the distribution of charge across PMTs is used to estimate the location within
the AD of the event. The position is then used to apply a \emph{nonuniformity}
adjustment to the visible energy, to correct for the position-dependent response
of the detector. This gives the \emph{reconstructed energy}, which is used in
most subsequent analysis stages.

The reconstructed energy should not be regarded as a best estimate of the true
energy deposited by the event, given the complexities involved in the
nonlinearity of the scintillator and its varying responses to different particle
types. Rather, reconstructed energy should be considered a
\emph{position-corrected measure of the total observed amount of light}, and
hence should be regarded as proportional to the total amount of light produced
in the scintillator. Due to the calibration methods used, reconstructed energy
\emph{does} agree (by construction) with deposited energy for the 8~MeV gamma
cascade from nGd capture, but this is only a special case.

Daya Bay has developed multiple independent reconstruction algorithms. The two
that have been widely used in published results are known as AdSimple and
AdScaled. They differ primarily in their calibration procedures, their vertex
reconstruction algorithms, and their methods of correcting for
nonuniformity. Both give consistent results in the oscillation analysis, and
both will be detailed in this chapter, but only AdSimple will be used in our
analysis.

\section{Energy reconstruction}
\label{sec:reconEnergy}

\subsection{Event charge determination}
\label{sec:reconEnergyCharge}

\subsubsection{Hit selection}
\label{sec:reconHitSelection}

The first step in the energy reconstruction is to estimate the total
\emph{charge}, i.e., number of photoelectrons, observed from the underlying
interaction. Here, the main consideration is the choice of hits to include in
the sum. Based on the design of the trigger electronics, a trigger will be
issued about \SI{1550 \pm 50}{ns} after the hits are registered by the
discriminator/TDC circuit. This would imply that a window of around [-1650,
-1450]~ns would be reasonable. In practice, Daya Bay actually uses a window of
[-1650, -1250]~ns. The justification for this wider window is related to the
properties of the liquid scintillator itself.

When an interaction deposits ionization energy in the LS, various molecular
excited states decay stochastically, emitting light in the process. In the Daya
Bay LS, the light emission can be accurately modeled with three components: a
fast one ($\sim$5~ns time constant), a medium one ($\sim$30~ns), and a slow one
($\sim$150~ns). The time for light to propagate, directly or via reflections,
adds a position-dependent delay of a few dozen ns. Altogether, 5\% of PMT hits
occur some 50-150~ns after the primary peak (XXX doc-8732). In order to include
this ``late'' light, and thereby hopefully improve the energy resolution, Daya
Bay uses the widened hit selection window of [-1650, -1250]~ns.

With a window defined for hit selection, the next question is which hits to use
from inside this window. Based on the measures discussed in
\autoref{sec:calibHitCharge} for correcting the biases in closely-spaced hits,
in principle every hit should be trustworthy. In practice, hits that arrive
within 100~ns of each other will produce a single shaped peak, and hence only
the first hit will have a nonzero calibrated charge. Since most primary light
hits \emph{do} in fact arrive within 100~ns of each other, there is usually no
difference between taking all hits and taking only the first hit. The
\emph{default} or \emph{nominal} charge is accordingly defined as \emph{the sum
  across channels of the earliest hit in the time window of [-1650, -1250]~ns
  (relative to the trigger time).}

The nominal charge will generally account for all of the fast/medium light, but
will omit \emph{some} of the slow light \emph{unless there is no fast/medium
  light seen by the channel}. As such, high-energy events will miss a greater
proportion of slow light compared to low-energy events, since in the latter case
there will be more channels seeing no fast/medium light. This introduces a
degree of nonlinearity in the overall detector response. If, instead, one were
to take \emph{all} hits in [-1650, -1250]~ns, instead of just the earliest hit,
the sum would in principle accurately include all of the components, without the
aforementioned nonlinearity. This does not appear to have ever been proposed;
the reasons are unknown, but may be related to the fact that this method is more
sensitive to the details of the corrections for closely-spaced hits.

However, there \emph{was} an alternative to the nominal charge that was
discussed for a time in the earlier days of the experiment. The \emph{peak
  charge} was defined as the sum across channels of the earliest hit in [-1650,
-1480]~ns. This time window effectively excludes the late light, and thus
mitigates the associated nonlinearity found in the nominal charge. One (perhaps
insignificant) downside of the peak charge is that it includes slightly less
light, but late light accounts for only 5\% of the total, and the nominal charge
misses some of it anyway, so the overall loss of photon statistics is only on
the order of a couple percent. Most likely, in the author's opinion, is that the
nominal charge was retained primarily due to inertia. In any case, it is
possible to measure and correct for any nonlinearity inherent in the charge
calculation, as discussed in \autoref{sec:reconEnergyNL}, so there is a degree
of latitude in choosing from among these various methods.

\subsubsection{Active channel correction}
\label{sec:reconActiveChan}

At any given time, there may be dead or malfunctioning channels in an AD. As
described in \autoref{sec:calibCQ}, these are recorded in the channel quality
(CQ) database according to a number of criteria. If, at the time of a given
trigger, a channel is marked as ``bad'', then its charge is \emph{not} included
in the total nominal charge. This, naturally, will result in a downward bias on
the total. In principle, the size of the effect depends on the position of the
event: The effect is larger if the event is closer to the PMT, and vice
versa. In practice, however, Daya Bay uses a simple, position-independent
correction of $192/N$, where $N$ is the number of active channels. Given that
the Daya Bay ADs almost always have fewer than two bad channels, this correction
was found to reliably correct the bias, with negligible impact on the
resolution.

\subsubsection{Summary}
\label{sec:reconChargeSummary}

In summary, the nominal charge is computed as follows: For every active channel,
take the calibrated charge of the earliest hit in the window of [-1650,
-1250]~ns pre-trigger. Sum these up, and then apply a correction of $192/N$,
where $N$ is the number of active channels. In subsequent stages of the energy
reconstruction, the nominal charge (in PE) is scaled by a time-dependent energy
scale to give the \emph{visible energy} (in MeV), then adjusted by a time- and
position-dependent nonuniformity correction to give the \emph{reconstructed
  energy} and, finally, at the highest levels of analysis, adjusted again to
correct for electronics nonlinearity, scintillator nonlinearity, and IBD
kinematics to give the \emph{true neutrino energy}. These stages are discussed
below.

\subsection{Energy scale calibration}
\label{sec:reconEnergyScale}

\subsection{Nonuniformity correction}
\label{sec:reconEnergyNU}

\subsubsection{Time-dependent nonuniformity}
\label{sec:reconEnergyTDNU}

\subsection{Nonlinearity correction}
\label{sec:reconEnergyNL}

\section{Vertex reconstruction}
\label{sec:reconVertex}

\end{document}
