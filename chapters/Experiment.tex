\documentclass[../thesis.tex]{subfiles}

\begin{document}

\chapter{The Daya Bay experiment}
\label{chap:experim}

\section*{Introduction}

The Daya Bay experiment was designed to measure $\theta_{13}$ by observing the antineutrinos produced by the six 2.9~GW$_{\text{th}}$ nuclear reactors of the Daya Bay and Ling Ao power plants, located near Shenzhen in southern China. A total of eight functionally identical antineutrino detectors (ADs) were deployed, each containing a target of 20~tons of gadolinium-doped liquid scintillator (GdLS). Four of the ADs were evenly divided among two near halls ($\sim$350-600~m baselines from the cores), and the remaining four were placed in a single far hall ($\sim$1500-1950~m baselines). Shielding from cosmic rays was provided by $\sim$100~m and $\sim$300~m, respectively, of mountainous overburden at the near and far halls. The ADs in each hall were immersed in instrumented water pools which provided shielding from ambient radioactivity and detection of Cerenkov radiation from atmospheric muons. Redundant detection of muons, as well as directional information, were made available by resistive plate chambers (RPCs) laid on top of the water pools. In this chapter we discuss further details of the layout, the detectors, and the shielding and vetoing system of the experiment.

\section{Site layout}
\label{sec:expLayout}

\begin{figure}[ht]
  \includegraphics[scale=0.4]{inkLayout.pdf}
  \caption{The layout of the Daya Bay experiment. Modified from \cite{SideBySide}.}
  \label{fig:layout} 
\end{figure}

As shown in \autoref{fig:layout}, the power reactors are divided into three nuclear power plants (NPPs) of two cores each. One of the two clusters contains the Daya Bay NPP (cores D1 and D2), while the other cluster consists of the Ling Ao (L1 and L2) and Ling Ao-II (L3 and L4) NPPs. EH1 is located around 350~m from the Daya Bay NPP, while EH2 is roughly 500~m from the two Ling Ao NPPs. The far hall, EH3, in turn is located about 1900~m from the Daya Bay NPP and 1500~m from the Ling Ao NPPs. The measured baselines, as determined from a combined GPS and total station theodolite survey, are given in \autoref{tab:expBaselines}. The uncertainty of $\sim$2~cm in these measurements was shown to have negligible impact on the analysis; likewise, reactor simulations determined that the centroid of $\nubar_e$ emission to be within $\sim$2~cm of each core's center. Although $\nubar_e$ emission was distributed across the $\sim$3~m-scale volume of each core, the analysis treats each core as a point source, given that these geometric effects are negligible at Daya Bay's baselines.

\begin{table}[ht]
  \begin{tabular}{lcrrrrrr}
    \toprule
    \multicolumn{2}{c}{} & \multicolumn{6}{c}{Reactor baseline [m]} \\
    \cmidrule{3-8}
    Hall & Detector & \multicolumn{1}{c}{D1} & \multicolumn{1}{c}{D2} & \multicolumn{1}{c}{L1} & \multicolumn{1}{c}{L2} & \multicolumn{1}{c}{L3} & \multicolumn{1}{c}{L4} \\
    \midrule
    EH1  & AD1      & 362.38  & 371.76  & 903.47  & 817.16  & 1353.62 & 1265.32 \\
         & AD2      & 357.94  & 368.41  & 903.35  & 816.90  & 1354.23 & 1265.89 \\
    EH2  & AD3      & 1332.48 & 1358.15 & 467.57  & 489.58  & 557.58  & 499.21  \\
         & AD8      & 1337.43 & 1362.88 & 472.97  & 495.35  & 558.71  & 501.07  \\
    EH3  & AD4      & 1919.63 & 1894.34 & 1533.18 & 1533.63 & 1551.38 & 1524.94 \\
         & AD5      & 1917.52 & 1891.98 & 1534.92 & 1535.03 & 1554.77 & 1528.05 \\
         & AD6      & 1925.26 & 1899.86 & 1538.93 & 1539.47 & 1556.34 & 1530.08 \\
         & AD7      & 1923.15 & 1897.51 & 1540.67 & 1540.87 & 1559.72 & 1533.18 \\
    \bottomrule
    % \multirow{2}{*}{EH1} & AD1      & 362.38  & 371.76  & 903.47  & 817.16  & 1353.62 & 1265.32 \\
  \end{tabular}
  \caption{Baselines between geometric centers of the ADs and of the reactor cores. From \cite{An_2017}.}
  \label{tab:expBaselines}
\end{table}

\section{Antineutrino detectors}
\label{sec:expADs}

\begin{figure}[ht]
  \includegraphics[scale=0.4]{exp_AD_structure.pdf}
  \caption{Structure of a Daya Bay antineutrino detector. From \cite{An_2017}.}
  \label{fig:expDetector}
\end{figure}

The design of the Daya Bay ADs is shown in \autoref{fig:expDetector}. Each AD is made of a cylindrical stainless steel vessel (SSV), $\sim$5~m in diameter and height, containing two nested cylinders of UV-transparent acrylic. The 3m-square inner acrylic vessel (IAV) contains the target mass of 20~t of GdLS\footnote{Daya Bay's liquid scintillator consists of linear alkyl benzene (LAB) as the solvent and medium, 3~g/L of 2,5-diphenyloxazole (PPO) as the fluor, and 15~mg/L of p-bis-(o-methystyril)-benzene (bis-MSB) as the wavelength shifter. For the GdLS, 0.1\% $^{\text{nat}}$Gd by mass was added in the form of a complex with 3,5,5-trimethylhexanoic acid (THMA). Further details can be found in \cite{Beriguete_2014}.}. Surrounding it is the 4m-square outer acrylic vessel (OAV), which contains 20~t of \emph{Gd-free} LS. This ``gamma catcher'' volume ensures the full containment and measurement of gammas produced near the edge of the GdLS, while also providing additional target mass for studies (including oscillation  fits) that make use of neutron capture on hydrogen instead of on gadolinium. Between the OAV and the inner wall of the SSV, a 37~t volume of transparent mineral oil (MO) provides shielding from radioactivity in the detector materials, in addition to its role in balancing the stress on the OAV wall.

Within the MO volume, the inner sidewall of the SSV supports 192 8-inch Hamamatsu R5912 photomultipler tubes (PMTs) to detect the light from scintillation. The PMTs are arranged in eight rings of 24 tubes whose photocathodes protrude from matte-black radial shields that fully cover the sidewalls, preventing light from reflecting off the walls. This simplifies the optical characteristics of the ADs, reducing the complexity of vertex reconstruction. Conversely, however, reflective discs are installed at the top and bottom of the AD, improving both the energy resolution and the uniformity of light collection versus event position. To further reduce nonuniformity effects, each PMT is outfitted with a FINEMET conical magnetic shield to minimize azimuthal variations in PMT response caused by the Earth's magentic field.

Each PMT is positively biased via a single coaxial cable in order to achieve a gain within 5\% of 10$^7$. Due to intrinsic differences between PMTs, the necessary high voltage varies between some 1300 to 1700~kV. Collected charge is passed through a passive decoupling circuit which removes the HV offset; this fast ($\sim$20~ns), unbiased pulse is then passed to the front-end electronics (FEE), where it is split and sent to two separate circuits. One of the circuits contains a $\sim$0.25-photoelectron (pe) discriminator, which initiates a TDC counter (of 1.6~ns resolution) to record the presence and time of the ``hit''. The other circuit is a CR-(RC)$^4$ shaper which stretches each pulse to a length of $\sim$200~ns; the shaped pulse is then split and sent to both a x10 ``high-gain'' amplifier and a x0.5 ``low-gain'' attenuator (for neutrino-like and muon-like events, respectively) and, finally, the two shaped and rescaled pulses are sampled by a 40~MHz 12-bit ADC. The output of a hardware-based peak-finding algorithm is then recorded as the amplitude (i.e. collected charge) of the pulse, for both the high-gain and low-gain circuits.

Although every hit is initially observed in this manner by the hardware, it is only recorded in the DAQ's output stream if a trigger is issued for the AD as a whole. An AD can be triggered both when the total observed charge is above a software-specified threshold, as well as when the number of ``hit'' channels is above threshold. Each FEE board, which reads up to 16 channels, sends two signals to the AD's \emph{local trigger board} (LTB): a digital count of recently-hit channels (NHIT) and an analog sum of the charge across all channels (ESUM). The LTB combines the inputs from all of the FEEs and issues a trigger when either NHIT or ESUM are above threshold; for ordinary physics data-taking, the thresholds are NHIT $\geq$ 45 or ESUM $\geq$ 65 pe ($\sim$0.4~MeV). When a trigger is issued, a GPS-synchronized clock (25 ns resolution) records the overall event timestamp, each hit's TDC is stopped (indicating the offset of each hit relative to the event timestamp), and the readout (including the TDC, high-gain ADC, and low-gain ADC for each hit within 1.2~$\mu$s of the trigger) is sent to the DAQ system.

Thus, the raw information collected from each AD consists of a set of triggers. Each trigger, in turn, contains a timestamp and a collection of \emph{hits}; each hit describes the PMT ID, the TDC count, and the high- and low-gain ADC values reported by the peak-finding circuitry. In order to be made useful for physics analysis, the raw ADC values must first be converted into photoelectron counts for each PMT, as described in \autoref{chap:calib}, and then the individual PMTs must be combined and corrected to produce the amount of energy deposited in the scintillator, as elaborated in \autoref{chap:recon}.

The vast majority of events recorded by the ADs are unrelated to neutrinos. Most events come from natural radioactivity in the detector and scintillator materials, as well as from cosmic muons and the byproducts of muon-nucleon reactions. Fortunately, it is possible to exploit the double-trigger nature of neutrino events in order to effectively extract them from the data, as discussed in the section that follows.

\subsection{Detection principle}
\label{sec:expDetPrinc}

Antineutrinos can interact with the GdLS target in a number of ways. Their interactions can be mediated either by the $W^-$ boson (corresponding to a \emph{charged current}, or CC, interaction), in which case the antineutrino becomes a positron, or they can be mediated by the $Z$ boson (corresponding to a \emph{neutral current}, or NC, interaction), in which case the antineutrino escapes from the detector after depositing some recoil energy in the target. Furthermore, the interaction may take place between the antineutrino and either an electron, a proton, a neutron, or (coherently) an entire nucleus. Among these many possibilities, however, only one channel provides a signature that allows for efficient discrimination from background: inverse beta decay (IBD),
\begin{equation*}
  \nubar_e + p \rightarrow e^+ + n.
\end{equation*}
This interaction is illustrated in \autoref{fig:expIBD}. An initial \emph{prompt} scintillation signal is produced by the positron (first from direct ionization, then from the gammas produced by annihilation). Meanwhile, the free neutron thermalizes and is then captured by a nucleus, which quickly de-excites, emitting gamma rays that provide a second, \emph{delayed,} signal, closely correlated in time with the prompt signal. This ``double-pulse'' signature effectively distringuishes IBDs from events produced by natural radioactivity, as the latter largely consists of single pulses. Meanwhile, muon-induced double-pulse events can be eliminated by simply vetoing for a sufficient length of time after each muon, as described in \autoref{chap:selection}. The use of gadolinium further improves background separation, both by reducing the neutron capture time constant from $\sim$200~$\mu$s to $\sim$20$\mu$s, and by increasing the total energy of the delayed gammas from 2.2 to 8~MeV. These two properties significantly reduce the probability of accidentally identifying a pair of uncorrelated events as an IBD, especially since the rate of uncorrelated events drops dramatically above 5~MeV. Thanks to gadolinium, such ``accidental'' backgrounds made up only 1-2\% of the selected IBD sample. Further details regarding background measurement and subtraction can be found in \autoref{chap:bkg}.

\begin{figure}[ht]
  \includegraphics[scale=0.4]{ibd.png}
  \caption{An illustration of the inverse beta decay reaction. Unlike a water Cerenkov detector, a Daya Bay AD cannot discern the direction of the positron. From \cite{Fernandez_2017}.}
  \label{fig:expIBD}
\end{figure}

\section{Shielding/vetoing system}
\label{sec:expShieldVeto}

\begin{figure}[ht]
  \includegraphics[scale=0.4]{exp_near_site_diagram.pdf}
  \caption{Water pool (including ADs) as configured in the near halls. The far hall is similar, with four ADs instead of two. From \cite{An_2017}.}
  \label{fig:expPool}
\end{figure}

\subfilebackmatter

\end{document}